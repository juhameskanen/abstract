\documentclass[11pt]{article}
% -- Common definitions for all papers

% --- Packages ---
\usepackage[utf8]{inputenc}
\usepackage[T1]{fontenc}
\usepackage{amsmath, amssymb, amsthm}
\usepackage{geometry}
\usepackage{hyperref}
\usepackage{cite}
\newcommand{\pdflink}[1]{} % By default, do nothing

% --- Page Setup ---
\geometry{letterpaper, margin=1in}
\setlength{\parindent}{0pt}
\setlength{\parskip}{1em}


% --- Author ---
\author{Abstract Researcher}

% Glossary
\newcommand{\SSP}{Spectral Selection Principle (SSP)}
\newcommand{\Lcost}{$\mathcal{L}$} % Added $ $ here
\newcommand{\Tobs}{$\mathcal{T}_{\mathrm{obs}}$} % Added $ $ here

\newcommand{\PrintGlossary}{
    \section*{Glossary of the Abstract Universe}
    \begin{description}
    \item[SSP] \SSP: The foundational principle asserting that reality is the subset of informational paths that admit the most efficient representation in the frequency domain. It replaces "laws of physics" with a statistical preference for compressibility.
    
    \item[\Tobs] Observer Filter

    \item[$\mathcal{L}$ (Spectral Encoding Length)] The measure of information required to represent a state or path $\gamma$. In this framework, $\mathcal{L}$ replaces the classical concept of \textit{Action}. Minimizing $\mathcal{L}$ is equivalent to the Principle of Least Action.
    
    \item[$\mathcal{T}_{\mathrm{obs}}$ (Observer Filter)] The subset of all possible configuration sequences capable of describing a subjective experience.
    
    \item[MSM (Microstructure Motif)] Recurring, compressible bit-patterns within configuration space (e.g., particles, fields). The density of these motifs determines the local spectral cost, manifesting as mass-energy.
    
    \item[Nyquist Horizon] The "Planck Scale" of the theory. It is the maximum frequency supportable by the discretization of configuration space. Any structure below this limit is mathematically "aliased" and cannot be rendered.
    
    \item[Induced Time] The sequential ordering of states along a path $\gamma$. An emergent property of the observer's trajectory.
    
    \item[Phase-Coherence] The informational alignment between disparate parts of the spectral encoding. This provides the mathematical basis for what is traditionally called \textit{quantum entanglement}.

    \end{description}
}


\addbibresource{../references.bib}

\title{Methods}

\date{1999}

\begin{document}

\maketitle

\section*{Methods}

\ifdefined\ishtml
\begin{center}
\small \href{methods.pdf}{Download PDF Version}
\end{center}
\fi

\subsection*{Operational Principle}

Building on the ontological equivalence of physical and informational configurations established in Paper I, We adopt the following methodological rule:

\begin{quote}
If a phenomenon can be simulated such that the simulation reproduces the qualitative observables of interest, then the execution trace of that simulation is sufficient to analyze and understand those observables.
\end{quote}

An \emph{execution trace} is the finite bitstring generated by running the simulation: the ordered record of state transitions and logged events relevant to the observables under study. All results in this work are grounded in analysis of such traces.

\subsection*{Ontological Equivalence}

The simulated system and the simulator (hardware and software) are not treated as ontologically distinct entities. They are two structurally different arrangements of the same underlying information, related by a mapping implemented by the simulation code.

Accordingly, time, causality, singularities, and observer continuity are not externally imposed primitives. They must emerge as regularities within the execution trace itself. Whatever is real for observers in the simulation must correspond to structure in that trace.

\subsection*{Trace Sufficiency}

Let $\mathcal{M}$ be a discrete simulation model, $\tau \in \{0,1\}^N$ its execution trace, and $\mathcal{O}$ a specified set of observables. The model is said to reproduce $\mathcal{O}$ if each observable can be recovered from $\tau$ by a computable extraction procedure, up to agreed qualitative invariants or tolerances.

When this condition holds, no additional external description is required. The execution trace is the sole data object needed for explanation, comparison, and prediction with respect to $\mathcal{O}$.

\subsection*{Computability Assumption}

All analysis is assumed to be algorithmic: any effective extraction of observables from the execution trace must be computable. This is a methodological commitment, not a claim about the ultimate nature of physical reality.

Computability does not imply tractability. Some properties may be computationally expensive, irreducible, or undecidable. Such limits are treated as boundaries of analysis rather than failures of the method.

\subsection*{Scope}

This method is explicitly relative to the chosen observables. It does not claim total equivalence between model and target, only equivalence with respect to what is reproduced and extracted. If essential degrees of freedom are not recorded in the trace, they are outside the scope of analysis.

\subsection*{Implications}

Within this framework, execution traces are primary scientific objects. In this work, entropy measures computed over traced subsystems are interpreted as geometric and causal features in the simulated perspective: entropy collapse corresponds to geometric singularity formation, while entropy expansion corresponds to emergent spacetime extension.

Understanding is operationally defined as the ability to extract the relevant observables from the execution trace by explicit, reproducible procedures.


\printbibliography
\end{document}
