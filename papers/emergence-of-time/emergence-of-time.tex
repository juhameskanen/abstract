\documentclass[12pt]{article}
\usepackage{amsmath,amsfonts,amssymb}
\usepackage{hyperref}
\usepackage{authblk}
\usepackage{graphicx}
\usepackage{physics}
\usepackage[backend=biber]{biblatex}
\newcommand{\pdflink}[1]{} % By default, do nothing

\addbibresource{references.bib}

\title{Algorithmic Selection of Time and Quantum Structure \\
from a Static Wheeler--DeWitt Universe}

\author{Juha Meskanen}
\date{December 2025}

\begin{document}
\maketitle

\begin{abstract}  
We extend the informational framework introduced in prior work to show that \textbf{ontological equivalence} between informational configurations inevitably arises from the combinatorial structure of finite bitstrings. A system of \(n\) bits generates \(2^n\) Everettian configurations, all formally equivalent at the ontological level. By defining a relational ordering of these configurations, we identify an emergent \emph{ordinal time} within the otherwise static Wheeler--DeWitt multiverse. Observer-experienced reality is then dominated by histories that minimize algorithmic description length, providing a unified mechanism for the emergence of time, unitarity, quantum probabilities, and the suppression of divergences. This work formalizes the connection between bitstring combinatorics, Everettian branching, and observer-relative relational histories, demonstrating that the flow of time and physical laws are algorithmically selected features of a fundamentally timeless informational universe.  
\end{abstract}

\ifdefined\ishtml
\begin{center}
\small \href{paper.pdf}{Download PDF Version}
\end{center}
\fi


\section{Ontological Framework}

We adopt the canonical formulation of quantum gravity, in which the universal state
\( \Psi[h_{ij}(\mathbf{x}),\phi(\mathbf{x})] \) satisfies the Wheeler--DeWitt (WdW) equation
\begin{equation}
\hat{H}\Psi = 0,
\end{equation}
and contains no fundamental time parameter \cite{DeWitt1967}. We interpret \( \Psi \) in an
Everettian sense as a static superposition over all admissible 3-geometries and matter field
configurations:
\begin{equation}
\ket{\Psi} = \sum_i c_i \ket{h_i,\phi_i}.
\end{equation}

In this view, the universe is a timeless informational structure. Any notion of dynamics must
arise from correlations internal to \( \Psi \), rather than from external temporal evolution.

\section{Algorithmic Selection Principle}

\paragraph{Algorithmic Selection Principle (ASP).}
\emph{Among all relational histories consistent with the universal state \( \Psi \), observer-experienced
reality is dominated by histories whose total description length is minimal.}

Let \( S \) denote a relational history induced by a particular factorization and ordering of
configurations. The relative weight of \( S \) is
\begin{equation}
P(S) \propto 2^{-K(S)},
\end{equation}
where \( K(S) \) is the Kolmogorov complexity (or minimal description length) of \( S \)
\cite{Kolmogorov1965}.

Description length is evaluated \emph{relative to a self-maintaining information-processing
structure}. Histories incompatible with observer persistence are not experienced, regardless of
their formal existence within \( \Psi \).

\section{Emergence of Quantum Structure}

\subsection{Wavefunction as Optimal Encoding}

A finite observer must encode correlations along its experienced history using a representation
that minimizes reconstruction error while preserving predictive power. The complex wavefunction
\( \psi \) emerges as the minimal encoding that supports linear superposition, phase coherence,
and stable composition of subsystems.

Highly irregular configurations require longer descriptions and are suppressed under ASP,
leading to an effective low-frequency, smooth structure.

\subsection{Unitarity from Informational Stability}

Persistence of observer identity requires that successive encodings preserve total information
content. In a linear representation space, this restricts admissible transformations to
norm-preserving maps. Continuous norm preservation uniquely selects unitary evolution, with a
Hermitian generator \( \hat{H} \), yielding
\begin{equation}
i\hbar \frac{\partial}{\partial t}\psi = \hat{H}\psi.
\end{equation}

\section{The Born Rule from Algorithmic Optimality}

We now strengthen the derivation of the Born rule by identifying it as the \emph{unique decoding
rule} compatible with algorithmic selection and observer persistence.

\subsection{Problem Statement}

Let \( \psi \in \mathcal{H} \) be the observer's compressed encoding of relational correlations.
A probability rule assigns to each outcome \( i \) a weight
\begin{equation}
P(i) = f(|\psi_i|),
\end{equation}
where \( f \) is a non-negative function.

We seek the unique \( f \) compatible with the following physically motivated constraints.

\subsection{Constraints}

\paragraph{(C1) Additivity under coarse-graining}
If outcomes \( i \) and \( j \) are grouped,
\begin{equation}
P(i \cup j) = P(i) + P(j).
\end{equation}

\paragraph{(C2) Composition consistency}
For independent subsystems \( A \) and \( B \),
\begin{equation}
P(i_A, j_B) = P(i_A) P(j_B).
\end{equation}

\paragraph{(C3) Basis invariance}
Probabilities must be invariant under unitary change of basis.

\paragraph{(C4) Reconstruction optimality}
The decoding rule must minimize mean squared reconstruction error between encoded and decoded
histories. This is the unique optimal decoder for finite observers using lossy compression.

\subsection{Uniqueness of the Squared Norm}

Constraints (C1) and (C2) imply that \( f(x) \) must be quadratic in amplitude magnitude:
\begin{equation}
f(x) = k x^2.
\end{equation}

Constraint (C3) excludes dependence on phase or basis-dependent quantities.

Constraint (C4) selects the \( L^2 \) norm uniquely: among all \( L^p \) norms, only \( p = 2 \)
yields linear projections, orthogonality preservation, and stable error minimization under
compression.

Normalization fixes \( k = 1 \), yielding
\begin{equation}
P(i) = |\psi_i|^2.
\end{equation}

\paragraph{Conclusion.}
The Born rule is not an independent axiom, but the unique probability assignment compatible with
algorithmic compression, compositional consistency, and observer persistence in Hilbert space.



\subsection{Physical Meaning of the Born Weight}

Multiplicity explains why the squared norm ($P = |\psi|^2$) is the correct measure selected by the optimal decoding rule.

\subsubsection{Lemma: Quadratic Scaling in Linear Encoding}
In any linear wave-based encoding scheme, the number of distinguishable microstates $\Omega$ consistent with a macroscopic wave constraint scales with the squared norm of the amplitude. Following Parseval's Theorem, the total energy (or informational "power") of a signal in Hilbert space is:
\begin{equation}
E = \int | \psi(x) |^2 dx
\end{equation}
In our framework, "Power" is equivalent to the **bandwidth of the execution trace**. A higher amplitude $A$ provides a quadratically larger state-space volume for microstate permutations that remain consistent with the observer's template. Thus, the multiplicity of observer-instances is $M \propto |\psi|^2$.

\subsubsection{The Universal Prior and Self-Locating Uncertainty}
Applying a universal prior (Solomonoff) over the ensemble $\Omega$, the probability $P$ of an observer finding themselves in state $S$ is defined by its Kolmogorov Complexity $\mathcal{K}(S)$:
\begin{equation}
P(S) \propto 2^{-\mathcal{K}(S) + c}
\end{equation}
where $c$ is a constant determined by the choice of universal Turing machine. This represents \textit{self-locating uncertainty}: the observer is not "choosing" a path, but is statistically distributed across all consistent bit-string permutations.

\subsubsection{Derivation of the Born Weight}
By combining the multiplicity of the linear wave-template with the universal prior, the relative measure of observer density across configuration space is:
\begin{equation}
P(\text{Alice} \in \psi) = \frac{|\psi|^2}{\int |\psi|^2 dx}
\end{equation}
This normalization reflects the relative density of observer-instances, not a stochastic collapse of the wavefunction.

\subsubsection{The Vanishing Measure of Divergences}
This derivation provides a natural "soft cutoff" for the infinities of Quantum Field Theory. While divergent states exist mathematically, their description length $\mathcal{K}$ explodes. Since:
\begin{equation}
\lim_{\mathcal{K} \to \infty} 2^{-\mathcal{K}} = 0
\end{equation}
the measure of observers inhabiting divergent branches is zero. Physical laws appear smooth and renormalized because only finite, highly-compressible (low-$\mathcal{K}$) branches possess sufficient multiplicity to be "lived."



\section{Singularities and Renormalization}

Configurations corresponding to classical singularities or ultraviolet divergences possess
vanishing or ill-defined informational structure. Such states require maximal description length
and are exponentially suppressed under ASP. Their observational probability is zero.

Renormalization in quantum field theory arises as a direct consequence: high-frequency modes are
algorithmically incompressible and do not contribute to dominant observer-compatible histories.
Effective field theories are therefore maximal-compressibility approximations to the static
informational substrate.

\section{Discussion}

Spacetime geometry, time, quantum probabilities, and classicality emerge as features of
algorithmically typical correlations within a timeless universal state. Singularities and
divergences mark the boundaries of informational accessibility, not physical breakdowns.

The Algorithmic Selection Principle shifts the explanatory burden from fundamental dynamics to
informational selection, providing a unified resolution of the problem of time, the origin of
quantum probabilities, and the absence of observable infinities.

\printbibliography

\end{document}
