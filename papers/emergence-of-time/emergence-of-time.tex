\documentclass[11pt]{article}
% -- Common definitions for all papers

% --- Packages ---
\usepackage[utf8]{inputenc}
\usepackage[T1]{fontenc}
\usepackage{amsmath, amssymb, amsthm}
\usepackage{geometry}
\usepackage{hyperref}
\usepackage{cite}
\newcommand{\pdflink}[1]{} % By default, do nothing

% --- Page Setup ---
\geometry{letterpaper, margin=1in}
\setlength{\parindent}{0pt}
\setlength{\parskip}{1em}


% --- Author ---
\author{Abstract Researcher}

% Glossary
\newcommand{\SSP}{Spectral Selection Principle (SSP)}
\newcommand{\Lcost}{$\mathcal{L}$} % Added $ $ here
\newcommand{\Tobs}{$\mathcal{T}_{\mathrm{obs}}$} % Added $ $ here

\newcommand{\PrintGlossary}{
    \section*{Glossary of the Abstract Universe}
    \begin{description}
    \item[SSP] \SSP: The foundational principle asserting that reality is the subset of informational paths that admit the most efficient representation in the frequency domain. It replaces "laws of physics" with a statistical preference for compressibility.
    
    \item[\Tobs] Observer Filter

    \item[$\mathcal{L}$ (Spectral Encoding Length)] The measure of information required to represent a state or path $\gamma$. In this framework, $\mathcal{L}$ replaces the classical concept of \textit{Action}. Minimizing $\mathcal{L}$ is equivalent to the Principle of Least Action.
    
    \item[$\mathcal{T}_{\mathrm{obs}}$ (Observer Filter)] The subset of all possible configuration sequences capable of describing a subjective experience.
    
    \item[MSM (Microstructure Motif)] Recurring, compressible bit-patterns within configuration space (e.g., particles, fields). The density of these motifs determines the local spectral cost, manifesting as mass-energy.
    
    \item[Nyquist Horizon] The "Planck Scale" of the theory. It is the maximum frequency supportable by the discretization of configuration space. Any structure below this limit is mathematically "aliased" and cannot be rendered.
    
    \item[Induced Time] The sequential ordering of states along a path $\gamma$. An emergent property of the observer's trajectory.
    
    \item[Phase-Coherence] The informational alignment between disparate parts of the spectral encoding. This provides the mathematical basis for what is traditionally called \textit{quantum entanglement}.

    \end{description}
}


\addbibresource{../references.bib}

\title{Spectral Selection of Time and Quantum Structure \\
from a Static Wheeler--DeWitt Universe}

\date{2024}

\begin{document}
\maketitle

\begin{abstract}  
Building on the ontological equivalence of informational configurations established in Paper I,
we derive a model of the universe based on ordering of these configurations.
We identify an emergent \emph{ordinal time} within the static Wheeler--DeWitt multiverse.
Observer-experienced reality is dominated by histories that minimize \emph{spectral complexity},
defined as the number of independent frequencies and phases required to encode correlations.
Such spectrally minimal paths are combinatorially dominant, providing a unified mechanism for the
emergence of time, unitarity, quantum probabilities, and the suppression of divergences.
\end{abstract}

\ifdefined\ishtml
\begin{center}
\small \href{emergence-of-time.pdf}{Download PDF Version}
\end{center}
\fi

\section{Ontological Framework}

We consider the universe as a static informational object \(U\) consisting of \(n\) bits.
All physically and observer-relevant structures are encoded within this data.
Observers are particular configurations \(O \subset U\) whose internal structure allows
them to encode correlations and maintain identity across ordered configurations.

Since \(U\) is atemporal, time cannot be fundamental.
We define an observer's experience of time as the \emph{ordering of configurations}
that preserves the internal correlations of \(O\).
Let
\begin{equation}
\mathcal{C}(O) = \{\pi : \pi \text{ is a permutation of configurations in } U
\text{ consistent with } O \}
\end{equation}
denote the set of orderings compatible with the observer's internal structure.
Only permutations in \(\mathcal{C}(O)\) correspond to possible temporal experiences for \(O\).

In this sense, the universe is a timeless informational structure.
Dynamics emerge from correlations inherent in static data, as experienced by observers
through specific orderings of configurations.
Arbitrary orderings that destroy these correlations are structurally incompatible
with observer experience.

\section{Emergence of Wave-Like Structure from Spectral Sparsity}

Among all observer-consistent orderings \(\mathcal{C}(O)\), those with minimal
\emph{spectral complexity} are combinatorially dominant.

\paragraph{Definition (Spectral Complexity).}
The spectral complexity \(\Sigma(\pi)\) of an ordering \(\pi\) is defined as the number
of independent frequency--phase components required to encode the correlations preserved
by \(\pi\) within finite observer resolution.

Formally, if an ordering induces a representation
\[
\psi = \sum_k a_k e^{i(\omega_k t + \phi_k)},
\]
then
\[
\Sigma(\psi) = \left| \{ (\omega_k,\phi_k) \mid a_k \neq 0 \} \right|.
\]

We then have
\begin{equation}
\#\{\pi \in \mathcal{C}(O) \mid \Sigma(\pi) \text{ minimal}\}
\gg
\#\{\pi \in \mathcal{C}(O) \mid \Sigma(\pi) \text{ large}\}.
\end{equation}

This dominance is structural rather than probabilistic: low-bandwidth histories
admit vastly more microstate realizations consistent with observer continuity.
Our numerical simulations confirm that spectrally sparse orderings dominate across
a wide range of ensembles.

The wavefunction arises as the minimal linear encoding that preserves correlations
across such spectrally minimal orderings.
Superposition emerges naturally because linear combinations of spectrally sparse
components preserve observer-relevant correlations.
Interference reflects overlap between distinct orderings within this linear encoding.

The effective Born weights arise from the combinatorial multiplicity of orderings
consistent with particular spectral envelopes.
Observers therefore experience quantum probabilities not as fundamental randomness,
but as a consequence of the structural abundance of spectrally minimal histories.

\subsection{Summary}

\begin{itemize}
\item The universe is a static informational substrate; time is an emergent property of observers.
\item Observers experience temporal order through permutations compatible with internal structure.
\item Spectrally minimal orderings dominate combinatorially among observer-consistent histories.
\item Wave phenomena and quantum probabilities arise from linear encodings of dominant spectral structure.
\end{itemize}

\section{Wheeler--DeWitt}

We adopt the canonical formulation of quantum gravity, in which the universal state
\(
\Psi[h_{ij}(\mathbf{x}),\phi(\mathbf{x})]
\)
satisfies the Wheeler--DeWitt equation
\begin{equation}
\hat{H}\Psi = 0,
\end{equation}
and contains no fundamental time parameter \cite{DeWitt1967}.
We interpret \( \Psi \) in an Everettian sense as a static superposition over all admissible
3-geometries and matter field configurations:
\begin{equation}
\ket{\Psi} = \sum_i c_i \ket{h_i,\phi_i}.
\end{equation}

In this view, the universe is a timeless informational object.
All apparent dynamics arise from correlations internal to \( \Psi \),
rather than from external temporal evolution.

\section{Spectral Selection Principle}

Among all relational histories consistent with the universal state \( \Psi \),
observer experience is dominated by histories of minimal spectral complexity.

Let \( S \) denote a relational history induced by a particular factorization
and ordering of configurations.
Let \( \Sigma(S) \) denote the number of independent frequency--phase components
required to encode its correlations within finite observer resolution.
The relative weight of \( S \) is
\begin{equation}
P(S) \propto e^{-\alpha \Sigma(S)},
\end{equation}
where \( \alpha \) is a scale parameter determined by observer bandwidth.

\paragraph{Spectral Selection Principle (SSP).}
Among all observer-consistent relational histories in a timeless universe,
those with minimal spectral complexity dominate observer experience.

\section{Emergence of Quantum Structure}

\subsection{Wavefunction as Optimal Encoding}

The wavefunction \( \psi \) emerges as the optimal linear encoding of observer-relevant correlations.
Highly irregular configurations require broad spectral support and are exponentially suppressed,
leading to an effective low-frequency, smooth structure.

\subsection{Unitarity from Informational Stability}

Persistence of observer identity requires preservation of total encoded information.
In a linear representation space, admissible transformations must preserve norm.
Continuous norm preservation uniquely selects unitary evolution,
with a Hermitian generator \( \hat{H} \):
\begin{equation}
i\hbar \frac{\partial}{\partial t}\psi = \hat{H}\psi.
\end{equation}

\section{The Born Rule from Spectral Optimality}

We derive the Born rule as the unique decoding rule compatible with
spectral selection and observer persistence.

\subsection{Problem Statement}

Let \( \psi \in \mathcal{H} \) be the observer's compressed spectral encoding.
A probability rule assigns to each outcome \( i \) a weight
\begin{equation}
P(i) = f(|\psi_i|).
\end{equation}

\subsection{Constraints}

\paragraph{(C1) Additivity}
\[
P(i \cup j) = P(i) + P(j).
\]

\paragraph{(C2) Composition}
\[
P(i_A,j_B) = P(i_A)P(j_B).
\]

\paragraph{(C3) Basis invariance}
Probabilities must be invariant under unitary transformations.

\paragraph{(C4) Reconstruction optimality}
The decoding rule must minimize mean squared reconstruction error.

\subsection{Uniqueness}

Constraints (C1) and (C2) imply quadratic scaling:
\[
f(x) = kx^2.
\]
Constraint (C3) excludes phase dependence.
Constraint (C4) uniquely selects the \(L^2\) norm.
Normalization fixes \(k=1\), yielding
\[
P(i) = |\psi_i|^2.
\]

\subsection{Physical Meaning of the Born Weight}

\paragraph{Lemma (Quadratic Scaling).}
In a linear wave-based encoding, the number of distinguishable microstates
consistent with a spectral envelope scales with the squared norm.
By Parseval's theorem,
\[
\int |\psi(x)|^2 dx
\]
measures the total spectral power, which corresponds to the available
microstate volume consistent with observer structure.

Thus, the multiplicity of observer-instances satisfies
\[
M \propto |\psi|^2.
\]

\section{Discussion}

Time, spacetime geometry, quantum probabilities, and classicality emerge as features
of typical correlations within a timeless universal state.
Singularities and divergences mark boundaries of informational accessibility,
not physical breakdowns.

The Spectral Selection Principle shifts the explanatory burden from fundamental dynamics
to informational structure, unifying the emergence of time, quantum theory,
and the absence of observable infinities.

\printbibliography

\end{document}
