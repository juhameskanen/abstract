\documentclass[11pt]{article}
% -- Common definitions for all papers

% --- Packages ---
\usepackage[utf8]{inputenc}
\usepackage[T1]{fontenc}
\usepackage{amsmath, amssymb, amsthm}
\usepackage{geometry}
\usepackage{hyperref}
\usepackage{cite}
\newcommand{\pdflink}[1]{} % By default, do nothing

% --- Page Setup ---
\geometry{letterpaper, margin=1in}
\setlength{\parindent}{0pt}
\setlength{\parskip}{1em}


% --- Author ---
\author{Abstract Researcher}

% Glossary
\newcommand{\SSP}{Spectral Selection Principle (SSP)}
\newcommand{\Lcost}{$\mathcal{L}$} % Added $ $ here
\newcommand{\Tobs}{$\mathcal{T}_{\mathrm{obs}}$} % Added $ $ here

\newcommand{\PrintGlossary}{
    \section*{Glossary of the Abstract Universe}
    \begin{description}
    \item[SSP] \SSP: The foundational principle asserting that reality is the subset of informational paths that admit the most efficient representation in the frequency domain. It replaces "laws of physics" with a statistical preference for compressibility.
    
    \item[\Tobs] Observer Filter

    \item[$\mathcal{L}$ (Spectral Encoding Length)] The measure of information required to represent a state or path $\gamma$. In this framework, $\mathcal{L}$ replaces the classical concept of \textit{Action}. Minimizing $\mathcal{L}$ is equivalent to the Principle of Least Action.
    
    \item[$\mathcal{T}_{\mathrm{obs}}$ (Observer Filter)] The subset of all possible configuration sequences capable of describing a subjective experience.
    
    \item[MSM (Microstructure Motif)] Recurring, compressible bit-patterns within configuration space (e.g., particles, fields). The density of these motifs determines the local spectral cost, manifesting as mass-energy.
    
    \item[Nyquist Horizon] The "Planck Scale" of the theory. It is the maximum frequency supportable by the discretization of configuration space. Any structure below this limit is mathematically "aliased" and cannot be rendered.
    
    \item[Induced Time] The sequential ordering of states along a path $\gamma$. An emergent property of the observer's trajectory.
    
    \item[Phase-Coherence] The informational alignment between disparate parts of the spectral encoding. This provides the mathematical basis for what is traditionally called \textit{quantum entanglement}.

    \end{description}
}


\addbibresource{../../references.bib}

\title{Supplementary Material: Reading a Timeless Universe}
\date{2025}

\begin{document}

\maketitle

\section*{Abstract}
This thought experiment explores the mechanics of time as an observer-relative ordinal within a static informational ensemble.
We derive the perceived arrow of time and the cosmological dynamics of expansion/collapse from the observer's self-location on a universal algorithmic probability landscape.

\section{The Static Ensemble and the Observer's Walk}
Consider a closed, finite informational universe represented by the static Superspace $\mathcal{C}$ (the ensemble of all possible bit-configurations).
Within this ensemble, no fundamental dynamics or temporal evolution exists. We posit that time $t$ is not a property of $\mathcal{C}$, but the \textbf{ordinal index} of an observer's sequential traversal of states:
\begin{equation}
    t = \{0, 1, 2, 3, \dots\}
\end{equation}
This "walk" represents the internal ordering imposed by the observer.

\section{Perspectival Arrows and the Selection Principle}
According to the \textbf{Spectral Selection Principle (SSP)}, an observer is exponentially more likely to self-locate along paths of \textbf{Minimal Description Length (MDL)}. The probability of a specific history $H$ is given by the universal distribution:
\begin{equation}
    P(H) \propto 2^{-K(H)}
\end{equation}
where $K(H)$ is the Kolmogorov complexity of the path. The perceived direction of time is thus a filter: observers find themselves on trajectories that maximize algorithmic redundancy and structural persistence.

\section{The Symmetry of Expansion and Collapse}
The perceived dynamics of the universe (Expansion vs. Collapse) are determined by the observer's relative position on the algorithmic probability "hill":

\begin{enumerate}
    \item \textbf{The Left Slope (Low-Entropy Start):} 
    The observer originates in a highly compressible, low-entropy state ($H \approx 0$). As the walk progresses, the configuration space "unfolds," leading to an emergent motif density. The observer perceives an \textbf{expanding universe} with a singularity in the relative past.
    
    \item \textbf{The Right Slope (High-Entropy Start):} 
    The observer originates in a high-entropy region and traverses toward denser, more redundant configurations. The observer perceives a \textbf{collapsing universe} where motifs converge toward a singularity in the relative future.
\end{enumerate}

\begin{figure}[h!]
\centering
\includegraphics[width=0.7\textwidth]{figures/two-arrows-of-time.png}
\caption{The arrow of time as observer property}
\end{figure}


\section{Conclusion}
Expansion and collapse are not inherent properties of the informational ensemble, but complementary "readings" of the same static object.
The arrow of time is entirely perspectival; it is the geometric manifestation of an observer following the gradient of maximal compressibility.
In this framework, the "Big Bang" and "Black Holes" are symmetric boundary conditions of the informational trace.

\end{document}
