\documentclass[12pt]{article}
\usepackage{amsmath, amssymb, hyperref}
\usepackage{amsthm}
\usepackage[backend=biber]{biblatex}
\usepackage{graphicx}
\addbibresource{../references.bib} 
\newtheorem{lemma}{Lemma}
\usepackage{geometry}
\geometry{margin=1in}
\newcommand{\pdflink}[1]{} % By default, do nothing

\title{Emergent Information-Theoretic Suppression of QFT Divergences}
\author{Juha Meskanen \\ \texttt{meskanen.com}}
\date{2021}

\begin{document}
\maketitle

\begin{abstract}
  Building on the informational perspective of reality developed in Papers II–V, we investigate ultraviolet divergences in quantum field theory assuming the wavefunction emerges as a minimal, compressible program encoding the underlying bitstring field configuration. 
  High-frequency modes correspond to incompressible states with vanishing amplitude, leading naturally to suppression of divergent contributions.
  Numerical simulations of a 1D lattice $\phi^4$ theory demonstrate that amplitude-weighted one-loop integrals remain finite and flat across momentum space, while standard integrals diverge.
  These results suggest that interpreting quantum fields as emergent, algorithmic structures provides an intrinsic, information-theoretic UV regularization mechanism, obviating the need for conventional cutoffs or counterterms. 
\end{abstract}


\ifdefined\ishtml
\begin{center}
\small \href{qft-divergences.pdf}{Download PDF Version}
\end{center}
\fi


\section{Introduction}
Traditional QFT encounters ultraviolet (UV) divergences in loop integrals. Regularization and renormalization schemes are typically introduced to control these infinities. Here we propose an alternative: treating the wavefunction as an emergent, algorithmic-information-theoretic object. Highly complex (incompressible) field configurations contribute negligibly to physical amplitudes. This naturally suppresses UV contributions, providing a built-in regularization.

\section{Methods}
\subsection{Lattice \(\phi^4\) Field Simulation}
We consider a one-dimensional lattice with \(N\)  sites, generating field configurations \(\phi(x)\)  subject to a \(\phi^4\)  interaction:
\begin{equation}
\mathcal{L} = \frac{1}{2} (\partial \phi)^2 + \frac{1}{2} m^2 \phi^2 + \frac{\lambda}{4!} \phi^4.
\end{equation}

Each configuration is mapped to a bitstring encoding the discretized field:
\begin{equation}
\phi_i \in [\phi_{\min}, \phi_{\max}] \rightarrow b_i \in \{0,1\}^{\text{bits\_per\_site}}.
\end{equation}

\subsection{QBitwave Wavefunction Reconstruction}
Given a bitstring \(\mathbf{b}\) , the QBitwave class constructs a normalized complex amplitude array \(\psi(\mathbf{b})\)  representing the minimal program reproducing \(\mathbf{b}\) . Compression-based weights are derived from amplitude norms:
\begin{equation}
w(\mathbf{b}) = \|\psi(\mathbf{b})\|^2.
\end{equation}

\subsection{One-Loop Integral Evaluation}
The one-loop integral for a lattice momentum cutoff \(k_\text{cut}\)  is computed as:
\begin{align}
I_\text{std}(k_\text{cut}) &= \frac{1}{N_\text{realizations}} \sum_{r=1}^{N_\text{realizations}} \frac{1}{2\sqrt{4 \sin^2(k_\text{cut}/2) + m^2}}, \\
I_\text{info}(k_\text{cut}) &= \frac{1}{N_\text{realizations}} \sum_{r=1}^{N_\text{realizations}} w(\mathbf{b}_r) \frac{1}{2\sqrt{4 \sin^2(k_\text{cut}/2) + m^2}}.
\end{align}

\section{Results}
\subsection{Simulation Parameters}
To validate the QBitwave suppression, we performed a Metropolis-Hastings MCMC simulation. Parameters are summarized in Table~\ref{tab:params}.

\begin{table}[h]
\centering
\begin{tabular}{|l|c|r|}
\hline
Parameter & Symbol & Value \\ \hline
Lattice Sites & \(N\)  & 128 \\
Encoding Resolution & \(n\)  & 8 bits/site \\
Mass & \(m\)  & 1.0 \\
Coupling Range & \(\lambda\)  & \(\{0.1, 1.0, 10.0\}\)  \\
MCMC Steps & \(N_{steps}\)  & 1000 \\ \hline
\end{tabular}
\caption{Parameters for the information-theoretic \(\phi^4\)  simulation.}
\label{tab:params}
\end{table}

\subsection{Integral Comparison}
Figure~\ref{fig:integrals} shows the cumulative one-loop integrals versus momentum cutoff \(k\) :

\begin{figure}[h!]
\centering
\includegraphics[width=0.75\textwidth]{figures/qft_integrals.png}
\caption{Comparison of standard (divergent) one-loop integral \(I_\text{std}\)  and QBitwave amplitude-weighted integral \(I_\text{info}\) .}
\label{fig:integrals}
\end{figure}

\subsection{Entropy Signature}
The bitstring Shannon entropy, averaged over realizations, rises from near zero to close to unity as the lattice modes are sampled:

\begin{figure}[h!]
\centering
\includegraphics[width=0.75\textwidth]{figures/qft_entropy.png}
\caption{Average bitstring entropy \(H(k)\)  versus momentum cutoff \(k\) .}
\label{fig:entropy}
\end{figure}

\section{Discussion}
The results suggest that high-momentum modes correspond to incompressible bitstrings with near-zero amplitude in the QBitwave framework.
This emergent wavefunction perspective provides a natural, information-theoretic UV regularization without the need for manual counter-terms.

\section{Conclusion and Future Work}
We have demonstrated that interpreting the wavefunction as a compression-based emergent object successfully regulates one-loop QFT divergences.
Future research will focus on:
\begin{itemize}
    \item \textbf{High-Dimensional Scaling:} Extending the framework to \(3+1\)  dimensions.
    \item \textbf{Non-Abelian Gauge Fields:} Applying weighting to \(SU(N)\) theories.
\end{itemize}

\section*{Supplementary Materials}

\begin{itemize}
\item \href{simulations/phi4\_qbitwave\_mcmc\_2.py}{phi4\_qbitwave\_mcmc\_2.py: UV suppression in Quantum Field Theory}
\item \href{supplementary-qbitwave.pdf}{QBitwave supplementary paper}
\end{itemize}


\begin{verbatim}
from qbitwave import QBitwave
\end{verbatim}

\end{document}
