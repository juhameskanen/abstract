\documentclass[12pt]{article}
\usepackage{amsmath, amssymb, hyperref}
\usepackage{amsthm}
\usepackage[backend=biber]{biblatex}
\addbibresource{../references.bib}
\newtheorem{lemma}{Lemma}
\usepackage{geometry}
\geometry{margin=1in}

\title{Emergent Information-Theoretic Suppression of QFT Divergences: \\ A QBitwave Approach}
\author{Juha Meskanen \\ \texttt{meskanen.com}}
\date{July 2025}

\begin{document}
\maketitle

\begin{abstract}
We investigate the ultraviolet divergences of quantum field theory (QFT) from an information-theoretic perspective, employing the QBitwave framework in which the wavefunction emerges as a minimal, compressible program reproducing the underlying bitstring configuration of a field. By interpreting high-frequency modes as incompressible states with vanishing amplitude, we show that one-loop integrals are naturally regularized without imposing external cutoffs. Numerical simulations confirm that the QBitwave amplitude-weighted integrals remain finite and flat across momentum space, while standard integrals diverge.
\end{abstract}

\section{Introduction}

Traditional QFT encounters ultraviolet (UV) divergences in loop integrals. Regularization and renormalization schemes are typically introduced to control these infinities. Here we propose an alternative: treating the wavefunction as an emergent, algorithmic-information-theoretic object. Highly complex (incompressible) field configurations contribute negligibly to physical amplitudes. This naturally suppresses UV contributions, providing a built-in regularization.

\section{Methods}

\subsection{Lattice $\phi^4$ Field Simulation}

We consider a one-dimensional lattice with $N$ sites, generating field configurations $\phi(x)$ subject to a $\phi^4$ interaction:
\begin{equation}
\mathcal{L} = \frac{1}{2} (\partial \phi)^2 + \frac{1}{2} m^2 \phi^2 + \frac{\lambda}{4!} \phi^4.
\end{equation}

Each configuration is mapped to a bitstring encoding the discretized field:
\begin{equation}
\phi_i \in [\phi_{\min}, \phi_{\max}] \rightarrow b_i \in \{0,1\}^{\text{bits\_per\_site}}.
\end{equation}

\subsection{QBitwave Wavefunction Reconstruction}

Given a bitstring $\mathbf{b}$, the QBitwave class constructs a normalized complex amplitude array $\psi(\mathbf{b})$ representing the minimal program reproducing $\mathbf{b}$. Compression-based weights are derived from amplitude norms:
\begin{equation}
w(\mathbf{b}) = \|\psi(\mathbf{b})\|^2.
\end{equation}

\subsection{One-Loop Integral Evaluation}

The one-loop integral for a lattice momentum cutoff $k_\text{cut}$ is computed as:
\begin{align}
I_\text{std}(k_\text{cut}) &= \frac{1}{N_\text{realizations}} \sum_{r=1}^{N_\text{realizations}} \frac{1}{2\sqrt{4 \sin^2(k_\text{cut}/2) + m^2}}, \\
I_\text{info}(k_\text{cut}) &= \frac{1}{N_\text{realizations}} \sum_{r=1}^{N_\text{realizations}} w(\mathbf{b}_r) \frac{1}{2\sqrt{4 \sin^2(k_\text{cut}/2) + m^2}}.
\end{align}

Here $I_\text{std}$ is the conventional integral and $I_\text{info}$ is the QBitwave-weighted integral.

\section{Results}

\subsection{Simulation Parameters}

We used:
\begin{itemize}
\item Lattice sites $N = 128$
\item Bits per site = 8
\item One-loop realizations = 300
\item Mass $m = 1.0$, interaction $\lambda = 1.0$
\end{itemize}

\subsection{Integral Comparison}

Figure~\ref{fig:integrals} shows the cumulative one-loop integrals versus momentum cutoff $k$:

\begin{figure}[h!]
\centering
\includegraphics[width=0.75\textwidth]{figures/qft_integrals.png}
\caption{Comparison of standard (divergent) one-loop integral $I_\text{std}$ and QBitwave amplitude-weighted integral $I_\text{info}$. The QBitwave integral remains flat, demonstrating natural suppression of UV modes.}
\label{fig:integrals}
\end{figure}

\subsection{Entropy Signature}

The bitstring Shannon entropy, averaged over realizations, rises from near zero to close to unity as the lattice modes are sampled:

\begin{figure}[h!]
\centering
\includegraphics[width=0.75\textwidth]{figures/qft_entropy.png}
\caption{Average bitstring entropy $H(k)$ versus momentum cutoff $k$. The gradual rise indicates that the field configurations gradually fill the accessible informational space.}
\label{fig:entropy}
\end{figure}

\section{Results}

\subsection{Simulation Parameters}
To validate the QBitwave suppression, we performed a Metropolis-Hastings MCMC simulation of the lattice field. The simulation parameters are summarized in Table~\ref{tab:params}.

\begin{table}[h]
\centering
\begin{tabular}{|l|c|r|}
\hline
Parameter & Symbol & Value \\ \hline
Lattice Sites & $N$ & 128 \\
Encoding Resolution & $n$ & 8 bits/site \\
Mass & $m$ & 1.0 \\
Coupling Range & $\lambda$ & $\{0.1, 1.0, 10.0\}$ \\
MCMC Steps & $N_{steps}$ & 1000 \\ \hline
\end{tabular}
\caption{Parameters for the information-theoretic $\phi^4$ simulation.}
\label{tab:params}
\end{table}

\subsection{Integral Comparison and IR Convergence}
The cumulative one-loop integral $I_{info}(k)$ was evaluated across the momentum spectrum. As shown in Figure~\ref{fig:integrals}, $I_{info}$ tracks $I_{std}$ at low $k$ (IR limit), indicating that the QBitwave weighting preserves the fundamental mass $m$. However, as $k$ increases, $I_{info}$ plateaus, demonstrating an emergent, finite UV regularization.

\section{Discussion}

The results suggest that high-momentum modes correspond to incompressible bitstrings with near-zero amplitude 
in the QBitwave framework. This emergent wavefunction perspective provides a natural, information-theoretic UV 
regularization without the need for manual counter-terms or external cutoffs. 
Furthermore, the bitstring entropy curve (Figure~\ref{fig:entropy}) confirms that the field configurations gradually 
fill the accessible informational space, with $\lambda$ acting as a constraint on the manifold of compressible states.

\section{Conclusion and Future Work}

We have demonstrated that interpreting the wavefunction as a compression-based emergent object successfully regulates 
one-loop QFT divergences. Future research will focus on:

\begin{itemize}
    \item \textbf{High-Dimensional Scaling:} Extending the framework to $3+1$ dimensions to verify the stability of the 
    emergent cutoff.
    \item \textbf{Non-Abelian Gauge Fields:} Applying weighting to $SU(N)$ theories to investigate mass gaps and 
    confinement.
    \item \textbf{Gravitational Coupling:} Investigating how the bitstring field complexity defines the local 
    geometry of a dynamical lattice, providing a pathway to a unified theory of quantum gravity.
\end{itemize}

\section*{Simulation Code}

Python simulation is implemented in \texttt{phi4\_qbitwave\_final.py}, utilizing the QBitwave class:

\begin{verbatim}
from qbitwave import QBitwave
# Generate phi^4 lattice fields, encode as bitstrings,
# reconstruct wavefunction, and compute weighted integrals.
\end{verbatim}

\end{document}
