\documentclass[12pt]{article}
\usepackage{amsmath, amssymb, hyperref}
\usepackage{amsthm}
\usepackage[backend=biber]{biblatex}
\usepackage{graphicx}
\addbibresource{../references.bib} 
\newtheorem{lemma}{Lemma}
\usepackage{geometry}
\geometry{margin=1in}
\newcommand{\pdflink}[1]{} % By default, do nothing
\usepackage{listings}

\lstset{
    basicstyle=\ttfamily\small,
    breaklines=true
}

\title{Supplementary Material: \\ The \texttt{QBitwave} Framework}
\author{Juha Meskanen}
\date{2019 ... 2024}

\begin{document}
\maketitle

\section{Purpose and Scope}
This document provides a precise reference specification for the Python class \texttt{QBitwave}, used in the associated manuscripts. 
It formalizes the mapping from discrete bitstrings to emergent complex amplitudes and describes derived quantities such as Shannon entropy, compressibility, and coherence.

\section{Preliminaries}
Let:
\begin{itemize}
    \item $b = (b_1,\dots,b_N) \in \{0,1\}^N$ denote a bitstring of length $N$.
    \item $B = \{0,1\}^N$ denote the set of all bitstrings of length $N$.
    \item $n_\text{block}$ denote the block size for bit-to-amplitude conversion (even integer).
    \item $\mathbb{C}$ denote the complex numbers.
    \item $\|\cdot\|_2$ denote the Euclidean norm.
\end{itemize}

\section{Forward Mapping: Wavefunction to Bitstring}
Given normalized complex amplitudes $\psi = (\psi_1, \dots, \psi_M) \in \mathbb{C}^M$ and associated phases $\phi = (\phi_1,\dots,\phi_M)$:
\begin{align}
    \psi_j &= r_j e^{i \phi_j}, \quad r_j = |\psi_j|, \quad j = 1,\dots,M, \\
    b_\text{amp}^j &= \text{binary\_encode}(r_j, k_\text{amp}), \\
    b_\text{phase}^j &= \text{binary\_encode}\left(\frac{\phi_j}{2\pi}, k_\text{phase}\right),
\end{align}
where $k_\text{amp}, k_\text{phase}$ are user-selected bits per amplitude/phase.  
The flattened bitstring is
\[
b = \bigoplus_{j=1}^M \bigl( b_\text{amp}^j \, || \, b_\text{phase}^j \bigr),
\]
with $||$ denoting concatenation.

\section{Reverse Mapping: Bitstring to Wavefunction}
Partition the input bitstring $b$ into consecutive blocks of size $n_\text{block}$:
\begin{align}
    b &= (b_1, \dots, b_N), \quad n_\text{block} \mid N, \\
    n_\text{blocks} &= N / n_\text{block}.
\end{align}

Each block is split into two halves (real and imaginary parts):
\begin{align}
    \text{Re}(\psi_j) &= \text{bits\_to\_signed\_float}\left(b_{(j-1)n_\text{block}+1:j n_\text{block}/2}\right), \\
    \text{Im}(\psi_j) &= \text{bits\_to\_signed\_float}\left(b_{j n_\text{block}/2+1:j n_\text{block}}\right),
\end{align}
and then normalized:
\[
\psi = \frac{\psi}{\|\psi\|_2}.
\]

\begin{figure}[h!]
\centering
\includegraphics[width=0.75\textwidth]{figures/qbitwave.jpg}
\caption{The structure of QBitwave}
\label{fig:qbitwave}
\end{figure}



\section{Derived Quantities}

\subsection{Amplitude Probabilities}
\[
p_j = |\psi_j|^2, \quad \sum_{j=1}^M p_j = 1.
\]

\subsection{Shannon Entropy of Wavefunction}
\[
S_\psi = - \sum_{j=1}^M p_j \log_2 p_j.
\]

\subsection{Bitstring Entropy (Syntactic Entropy)}
Let $p_0, p_1$ be the fractions of zeros and ones in $b$:
\[
H_b = -p_0 \log_2 p_0 - p_1 \log_2 p_1.
\]

\subsection{Coherence}
The Kullback–Leibler divergence between bit-level and amplitude-level distributions:
\[
D_\text{KL}(P_b \parallel P_\psi) = \sum_{i=0,1} P_b(i) \log_2 \frac{P_b(i)}{P_\psi(i)},
\]
where $P_\psi$ is truncated/aligned to the same length as $P_b$.

\subsection{Compressibility}
Define the Fourier transform of amplitudes:
\[
\hat{\psi} = \text{FFT}(\psi),
\]
and let $N_\text{sig}$ be the number of Fourier coefficients with magnitude above threshold $\tau$. Then
\[
C = 1 - \frac{N_\text{sig}}{M}, \quad C \in [0,1].
\]

\section{Randomization / Mutation Operations}
Small perturbations of the amplitudes $\psi \to \psi + \eta$, with $\eta \sim \mathcal{CN}(0, \sigma^2)$ (complex Gaussian), are normalized to unit norm.  
Random bit flips on $b$ trigger automatic recomputation of $\psi$.

\section{API Summary (Illustrative)}
\begin{lstlisting}[language=Python]
from qbitwave import QBitwave

bw = QBitwave(bitstring)
amps = bw.get_amplitudes()
entropy = bw.entropy()
bit_entropy = bw.bit_entropy()
coherence = bw.coherence()
compress = bw.compressibility()
bw.mutate(level=0.01)
bw.flip(n_flips=5)
\end{lstlisting}

\section{Relation to Theory in Main Text}
The following properties are essential to the analytical results:
\begin{itemize}
    \item Normalization of amplitudes
    \item Mapping from bitstring to complex amplitudes
    \item Computation of Shannon entropy and compressibility
\end{itemize}
All other methods (e.g., random flips, mutation) are ancillary and used only in simulation studies.

\end{document}
