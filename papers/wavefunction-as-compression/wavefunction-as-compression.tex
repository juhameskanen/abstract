\documentclass[12pt]{article}
% -- Common definitions for all papers

% --- Packages ---
\usepackage[utf8]{inputenc}
\usepackage[T1]{fontenc}
\usepackage{amsmath, amssymb, amsthm}
\usepackage{geometry}
\usepackage{hyperref}
\usepackage{cite}
\newcommand{\pdflink}[1]{} % By default, do nothing

% --- Page Setup ---
\geometry{letterpaper, margin=1in}
\setlength{\parindent}{0pt}
\setlength{\parskip}{1em}


% --- Author ---
\author{Abstract Researcher}

\addbibresource{../references.bib}

\title{The Wave Function as a Compression Algorithm}

\date{2019}

\begin{document}

\maketitle

\begin{abstract}
  Building on the informational interpretation of reality and the emergence of structure from entropy (Papers II–IV), we propose that the quantum wave function is not a primary physical entity but a universal compression algorithm.
  In this framework, the deterministic evolution of $\psi$ encodes the most probable, compressed representation of all underlying information.
  We present a computational simulation in a linear quantum system, demonstrating that coefficient dynamics derived from compression-optimal bases reproduce near-unitary evolution and approximately Hermitian generators.
  These results suggest that the formal structure of quantum mechanics arises naturally from information-theoretic optimization, supporting a view in which physics is the study of efficiently encoded informational outputs rather than fundamental ontological processes.
\end{abstract}

\ifdefined\ishtml
\begin{center}
\small \href{wavefunction-as-compression.pdf}{Download PDF Version}
\end{center}
\fi


\section{Theoretical Framework}
Our theory rests on three fundamental assumptions:

\subsection*{Assumption 1: The Universe is fundamentally abstract and informational.}
The cosmos is fundamentally informational, characterized by ``white noise,'' representing the myriad of virtual particles and ephemeral fluctuations at the quantum foam level—short-lived, small, random informational structures. However, interspersed within this randomness are sections that encode profound symmetries.

\subsection*{Assumption 2: Maximally Compressed States}
In any set of information, the number of highly compressed sequences (those with low Kolmogorov complexity) will statistically dominate over purely random,
incompressible sequences in terms of \textit{persistence} and \textit{structure}.
Maximum compression implies maximum probability. Predictable information compresses well, which explains why we find ourselves in a universe that is the most probable one,
governed by predictable laws of physics.

\subsection*{Assumption 3: The Wave Function as the Compression Algorithm}
The \textbf{wave function $\psi$ is the universal compression algorithm} of the cosmos. The deterministic evolution described by quantum mechanics,
which governs the behavior of microstructures, is the direct consequence of this algorithm's operation.

\section{Computational Simulation}
To explore the ``wave function-as-compression-coefficients'' idea, we designed a simulation to test its predictions in a linear regime.

\subsection{Simulation Setup}
We consider a 1D lattice with $N=64$ sites. We simulate the time evolution of a single-particle tight-binding model, which is a standard representation of a particle hopping between nearest-neighbor sites. This model is governed by a nearest-neighbor hopping Hamiltonian, resulting in complex wave functions $\psi(t)$ over $T=200$ timesteps.

The simulated time evolution adheres to the Schrödinger equation:
\[
i\hbar \frac{\partial}{\partial t} \psi(x,t) = H \psi(x,t)
\]
where $H$ is the tight-binding Hamiltonian. For simplicity, we set $\hbar=1$.

From the complex wave function $\psi_t$ at each timestep $t$, we generate ``observed frames'' in two variants:
\begin{enumerate}
    \item \textbf{Real part:} $x_t = \text{Re}(\psi_t)$
    \item \textbf{Intensity (probability density):} $x_t = |\psi_t|^2$
\end{enumerate}

\subsection{Data Processing and Analysis}
\begin{enumerate}
    \item \textbf{Principal Component Analysis (PCA):} We applied PCA to the dataset of frames $\{x_t\}_{t=1}^{T}$. PCA finds an orthogonal basis that best captures the variance in the data. We projected the data onto the top 8 principal components, yielding coefficient vectors $\{c_t\}$ where $c_t \in \mathbb{R}^8$.
    \item \textbf{Linear Propagator Fitting:} Assuming the compression process generates deterministic evolution in the coefficient space, we fit a linear propagator matrix $G$ between successive vectors using least squares:
    \[
    c_{t+1} \approx G c_t
    \]
    \item \textbf{Unitarity and Hermiticity Test:} We test if the propagator $G$ is approximately unitary ($G^\dagger G \approx I$) and if the generator $H_{gen} = -i \log G$ is approximately Hermitian ($H_{gen}^\dagger \approx H_{gen}$).
\end{enumerate}

\subsection{Results and Plots}
The simulation produced the following key visualizations:

\begin{figure}[h]
    \centering
    \includegraphics[width=0.45\textwidth]{figures/wavefunction-as-compresssion-1.png}
    \includegraphics[width=0.45\textwidth]{figures/wavefunction-as-compresssion-2.png} \\
    \includegraphics[width=0.45\textwidth]{figures/wavefunction-as-compresssion-3.png}
    \includegraphics[width=0.45\textwidth]{figures/wavefunction-as-compresssion-4.png}
    \caption{PCA components (real part) 1 through 4.}
\end{figure}

\begin{figure}[h]
    \centering
    \includegraphics[width=0.45\textwidth]{figures/wavefunction-as-compresssion-5.png}
    \includegraphics[width=0.45\textwidth]{figures/wavefunction-as-compresssion-6.png}
    \caption{Left: Singular values of fitted propagator $G$. Right: Reconstruction MSE vs number of PCA components.}
\end{figure}

\subsection{Positive Support}
The simulation yielded significant positive support for the hypothesis. In a clean, linear-unitary toy model, a compression-optimal linear basis produces coefficient dynamics that are demonstrably \textbf{near-unitary} and possess a \textbf{near-Hermitian generator}. This behavior precisely matches the predictions of our informational theory.


\section{Discussion and Caveats}
While the simulation provides compelling initial support, we acknowledge several limitations:
\begin{itemize}
    \item \textbf{Linearity Assumption:} PCA is a linear transform. If the relationship between bitstrings and amplitudes is nonlinear.
    \item \textbf{Decoherence:} The model neglects the effects of open quantum systems.
\end{itemize}

\section{Conclusion}
We propose that reality can be viewed as a geometric interpretation of information, with the wave function acting as a universal compression algorithm.
Finiteness observed in physics is not merely a boundary condition, but statistical phenomenon. The universe does not deal in infinities because infinities
cannot be efficiently stored or propagated by a maximally compressing system and are hence unprobable.

\section*{Simulation Software}

\begin{itemize}
\item \href{simulations/wavefunction-as-compression.py}{wavefunction-as-compression.py: Wavefunction as compression}
\end{itemize}

\printbibliography

\end{document}
