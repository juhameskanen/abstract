\chapter{Doctrines of Mathematics}

Logic and mathematics are abstract by nature. What is common to the addition of two bananas and the addition of two apples is the expression $2+2$. Yet $2+2$ is not something one can eat, touch, or weigh. It is not located anywhere in space. It has no mass, no electric charge.
It cannot be detected with any measurement device, not even LIGO or JWST!  What, then, is it?

This question has occupied philosophers and mathematicians for centuries. Several major doctrines attempt to explain the nature of mathematical objects.

\section{Platonism}

According to mathematical Platonism, numbers are real but non-physical objects. They do not exist in space or time, yet they exist independently of human minds. On this view, $2+2=4$ would remain true even if no humans — or no physical universe at all — existed. Mathematics is discovered rather than invented.

This doctrine explains the apparent universality and necessity of mathematics. However, it raises a difficult question: how can physical beings gain knowledge of non-physical entities?

\section{Formalism}

Formalism, associated with David Hilbert, treats mathematics as the manipulation of symbols according to explicit rules. Under this view, $2+2$ is not an abstract object but a string of symbols transformed by axioms and inference rules.

Mathematics becomes a formal game, similar to chess. The symbols themselves carry no intrinsic meaning; meaning arises only from their role within the system. This avoids metaphysical commitments, but it struggles to explain why mathematics applies so effectively to the physical world.

\section{Logicism}

Logicism, developed by Gottlob Frege and Bertrand Russell, attempts to reduce mathematics to logic. In this framework, numbers are logical constructions. The number $2$ can be defined as the class of all sets containing exactly two elements.

Thus, two bananas and two apples share the same logical structure. The number is not a physical object but a property of collections. Mathematics becomes an extension of pure logic.

\section{Structuralism}

Structuralism offers a more relational perspective. Numbers are not independent objects but positions within a structure. The number $2$ is defined by its relations — it follows $1$ and precedes $3$ within the natural number sequence.

On this view, what bananas and apples share is not a substance but a structural pattern. Mathematics describes invariant relational forms rather than standalone entities.

\section{Cognitive Accounts}

Modern cognitive science suggests that numbers arise from abstraction within the brain. Humans and even some animals possess an ability to track quantity. Through repeated interaction with discrete objects, the mind forms stable abstractions such as “two-ness.”

Here, $2$ is neither a Platonic entity nor merely a formal symbol. It is a highly compressed mental representation of recurring patterns in experience.

\section{The Underlying Puzzle}

Despite their differences, all these doctrines attempt to answer the same question: what kind of thing is \(2+2\)?

When we remove bananas and apples, what remains is something we call abstract. Remarkably, the universe appears to honor such abstractness without a single exception.

Why does the physical universe so consistently instantiate these abstractions?

Whether mathematics is discovered, invented, constructed, or cognitively abstracted, it captures something stable and unavoidable about reality. The same formal expression governs bananas, apples, electrons—everything we can observe. This suggests that mathematical structure is not accidental.

It reflects some fundamental feature of the universe.

But can the human mind, let alone an average programmer, truly comprehend the deep nature of abstractness?
