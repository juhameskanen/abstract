\chapter{Free Will}

If DNA is the blueprint of life and its operations can be described as an axiomatic system, then everything within the human experience must itself be axiomatic by nature. According to the Church--Turing thesis, humans can therefore---at least in principle---be simulated on a computer.

Computers, in general, are deterministic systems. They do not exhibit truly random behavior; even a so-called ``random value generator'' in a computer is based on deterministic logic, producing values that only appear random to an external observer.

It is evident that such a deterministic system cannot possess \emph{free will} as it is typically understood. A Turing machine running a DNA simulation is bound to follow its logic without any capacity to choose otherwise. This implies that individuals within a simulated universe cannot possess free will either; they are constrained to behave according to the underlying rules of the machine executing the simulation.

However, nothing prevents us from implementing a truly random generator. According to quantum mechanics, there is genuine randomness inherent in nature, which could, in principle, be leveraged to create an indeterministic computer. Could such a ``quantum-boosted'' machine grant simulated individuals free will?

Consider a person contemplating whether to turn left. Suddenly, a completely unexpected and indeterministic event occurs: a high-energy particle burst emitted by the Sun. A few particles pass through critical brain cells, perturbing the person’s neural activity in a genuinely unpredictable manner. As a result, instead of turning left, the person decides to move forward.

Does this random, unpredictable event introduce free will?

The person had no control over the distant particle burst, nor over the manner in which it affected their internal processing. Consequently, the person has no more free will with this random influence than without it. The only effect of the random event was that the resulting decision became detached from the relevant information and reasoning available to the individual.

In trivial situations, such randomness might appear harmless or even amusing, adding a sense of spontaneity. However, consider a scenario in which a person’s life is at stake. In such circumstances, the ability to make decisions based on relevant data is crucial. An indeterministic disturbance that disrupts logical reasoning does not enhance free will; it undermines it. We would not want to define free will as something that applies only to inconsequential choices. Survival depends on making decisions guided by meaningful information.

If our decisions are not influenced by indeterministic events, then they are the result of deterministic reasoning. If they \emph{are} influenced by such events, then they are driven by phenomena over which we have no control. Neither case supports the existence of free will.

Allowing randomness to influence decisions does not introduce freedom; it merely replaces reasoning with noise. A coin toss is not an act of will.

\begin{quote}
\textbf{Hypothesis of Free Will:}  
There is no free will.
\end{quote}

We do not possess the freedom to decide when to commit a sin or when to compensate through prayer. Every conscious decision we make follows logical processes in which $2 + 2$ invariably equals $4$. Like computer software, we are bound to follow our internal logic, without the capacity to choose otherwise.
