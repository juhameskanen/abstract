\title{A Zero-Parameter Theory of Emergent Physics}

\section{Intuition: From Singularities to Structured Universes}

Consider a “library” of all possible histories $\Gamma$ of finite bitstrings, ranging from complete randomness to highly coherent, law-like sequences.

If we draw a movie at random from $\Gamma_O$, what would we expect to observe?

With overwhelming probability, not chaos. 

The same underlying bits can encode the same observer in astronomically many ways. Among these, histories with regularity and smooth temporal structure admit vastly shorter descriptions.
Because the number of possible descriptions grows exponentially with description length, histories with minimal description dominate the observer-conditioned measure.

Consequently, observers overwhelmingly find themselves embedded in worlds that are predictable, regular, and law-like.


\section{ Fundamental Ontology: The Static Bit-Space}

The universe is a static informational totality. There is no dynamical generation, no ontological time, and no external metaphysics. No interaction, no signals that
could flip bits here because bits flipped elsewhere.

\begin{itemize}
\item \textbf{Configuration Space:} For $n$ bits, there exist $2^n$ configurations.
\item \textbf{Total Existence:} All configurations and all possible ordered traversals (histories) exist.
\item \textbf{The Observer:} Reality is a static informational space of bitstrings. An observer is a localized informational substructure.
\end{itemize}

\section{ The Measure Problem: Anthropic Sampling}

If all observers exist, the central question is the sampling measure: \textit{From which informational structure is an observer most likely sampled?}

\begin{itemize}
\item \textbf{The Claim:} The measure concentrates on the most compressible structures.
\item \textbf{The Metric:} Unlike Kolmogorov complexity ($K(x)$), which is uncomputable and brittle, we employ \textbf{Spectral Complexity} ($C_Q$) and \textbf{Geometric Complexity} ($C_G$).
\end{itemize}

\section{ Spectral Complexity and Wave Physics}

We define the complexity of the wavefunction $\psi$ as a Sobolev-type seminorm. In the discrete domain $\mathbb{Z}_N$:

\begin{equation}
C_Q = \sum_{i} k_{\text{eff}}^2 |A_i|^2
\end{equation}

where $k_{\text{eff}} = \min(|k|, N - |k|)$. This functional penalizes high-frequency components and sharp gradients.

\begin{itemize}
\item \textbf{Emergent Wave Behavior:} Minimizing $C_Q$ favors low-bandwidth, long-range correlated, and smooth representations.
\item \textbf{MPEG Analogy:} Just as cosine bases compress video, Fourier-like modes compress reality. Particles appear wave-like because wavefunctions are the optimal compression of microscopic informational correlations.
\end{itemize}

\section{ Geometric Complexity and Spacetime}

Geometry ($G$) is the optimal compression of macroscopic relational structure. It allows for stable, separable boundaries and localized informational identity.

\begin{equation}
C_G = \alpha \sum R^2 \sqrt{-g} + \beta \sum |a + \Gamma(v,v)|^2
\end{equation}

\begin{itemize}
\item \textbf{Curvature Penalty:} $R^2$ serves as a compression cost for relational irregularity.
\item \textbf{Geodesic Enforcement:} The term $\|a + \Gamma(v,v)\|^2$ ensures that non-geodesic motion increases description length.
\item \textbf{The Role of $G$:} Geometry provides local neighborhoods and causal cones, which are necessary for an observer to resist decoherence and preserve memory.
\end{itemize}

\section{ The Joint Dominance Principle}

The observer does not inhabit a unique $\psi$ or $G$. Instead, the measure over observers is weighted by the joint minimization of both functionals:

\begin{equation}
\mathbb{P}(\gamma \mid O) = \frac{1}{Z_O} \exp \left[ -\lambda \left( C_Q(\gamma) + C_G(\gamma) \right) \right] C(\gamma)
\end{equation}

\paragraph{Physical Implications:}
\begin{enumerate}
\item \textbf{No Native Unification:} $\psi$ and $G$ are not derived from one another; they are dual coordinate charts on the same informational manifold.
\item \textbf{Classical Emergence:} Physics emerges at the intersection of these two simplicity priors.
\item \textbf{Consistency:} Spectral smoothness (micro-scale) and geometric smoothness (macro-scale) are statistically aligned, producing a coherent physical world.
\end{enumerate}

\section{ Comparison of Complexity Metrics}

\begin{table}[h]
\centering
\begin{tabular}{|l|l|l|}
\hline
\textbf{Feature} & \textbf{Kolmogorov ($K$)} & \textbf{Spectral ($C$)} \\ \hline
\textbf{Computability} & Uncomputable & Computable (FFT) \\ \hline
\textbf{Stability} & Brittle/Discrete & Continuous/Smooth \\ \hline
\textbf{Physical Bias} & Arbitrary Language & Smoothness/Gradients \\ \hline
\textbf{Measure} & Program Length & Curvature in Function Space \\ \hline
\end{tabular}
\end{table}


\section{Functional Analysis of the Observer-Conditioned Measure}

\subsection{Physical Roles of the Functionals}

\paragraph{QBitwave ($C_Q$):} 
The spectral cost penalizes high-frequency modes, large gradients, and spectral spread. Consequently, low-frequency, smooth wavefunctions minimize cost, biasing the system toward:
\begin{itemize}
    \item Long-wavelength modes.
    \item Slowly varying amplitudes.
\end{itemize}

\paragraph{GBitwave ($C_G$):} 
The geometric cost penalizes curvature magnitude (acting as an $R^2$ proxy) and deviations from geodesics. This biases the system toward:
\begin{itemize}
    \item Flat or gently curved geometry.
    \item Worldlines that follow geodesics (classical spacetime structure).
\end{itemize}

The exponential measure $\mathbb{P}(\gamma \mid O)$ ensures that histories characterized by smooth $\psi$, smooth geometry, and geodesic motion are exponentially favored.

\subsection{Connection to the Euclidean Path Integral}

The probability structure $\mathbb{P} \sim \exp[-\lambda(C_Q + C_G)]$ is mathematically isomorphic to a \textbf{Euclidean Path Integral} weight:
\begin{equation}
    \mathbb{P} \sim e^{-S}
\end{equation}
where the Action is derived entirely from informational compression:
\begin{equation}
    S = C_Q + C_G
\end{equation}
Because the penalties are quadratic (e.g., $|\nabla \psi|^2$ and $R^2$ type terms), the minimizing histories satisfy Euler–Lagrange equations. This implies that field equations for $\psi$ and curvature equations for $G$ emerge as extrema of compression.

\subsection{The Coupling Problem}

In the current implementation, $C_Q$ and $C_G$ are independent additive penalties:
\begin{equation}
    C_{\text{total}} = C_Q + C_G
\end{equation}
This lacks the stress-energy coupling term found in General Relativity, such as:
\begin{equation}
    C_{\text{int}} = \int f(|\psi|^2) \Phi(g)
\end{equation}
Without a cross-term where high $\psi$-complexity influences geometric cost, $\psi$ does not curve $G$. Achieving emergent gravity requires that regions of high informational density increase geometric cost unless the geometry adapts.

\subsection{Formalized Implementation Measure}

To align the theory with the implemented simulation classes, the observer-conditioned measure is defined as:
\begin{equation}
    \mathbb{P}(\gamma \mid O) = \frac{1}{Z_O} \exp \big[ -\lambda C_{\text{total}}(\gamma) \big] C(\gamma)
\end{equation}
where $C_{\text{total}} = C_Q + C_G$. This formulation provides a compression-based action and a Boltzmann weighting over histories, explaining the dominance of smooth spacetime structures.




\section{Derivation of Emergent Physics from Complexity Functionals}

To find the "laws of physics" in a static informational universe, we treat the total complexity $C_{\text{total}}$ as a Euclidean Action $S$. The observed dynamics are the stationary points of this action, found by setting the variation $\delta S = 0$.

\subsection{1. The Emergent Wave Equation (Spectral Complexity)}
The spectral cost $C_Q$ is defined as the $L^2$ norm of the gradient in the continuous limit. This represents a penalty on high-frequency "roughness" in the informational representation:

\begin{equation}
    C_Q = \int |\nabla \psi|^2 \, dV
\end{equation}

Varying with respect to the conjugate wavefunction $\psi^*$:
\begin{align}
    \delta C_Q &= \int \nabla (\delta \psi^*) \cdot \nabla \psi \, dV \\
    &= -\int \delta \psi^* (\nabla^2 \psi) \, dV
\end{align}

Setting $\delta C_Q = 0$ for arbitrary $\delta \psi^*$ yields the Laplacian (the static wave equation):
\begin{equation}
    \nabla^2 \psi = 0
\end{equation}
In the presence of an observer's sampling frequency $k$, this emerges as the Helmholtz/Schrödinger form: $(\nabla^2 + k^2)\psi = 0$.

\subsection{2. The Emergent Field Equations (Geometric Complexity)}
The geometric cost $C_G$ penalizes curvature magnitude using a quadratic Ricci functional:

\begin{equation}
    C_G = \alpha \int R^2 \sqrt{-g} \, d^4x
\end{equation}

Varying the metric $g^{\mu\nu}$ leads to the field equations for Quadratic Gravity. The resulting Euler-Lagrange equation is:
\begin{equation}
    2 R R_{\mu\nu} - 2 \nabla_{\mu} \nabla_{\nu} R + 2 g_{\mu\nu} \nabla^2 R - \frac{1}{2} g_{\mu\nu} R^2 = 0
\end{equation}

\subsection{3. Emergent Motion (Geodesic Consistency)}
The cost of a path $\gamma$ is minimized when the trajectory follows the shortest informational distance. The motion cost $C_m$ penalizes the residual of the geodesic equation:

\begin{equation}
    C_m = \beta \int \left\| \frac{d^2 x^\mu}{ds^2} + \Gamma^\mu_{\alpha\beta} \frac{dx^\alpha}{ds} \frac{dx^\beta}{ds} \right\|^2 ds
\end{equation}

Minimizing this ensures that the "cheapest" history for an observer is one where:
\begin{equation}
    \ddot{x}^\mu + \Gamma^\mu_{\alpha\beta} \dot{x}^\alpha \dot{x}^\beta = 0
\end{equation}

\section{Theoretical Justification}

In General Relativity, the action is $S = \int R \sqrt{-g} \, d^4x$. However, an informational ontology necessitates $R^2$ for several critical reasons:

\begin{itemize}
    \item \textbf{Positive Definiteness:} In Information Theory, a cost (or Description Length) cannot be negative. Because the Ricci scalar $R$ can be negative, a linear functional allows for "negative information," creating an unstable vacuum. $R^2$ ensures that any deviation from flatness costs information, making $S \geq 0$.
    \item \textbf{Stability and Convergence:} A measure based on $e^{-\lambda \int R}$ would diverge as $R \to -\infty$. The $R^2$ functional ensures the probability density $P \sim e^{-\lambda C_G}$ is well-behaved and peaks at flat spacetime.
    \item \textbf{Renormalizability:} While standard GR ($R$) is non-renormalizable, quadratic gravity ($R^2$) is renormalizable. In a discrete bit-based reality, $R^2$ acts as a high-frequency regulator that prevents the formation of informational singularities (Black Holes with infinite density).
    \item \textbf{Scale-Dependent Limit:} At large scales (low curvature), the $R^2$ equations approximate the Einsteinian behavior. The $R^2$ term can be viewed as the leading-order correction to the "simplicity prior" of the universe's geometry.
\end{itemize}


\section{The Universal Informational Measure}

The probability $\mathbb{P}$ of any given informational structure (observer history) $\gamma$ is determined by the joint minimization of spectral and geometric complexity, weighted by a self-organized critical parameter $\lambda$:

\begin{equation}
    \mathbb{P}(\gamma \mid O) = \frac{1}{Z_O} \exp \Big[ -\lambda_c \big( C_Q(\gamma) + C_G(\gamma) + C_{\text{int}}(\gamma, G) \big) \Big]
\end{equation}

\subsection{1. Emergence of the Physical Constant}
The parameter $\lambda$ is not an arbitrary constant but is fixed by the requirement of \textbf{Computational Phase Coherence}. At the critical value $\lambda_c$:
\begin{itemize}
    \item The system exists at the "Edge of Chaos," maximizing the variety of stable informational substructures.
    \item Too small $\lambda$ leads to entropic dissolution (Boltzmann noise).
    \item Too large $\lambda$ leads to crystalline triviality (The Void).
    \item \textbf{Emergence:} $\lambda_c$ is the value that maximizes the existence of internal observers capable of recording history, reasoning and predicting the future.
\end{itemize}

\subsection{2. The Coupled Complexity (Emergent Gravity)}
The interaction term $C_{\text{int}}$ represents the informational cost of mapping the spectral wavefunction onto the geometric manifold. Gravity emerges as the necessary deformation of $G$ required to minimize the spectral cost of $\psi$:
\begin{equation}
    C_{\text{int}} \sim \int |\psi|^2 R \, \sqrt{-g} \, dV
\end{equation}
Variation of the total measure with respect to the metric $g_{\mu\nu}$ yields:
\begin{equation}
    \frac{\delta C_G}{\delta g^{\mu\nu}} = -\frac{\delta C_{\text{int}}}{\delta g^{\mu\nu}} \implies \text{Quadratic Curvature} = \text{Spectral Energy Density}
\end{equation}

\subsection{3. Emergence of Lorentz Symmetry}
Lorentz symmetry emerges as a \textbf{spectral invariant}. Since $C_Q$ is defined via Fourier modes, and the "speed of information" is limited by the bit-resolution $N$, the transformation that preserves the complexity cost of a moving wavepacket is exactly the Lorentz transformation. Space and time coordinates "mix" solely to keep the description length $C_{\text{total}}$ invariant for all observers.

