\chapter{DNA}

\section{DNA as the Blueprint of Life}

Based on the recent scientific discoveries DNA is the blueprint of life, carrying the information needed to build every living organism on Earth. The journey to understanding it has been long and fascinating.

In 1869, Dr.\ Friedrich Miescher isolated a new chemical substance from human white blood cells, which he called \emph{nuclein}. This discovery marked the first step toward understanding the molecular basis of heredity.
(\href{https://en.wikipedia.org/wiki/Friedrich_Miescher}{Friedrich Miescher},
\href{https://en.wikipedia.org/wiki/Nuclein}{nuclein})

In 1888, Theodor Boveri studied cell division and observed that tiny rods split alongside the cell. These rods were later named chromosomes.
(\href{https://en.wikipedia.org/wiki/Theodor_Boveri}{Theodor Boveri},
\href{https://en.wikipedia.org/wiki/Chromosome}{chromosome})

Thomas Hunt Morgan, studying fruit flies, created the first chromosome maps, linking traits to specific chromosomal regions. These regions were called \emph{genes}.
(\href{https://en.wikipedia.org/wiki/Thomas_Hunt_Morgan}{Thomas Hunt Morgan},
\href{https://en.wikipedia.org/wiki/Gene_mapping}{gene mapping})

In 1928, Frederick Griffith performed his famous experiment with \emph{Streptococcus pneumoniae}. By mixing heat-killed lethal bacteria with harmless living bacteria and injecting the mixture into mice, he demonstrated that some ``transforming principle'' from the dead bacteria could make the living ones lethal. This hinted at DNA as the carrier of hereditary information.
(\href{https://en.wikipedia.org/wiki/Griffith%27s_experiment}{Griffith experiment})

The double-helix structure of DNA was revealed in 1953 by James Watson and Francis Crick, based on X-ray diffraction images captured by Rosalind Franklin. The mechanism of replication was confirmed in 1958 by the Meselson--Stahl experiment, demonstrating semi-conservative replication---each new DNA molecule consists of one old and one new strand. This showed how information is faithfully transmitted from cell to cell.
(\href{https://en.wikipedia.org/wiki/James_Watson_and_Francis_Crick}{Watson and Crick},
\href{https://en.wikipedia.org/wiki/Rosalind_Franklin}{Rosalind Franklin},
\href{https://en.wikipedia.org/wiki/Meselson%E2%80%93Stahl_experiment}{Meselson--Stahl experiment})


\section{Homeobox Genes}

DNA explains the blueprint, but why do cells differentiate into muscles, teeth, or neurons when they all share the same genome? The answer lies in regulatory genes.

In the 1980s, Walter Gehring and colleagues discovered the \emph{homeobox} genes while studying fruit flies. One mutant fly had a leg growing in its head, revealing eight genes that control where and when other genes are activated. These genes are remarkably conserved across species.
(\href{https://en.wikipedia.org/wiki/Homeobox}{Homeobox},
\href{https://en.wikipedia.org/wiki/Walter_Gehring}{Walter Gehring})

A striking experiment involved transplanting a gene responsible for eye development from a mouse into a fruit fly embryo. The fruit fly developed extra, fully functioning fly eyes---not mouse eyes---demonstrating that the underlying genetic instructions were functionally conserved. This highlighted \textbf{the common ancestry of all life}, just as Charles Darwin predicted in \emph{On the Origin of Species} (1859).
(\href{https://en.wikipedia.org/wiki/On_the_Origin_of_Species}{On the Origin of Species}). 
We software developers, it appears, still have some catching up to do when it comes to code reusability.


\section{Applications}

Since DNA’s discovery, research has advanced tremendously, leading to numerous applications:

\begin{itemize}
  \item \textbf{Medicine:} Genetic testing and DNA sequencing enable early detection of disorders and improved treatments. Gene therapy can correct faulty genes.
  (\href{https://en.wikipedia.org/wiki/Genetic_testing}{Genetic testing},
  \href{https://en.wikipedia.org/wiki/DNA_sequencing}{DNA sequencing},
  \href{https://en.wikipedia.org/wiki/Gene_therapy}{Gene therapy})

  \item \textbf{Forensics:} DNA analysis is invaluable in solving crimes.
  (\href{https://en.wikipedia.org/wiki/Forensic_DNA_analysis}{Forensic DNA analysis})

  \item \textbf{Agriculture:} Genetic modification improves crop yields, pest resistance, and nutritional content. DNA barcoding aids species identification, conservation, and combating wildlife trafficking.
  (\href{https://en.wikipedia.org/wiki/Genetic_engineering}{Genetic engineering},
  \href{https://en.wikipedia.org/wiki/DNA_barcoding}{DNA barcoding})

  \item \textbf{Evolutionary Biology:} DNA sequences allow reconstruction of evolutionary histories with unprecedented accuracy.
  (\href{https://en.wikipedia.org/wiki/Molecular_phylogenetics}{Molecular phylogenetics})
\end{itemize}


\section{Digitized DNA}

Despite its complexity, DNA operates on surprisingly simple principles. Its four nucleotides---Adenine (A), Guanine (G), Cytosine (C), and Thymine (T)---pair specifically (A with T, C with G) in the double helix. This complementarity allows DNA to be replicated accurately, with each strand serving as a template for a new one.
(\href{https://en.wikipedia.org/wiki/Base_pair}{Base pairing},
\href{https://en.wikipedia.org/wiki/DNA_replication}{DNA replication})

Remarkably, DNA sequences can be translated into binary code for computers without loss of information. Entire genomes, including the human genome, are now digitized. Synthetic biology has even made it possible to build bacterial genomes from scratch, controlling living cells with artificial DNA.
(\href{https://en.wikipedia.org/wiki/Synthetic_biology}{Synthetic biology},
\href{https://en.wikipedia.org/wiki/Human_Genome_Project}{Human Genome Project})

One can imagine an inkjet-like printer that propels nucleotides instead of ink, ``printing'' life directly from digital DNA. While not commercially available yet, experiments such as those conducted by a UC Berkeley team in 2019 demonstrate that this is no longer pure science fiction.
(\href{https://en.wikipedia.org/wiki/Genome_synthesis}{Genome synthesis})


\section{Developer’s Angle}

It seems there is no ``soul'' or ``fire'' one would have to ignite in order to create life, at least for bacteria to behave as living organisms. Simply compose the desired structure from DNA molecules, place them into an appropriate environment, and one obtains ``life.''

Engineers might recognize parallels between cells and modern software development. A biological cell resembles a factory running sophisticated Computer-Aided Manufacturing (CAM): the genome is the software, and cellular machinery is the hardware executing it. Nature invented CAM long before humans did, and understanding DNA as both software and hardware provides a powerful perspective on evolution, development, and biotechnology.
