\chapter{Proof of Concept}

\section{From Formal Theory to Physical Demonstration}

The preceding chapters derive the theory at the level of axioms, measures, and limiting principles.
At this stage, the theory must be tested not philosophically, but operationally: we must demonstrate
that its defining assumptions are sufficient to produce structures resembling observed physics.

To this end, we construct a proof-of-concept simulation.
The simulation is not intended to reproduce the universe in full detail.
Rather, its purpose is sharply defined:

\begin{quote}
Given only an observer-conditioned informational measure and a compression principle,
do effective quantum behavior, inertia, and gravitational geometry emerge without being explicitly imposed?
\end{quote}

No forces, equations of motion, or spacetime structures are hard-coded.
Only information, ordering, and compression are permitted.

\section{The Role of the Observer}

The theory is fundamentally observer-centric.
Therefore, the simulation must begin with an observer, or more precisely,
with an observer-compatible informational trace.

At the implementation level, this requires specifying:
\begin{itemize}
    \item an informational capacity (a finite number of bits),
    \item Observer wavefunction - a minimal observer filter defining what constitutes observer continuity.
\end{itemize}

The simulation then explores informational configurations that contain such an observer trace.
For each compatible configuration, all admissible observer walks
(orderings of informational states consistent with observer survival)
are considered.

Each observer walk is evaluated by compressing it:
\begin{itemize}
    \item spectrally (minimal independent frequencies/phases),
    \item geometrically (minimal coherent spatial embedding).
\end{itemize}

From these compressions, a joint weight is computed.
Observer walks admitting minimal joint compression dominate the measure.

\begin{quote}
The universe an observer most likely finds themselves in
is the observer walk with maximal compressibility.
\end{quote}

Importantly, this formulation does not require that the observer or the total number of bits
be fundamental.
In the full theory, both are emergent:
the most probable observer is the one that arises in the most probable informational universe,
at the most probable scale.

The simulation fixes these quantities only to make computation possible,
not because the theory requires them as primitives.

\section{The Combinatorial Barrier}

A system with $n$ bits admits $2^n$ informational states.
The number of possible orderings (observer walks) over these states is $(2^n)!$.
This growth is super-exponential.

Even for modest $n$, exhaustive enumeration of configurations or observer walks is impossible.
Any claim to simulate the full space is therefore mathematically false.

This is not a limitation of the present work, but a structural fact.
Consequently, \emph{every} simulation of a theory of this type must:
\begin{itemize}
    \item sample,
    \item compress,
    \item constrain, or
    \item generatively construct
\end{itemize}
the state space.

These are not optional optimizations; they are unavoidable.
The relevant question is not whether a simulation is biased,
but whether the bias reflects the theory itself or is externally imposed.

\section{Theory-Driven Constraints and Optimizations}

The constraints used in the simulation are not ad hoc.
They follow directly from empirical facts and from the theory’s own measure.

For example:
\begin{itemize}
    \item Observed universes exhibit increasing entropy.
    Observer walks with global entropy decrease can therefore be excluded.
    \item Observed physics exhibits inertia and continuity.
    Sampling is restricted to geometrically coherent configurations,
    corresponding to minimal geometric description length.
\end{itemize}

These constraints do not add physics;
they eliminate observer walks that are already exponentially suppressed
by the observer-conditioned measure.

In this sense, the optimizations enforce the theory rather than distort it.

\section{Simulation Architecture}

The simulation proceeds as follows:
\begin{enumerate}
    \item Generate observer-compatible informational configurations.
    \item Enumerate or sample admissible observer walks.
    \item Compute spectral and geometric compressions for each walk.
    \item Assign joint weights according to the observer-conditioned measure.
    \item Identify dominant structures and trajectories.
\end{enumerate}

The full implementation is provided as supplementary material.
Here we present only high-level pseudocode and representative outputs.

\section{Emergent Phenomena}

Despite the absence of imposed dynamics, the simulation exhibits:
\begin{itemize}
    \item interference patterns characteristic of wave mechanics,
    \item inertial persistence of localized structures,
    \item effective attraction corresponding to geometric compression gradients.
\end{itemize}

These behaviors arise solely from the dominance of minimal-description observer walks.

\section{Interpretation}

This proof of concept does not claim numerical accuracy or cosmological completeness.
Its significance is structural:

\begin{quote}
Quantum and gravitational phenomena arise as statistical consequences
of observer-conditioned informational compression.
\end{quote}

The simulation demonstrates that the theory is not merely interpretive,
but generative.
