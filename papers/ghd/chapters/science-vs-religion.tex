\chapter{Science vs. Religion}


It is not difficult to find people who do not believe in science.
Physics would be nothing more than a religion---a belief system itself.
Instead of believing in sacred texts, one believes in scientific papers.

\section{Practicing Physics}

Physics is a field of science that is fundamentally about the study of observable phenomena.

All fields of science, including physics, follow the so-called scientific method.
The method defines how science is practiced.
First, one makes observations about the phenomenon to be studied.
Then one develops hypotheses to explain the phenomenon.
In the case of physics, these are typically described in the language of mathematics.
The new theory is then tested against available data.
Each new observation that is consistent with the predictions of the theory increases the credibility of the theory.

However, no amount of experimentation can ever prove a theory right.
Regardless of the amount of successful experimentation so far, nothing guarantees that the next experiment will support the theory.
A physical theory is always subject to falsification.
Even a single contradictory experiment can prove the theory wrong.

\section{Practicing Religion}

While different religions have different practices, there are some key elements that many of them share.
These include prophecy, prayer, rituals and ceremonies, and moral and ethical guidelines.

At the heart of all religions, however, are sacred texts and faith.
People read these texts, memorize them, and believe them.

It is important to note that rational thinking should not be applied to compare sacred texts to observations.
Experiments should not be carried out to test them.
This is because applying rational reasoning to religious texts leads to logical contradictions, and hence, doubt.
Doubt is something between believing and not believing.
All doubts are often related to Satan and his attempts to trick us away from the truth.

For example, according to some interpretations of holy texts, the Earth is only some thousands of years old.
However, we can observe annoying dinosaur fossils.
Based on science, with overwhelming observational evidence, they have to be much older.
What one observes would seem to be in total contradiction with what one believes.

These apparent contradictions can be solved by assuming God is great and so far beyond everything that no human being will ever come close to comprehending.
With our pitifully thin layer of grey brain matter, it is actually foolish to even try to question the holy texts.
God naturally put those dinosaur fossils there just to test one's faith!

One can also explain many of the apparent contradictions in sacred texts by assuming that the sacred texts are not to be taken literally.
One allows a suitable level of flexibility in their interpretation.

\section{Conclusions}

Religions require total unconditional belief, and all doubt is bad.
In science, the situation is exactly the opposite.
A theory is accepted as a scientific theory only when there is a significant amount of experimental data supporting it.
Theories of physics are always subject to experimental verification.
In fact, the most complex system humans have ever built---the Large Hadron Collider---was constructed specifically to challenge existing theories.

In science, one believes only what one sees, whereas in religions the situation is exactly the opposite.

The only conclusion one can draw by comparing science and religion is that the two concepts are fundamentally different.
Science is not a belief system.

In fact, the most central concept in physics---consistency with observations---would be lethal to religions.
If the claims of religions could be experimentally verified, there would be no room for believing anymore.
If we saw God, we would start studying the properties of the system and develop mathematical laws to model them.
Observation would turn religion into science.
