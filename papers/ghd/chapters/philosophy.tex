\chapter{Understanding Everything}

\section{Foundational Challenges in Physics}

Attempts to formulate a conventional theory of quantum gravity have repeatedly stalled, suggesting that a naive unification of gravity and quantum mechanics
may not exist within current theoretical frameworks. 

Physical theories must ultimately be tested against observation; without falsifiable consequences, a framework belongs more to philosophy than physics. Yet
the current mainstream physical theories mostly treat the observer as external or auxiliary rather than as a central element of the formalism.
Yet observation and experience are the only way the universe is ever known.

One might think that in a complete theory of everything, the observer must be central. The theory should describe both what is
observed, and the observer observing it.

Last but not least, the theories should make sense. They should allow us to understand not only how the universe works, but what it is that
works according to the laws of physics.

And the current theories fall short here. Does any of the interpretations, Kopenhagen or many-world, really make sense?


\section{Methods}

We are embedded inside the system we study, which creates a methodological problem for claims of total understanding.
However, as concluded in the Human as Axiomatic Systems chapter, whatever it is that we call time in our simulations, must emerge from
the computer running the simulation. Every bit in the simulation computer can be mapped to a particle in the simulated universe.

So when a target phenomenon can be modeled by a computer simulation that faithfully reproduces the qualitative observables (e.g., invariants, critical behaviors) of interest, the simulation's execution trace—a finite bitstring produced by running the simulator—contains the essential information needed to analyze and understand those observables.


\section{Gödel Incompleteness}

Considering how highly respected authors, such as Stephen Hawking, have noted that a Theory of Everything may be forever constrained by Gödelian limits,
we have to approach this with necessary humility. Gödel’s incompleteness theorems constrain formal systems capable of arithmetic. While an embedded observer
will inevitably encounter Gödel-style limits in their internal formal reasoning, these should not prevent us from developing full theory of everything.

The Method describe above allows us to step out of the universe which we are part of, and look it outside, from the perspective of equivalent Turing Machine.




\section{Minimal Assumptions}

So unlike all other fundamental theories of physics, we'll start with no assumptions!

And here we hit the Gödel right start. Stating "we assume nothing" is, in itself, an assumption.

So apparently we have to assume at least something. So we assume one thing, the only undeniable datum: The Observer exists.

From my personal experience, I know this for fact. I exist. I cannot be sure of anyone else, I might well imagine them. 

From this single point of certainty—the "I am"—we explore how a coherent physical universe, governed by smooth laws and complex structures, emerges.

And as I am, and I'm capable to observer, one has to make sure the theory agrees and predicts those observations.


