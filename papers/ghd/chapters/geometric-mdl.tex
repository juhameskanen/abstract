\section{Informational Complexity Functional and Emergence of Einstein-like Dynamics}

\subsection{Minimum Description Length Functional}

We interpret spacetime geometry as a compression model for worldline data.
The optimal geometry minimizes total description length:

\begin{equation}
C_{\mathrm{total}}[g;\mathcal{H}]
=
C_G[g]
+
C_Q[\mathcal{H} \mid g],
\end{equation}

where:

\begin{align}
C_G[g]
&=
\alpha \int (\partial g)^2 \, d^dx, \\
C_Q[\mathcal{H} \mid g]
&=
\beta \sum_a
\int
\left(
\ddot{x}_a^\mu
+
\Gamma^\mu_{\nu\rho}(g)\,
\dot{x}_a^\nu \dot{x}_a^\rho
\right)^2 dt.
\end{align}

Here:

\begin{itemize}
\item $C_G[g]$ represents geometric encoding cost (smoothness penalty),
\item $C_Q[\mathcal{H} \mid g]$ represents conditional worldline encoding cost,
\item $\Gamma^\mu_{\nu\rho}(g)$ are Christoffel symbols derived from $g$.
\end{itemize}

Geodesics satisfy:

\begin{equation}
\ddot{x}^\mu
+
\Gamma^\mu_{\nu\rho}
\dot{x}^\nu \dot{x}^\rho
= 0,
\end{equation}

and therefore minimize $C_Q$.

\subsection{Variation with Respect to Worldlines}

Varying $C_Q$ with respect to $x^\mu$ yields:

\begin{equation}
\delta C_Q = 0
\quad \Rightarrow \quad
\ddot{x}^\mu
+
\Gamma^\mu_{\nu\rho}
\dot{x}^\nu \dot{x}^\rho
= 0.
\end{equation}

Thus geodesics emerge as the minimal conditional encoding trajectories.

\subsection{Variation with Respect to Geometry}

We now vary the total functional with respect to the metric $g_{\mu\nu}$.

The geometric term yields:

\begin{equation}
\delta C_G
=
2\alpha \int
(\Box g_{\mu\nu})
\,
\delta g^{\mu\nu}
\, d^dx,
\end{equation}

where $\Box$ denotes the Laplacian operator in the weak-field limit.

For the worldline term:

\begin{equation}
\delta C_Q
=
2\beta \sum_a
\int
\left(
\ddot{x}^\mu
+
\Gamma^\mu_{\nu\rho}
\dot{x}^\nu \dot{x}^\rho
\right)
\delta \Gamma^\mu_{\nu\rho}
\dot{x}^\nu \dot{x}^\rho
\, dt.
\end{equation}

Since:

\begin{equation}
\delta \Gamma \sim \partial(\delta g),
\end{equation}

integration by parts yields:

\begin{equation}
\delta C_Q
=
\int
T^{\mu\nu}
\,
\delta g_{\mu\nu}
\, d^dx,
\end{equation}

where:

\begin{equation}
T^{\mu\nu}(x)
=
\sum_a
\int
\dot{x}_a^\mu
\dot{x}_a^\nu
\,
\delta^{(d)}(x - x_a(t))
\, dt.
\end{equation}

This is precisely the stress-energy tensor for point particles.

\subsection{Field Equation}

Stationarity of the total description length gives:

\begin{equation}
\delta C_{\mathrm{total}} = 0
\quad \Rightarrow \quad
\alpha \Box g_{\mu\nu}
=
\beta T_{\mu\nu}.
\end{equation}

This is the weak-field limit of the Einstein field equations:

\begin{equation}
R_{\mu\nu}
=
\kappa T_{\mu\nu}.
\end{equation}

\subsection{Conclusion}

Minimizing informational description length yields:

\begin{itemize}
\item Geodesic motion from conditional encoding minimization,
\item Einstein-like field equations from geometric encoding minimization.
\end{itemize}

Thus, in the weak-field continuum limit, the Minimum Description Length
principle reproduces the variational structure of General Relativity.

Spacetime geometry emerges as the optimal compression model
for dynamical histories.
