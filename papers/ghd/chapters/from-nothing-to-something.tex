\chapter{From Nothing to Something}

\section{The Time-Reversed Black Hole}

In the previous chapter, we established that vanishing Shannon entropy in an execution trace corresponds to geometric collapse. If every bit is identical, no distinction can be drawn between one ``location'' and another; the induced spacetime representation therefore converges to a single point. In the language of General Relativity, this limit resembles complete gravitational compression.

If entropy collapse corresponds to geometric compression, then the converse must also hold: increasing entropy corresponds to geometric unfolding.

This is not an additional hypothesis but a direct consequence of the entropy–geometry equivalence. Collapse and expansion are simply opposite directions within the same informational configuration space. In this sense, cosmic expansion may be understood as the time-reversed analogue of black hole formation. It is not an explosion of matter into a pre-existing container, but the progressive differentiation of information that manifests geometrically as increasing spatial separation.

\section{Entropy as a Driver of Expansion}

\subsection{Geometric Particle Filter}

Consider an execution trace
\[
S_t \in \{0,1\}^L, \quad t = 0, \dots, n,
\]
with fixed length $L$. We initialize the system at a state of zero Shannon entropy,
\[
S_0 = (0,0,\dots,0),
\]
which we identify with the informational singularity.

As the system evolves through random bit-flip mutations, its Shannon entropy
\[
H(S_t) = - \sum_{b \in \{0,1\}} p_t(b)\log p_t(b)
\]
increases.

To obtain a geometric representation, we introduce a decoding map
\[
D : \{0,1\}^L \rightarrow \mathbb{R}^3,
\]
which assigns subsets of bits to spatial coordinates. This induces a discrete spacetime fabric on which structural filters can be defined:

\begin{itemize}
    \item \textbf{Elementary particles:} Pairs of spatial points whose separation is less than a threshold $\varepsilon$.
    \item \textbf{Atoms:} Triplets of points forming tight, approximately equilateral triangles.
    \item \textbf{Molecules:} Clusters of atoms whose geometric centers lie within a small separation threshold.
\end{itemize}

Despite the minimal and deliberately abstract nature of this model, two robust features emerge:

\begin{enumerate}
    \item \textbf{Spontaneous hierarchical organization.}  
    As entropy increases, bits cluster into higher-order structures without explicit programming of such hierarchies.

    \item \textbf{Local stability amid global expansion.}  
    Bound structures remain coherent due to informational redundancy, while the average separation between clusters grows with entropy. Local organization persists even as the global geometry unfolds.
\end{enumerate}

The second property mirrors an essential feature of cosmology: gravitationally bound systems do not expand with the Hubble flow, whereas large-scale distances increase.

\section{The Goldilocks Zone and Open Problems}

Among the $2^L$ possible bit configurations, the zero-entropy state occupies measure zero. Yet maximal entropy corresponds to near-complete randomness, which does not support stable structure. Observers therefore require an intermediate regime:

\begin{itemize}
    \item \textbf{Low Entropy:} No distinguishable structure; geometry collapses to a point.
    \item \textbf{High Entropy:} No persistence; configurations fluctuate without stability.
    \item \textbf{Intermediate Entropy:} An unfolding phase in which hierarchical structures emerge and remain coherent.
\end{itemize}

This intermediate regime constitutes a structural ``Goldilocks zone'' in informational space: sufficiently differentiated to permit complexity, yet sufficiently constrained to preserve stability.

However, the framework leaves several fundamental problems unresolved:

\begin{itemize}
    \item \textbf{The Gravity Problem:}  
    By what mechanism do emergent informational structures generate effective attractive behavior analogous to gravitation?

    \item \textbf{The Boltzmann Brain Problem:}  
    Why does the informational ensemble favor large, lawful universes over isolated, short-lived fluctuations?

    \item \textbf{The Lawfulness Problem:}  
    Why does large-scale smoothness and predictability arise rather than persistent chaos?
\end{itemize}

These questions mark the boundary between structural emergence and dynamical explanation.

\section{Conclusion}

Geometric collapse and geometric expansion are dual aspects of the same informational principle.

Compression occurs when entropy vanishes.  
Unfolding occurs when entropy increases.

The expanding universe may thus be interpreted as an execution trace moving away from informational homogeneity toward combinatorial richness. What appears physically as cosmological expansion corresponds, at a deeper level, to the progressive differentiation of information.

