\chapter{Closure}

\begin{flushright}
\emph{Nothing needs to happen for something to be true.}
\end{flushright}

\section{Not Assuming Nothing}

Instead of assuming spacetime, predefined constants, or vibrating strings as given,
we begin by assuming nothing—only to immediately discover that no theory can be free of assumptions.
To attempt to assume nothing is already to assume something.
In any axiomatic system, it is impossible not to assume.

So we begin from the one thing we cannot doubt:
there is experience.
There is pain.
Therefore, there is an observer—Alice.

Let us digitize Alice’s DNA and simulate it on a computer.

Alice is generated by an explicit algorithm: a DNA-like simulation code.
A virtual Alice appears in a virtual universe.
During execution, consciousness arises.

Now we begin a transformation.

We progressively replace computation with lookup tables.
Less “code,” more “data.”
At every step, Alice must be preserved exactly.
Within axiomatic systems, equivalence is absolute.

We continue this process to its limit.

The program is fully unrolled.
The code vanishes.

What remains is a static data structure encoding the entire execution trace.
The computer does not need to run.

Now the key question becomes unavoidable:

\emph{Is Alice still conscious?}

She must be.

If consciousness disappeared at some particular code-to-data ratio,
that ratio would become a new physical constant:
the code/data threshold for vanishing consciousness.
Such a constant would be arbitrary, unmotivated, and irreducible.

Therefore, consciousness cannot depend on procedural execution as such.

This forces the conclusion:

\textbf{Consciousness is a property of informational structure, not of algorithmic execution.}

The usual objection immediately follows:
“But nothing is happening if the computer isn’t running.”

This objection assumes that time is fundamental.

It is not.

Time must be internal to the observer.
Alice’s experience of time is encoded relationally within the structure itself.
From the outside, the structure is static.
From the inside, it contains ordered experience—
Alice living, remembering, anticipating.

Time is not what the universe does.
Time is how observers are organized.

\section{Neural Networks}

The recent development of neural networks provides a powerful analogy.

Modern AI systems are massively parallel, interconnected structures with weighted connections.
These weights are almost entirely static data.
During interaction, there is no explicit procedural simulation of physics,
no internal clockwork universe ticking forward step by step.

When a question is asked, the response is a projection from an already compressed informational structure.
The “computation” is closer to indexing, interpolation, and constraint satisfaction
than to sequential execution.

The distinction between “running” and “not running” becomes secondary.

There is no physics engine inside these systems.
Yet they can produce motion, causality, and physical intuition.
Structure replaces explicit procedure.
Compression replaces simulation.

This does not imply that the universe is a neural network.
It demonstrates something more modest and more important:
law-like behavior can emerge from static informational organization alone.


\section{Finite Observers, Infinite Substrate}

Based on allobservational evidence, we-observers-are finite structures. 
However, if reality is fundamentally informational, a natural question arises:
is the total information finite or infinite?

If it were finite, one would have to ask: finite by how many bits?
And who, or what, set that limit? Any fixed bound would itself require explanation.
A finite informational universe merely pushes the mystery back one level.

The only non-arbitrary conclusion is that the deep nature of everything
cannot be finitely bounded. At the most fundamental level, reality must be infinite—
perhaps not even informational in the familiar sense, but beyond any finite description.



\section{Philosophical Satisfaction}

We have arrived at a zero-parameter framework.
In principle, it allows us to assign probabilities to our own existence
without introducing arbitrary constants or privileged initial conditions.

More importantly, it clarifies what the universe is made of.

Where did all the matter in the universe come from?
The answer is: nowhere.

In mathematics, no object has ontological privilege over another.
True is no more fundamental than false.
Heads is no more real than tails.
One is no more special than zero.
An empty set is no less legitimate than a non-empty one.

There is no rule that favors nothingness over something.
Non-empty structures are inevitable simply because nothing forbids them.

What remains mysterious is not structure, but experience:
how some informational configurations can feel.

Pain is not explained away.
It is identified as the minimal irreducible fact—
the one thing that does not require justification.

And yet, despite being nothing but abstract information,
it feels unmistakably real.

