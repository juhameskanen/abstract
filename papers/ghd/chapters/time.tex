\chapter{Time}

\section{The Problem of Time}

Time is a problem. In many equations of physics, time appears merely as a parameter. One can replace $t$ with $-t$ and the equations continue to work just fine. As far as the formal laws are concerned, there seems to be nothing that enforces a preferred direction of time.

A software programmer can easily implement a model of a universe using modern programming languages and simulation frameworks. Such a simulation would typically have a time slider. The animator can move the slider forward or backward at will, and the simulated universe responds accordingly. Slide it forward and someone gets eaten by a T-Rex. Slide it backward and the victim walks away unharmed.

In the real universe, however, there is no visible time slider. There is no external animator adjusting the present moment. Time flows, but only in one direction. We remember the past, not the future. Causes precede effects. Broken things do not spontaneously reassemble.

Yet the fundamental equations themselves seem largely indifferent to this asymmetry. This tension between lived experience and formal theory is the core of the problem of time.

\section{General Relativity vs.\ Quantum Mechanics}

The two most successful theories of modern physics—General Relativity and Quantum Mechanics—paint strikingly different pictures of time.

In General Relativity, time is not an external parameter but part of a four-dimensional geometric structure called spacetime. Space and time are interwoven, and their geometry is determined dynamically by the distribution of mass and energy. Objects trace world-lines through this geometry, and what we perceive as the present is simply a three-dimensional cross-section of a four-dimensional structure.

In this framework, there is no universal, continuously flowing time. Each observer measures their own proper time along their world-line. Different observers experience time at different rates depending on their motion and gravitational environment. A common interpretation of General Relativity—the so-called block universe view—suggests that past, present, and future events all coexist within the spacetime manifold. Whether this interpretation is ontologically correct remains a matter of debate, but it is consistent with the theory.

Cosmological observations indicate that the universe has evolved from a hot, dense early state roughly 13.8 billion years ago and that its large-scale expansion is currently accelerating. This evolution is not represented as motion through time toward a single point, but as the unfolding of spacetime geometry itself.

Quantum Mechanics offers a very different perspective. In quantum theory, physical systems are described by wavefunctions that encode all available information about their states. These wavefunctions evolve deterministically in time according to the Schrödinger equation, yet the outcomes of measurements are fundamentally probabilistic.

Importantly, quantum mechanics does not render time itself indeterminate. The uncertainty lies in observable quantities, not in the existence or continuity of time. Time remains an external parameter in standard formulations of quantum mechanics, unlike in General Relativity where it is part of the dynamical structure.

Some speculative theories suggest the existence of a smallest meaningful time scale, often associated with the Planck time, but this is not a prediction of quantum mechanics itself. It is an indication that our current theories may break down at extreme scales, and that a deeper framework—quantum gravity—may be required.

Other approaches highlight the tension further. The Wheeler–DeWitt equation, for example, describes a formalism in which time does not appear explicitly at all. String-theoretic models allow scenarios in which the Big Bang is not the beginning of time, but a transition between different geometric phases. None of these ideas has yet produced a complete and experimentally verified theory.

Despite their differences, both General Relativity and Quantum Mechanics share an important feature: their fundamental equations are largely time-symmetric. They do not, by themselves, explain why time appears to flow in one direction only.

\section{Strange Temporal Phenomena}

At small scales, nature exhibits behaviors that appear to challenge our classical intuitions about time.

Quantum tunneling allows particles to appear on the other side of potential barriers that would be impenetrable in classical physics. While tunneling does involve subtle temporal behavior, it does not occur instantaneously or without time entirely, even if its timing is difficult to define precisely.

Entangled particles exhibit correlations that persist across arbitrary distances. Measuring one particle instantaneously constrains the state of its partner. However, no information is transmitted faster than light; the correlations cannot be used for communication.

Photons traveling along null geodesics experience zero proper time between emission and absorption. From the photon's own perspective—if such a perspective could be meaningfully defined—no time elapses during its journey. This is a geometric property of spacetime, not a violation of causality.

None of these phenomena, however, reverse the arrow of time. They stretch our intuitions, but they do not allow macroscopic objects—or observers—to move backward into their own past.

\section{Entropy and the Arrow of Time}

Since the fundamental equations of physics do not impose a temporal direction, it is often argued that the arrow of time arises from the second law of thermodynamics.
In an isolated system, entropy tends to increase. This provides a statistical distinction between past and future.

Stephen Hawking famously illustrated this using a video recording. A film showing entropy decreasing—shattered objects reassembling, smoke returning to cigarettes—immediately appears unnatural. Entropy allows us to distinguish whether a process is running forward or backward.

Entropy also underlies biological time. Living systems require ordered energy sources. We eat food that is highly structured and expel waste that is more disordered. A universe in which entropy decreased would not support life as we know it.

Yet this explanation raises a deeper question: why should entropy increase at all? Why does the universe begin in a state of such extraordinarily low entropy?



\section*{Time Reversal in the Spectral Compression Framework}

Can time run backwards? Within this framework the question is, does it informationally cost more.
The magnitude of Fourier coefficients is invariant under time reversal. Complex conjugation changes phase
but $|F(k)|$ does not change, so time reversal costs exactly the same.

The compression functional is time-reversal symmetric.

Time always runs forward, because the algorithm is asymmetric.

The entire trajectory exists as a static object. “Flow” is just traversal of that index.
So reversing time is just reading the list backwards.

The compression principle evaluates the whole list at once. Therefore it should not care about ordering direction.



\subsection{Ontology of Time}

Can time be reversed? In this framework, time is not a fundamental parameter.  
An observer trajectory is a finite ordered sequence:

\[
\{ x_0, x_1, x_2, \dots, x_T \}
\]

Time is simply the ordinal index of this sequence.

The universe is therefore a static informational object.  
Temporal evolution corresponds to traversal of the index.

\subsection*{Time Reversal Operation}

Time reversal corresponds to reversing the sequence:

\[
\{ x_0, x_1, \dots, x_T \}
\quad \longrightarrow \quad
\{ x_T, x_{T-1}, \dots, x_0 \}
\]

Denote the trajectory signal by \( f(t) \).  
Time reversal is:

\[
f(t) \rightarrow f(T - t)
\]

\subsection{Spectral Complexity Functional}

The spectral complexity is defined as:

\[
C = \sum_{k} k_{\mathrm{eff}}^2 \, |A_k|^2
\]

where:

\[
A_k = \mathcal{F}[f](k)
\]

and

\[
k_{\mathrm{eff}} = \min(k, N-k)
\]

\subsection{Fourier Behavior Under Time Reversal}

If

\[
f(t) \rightarrow f(T - t)
\]

then its Fourier transform transforms as:

\[
A_k \rightarrow A_k^{*}
\]

That is, time reversal produces complex conjugation in Fourier space.

Importantly:

\[
|A_k|^2 \rightarrow |A_k^{*}|^2 = |A_k|^2
\]

\subsection*{Invariance of Complexity}

Since the complexity depends only on magnitude squared:

\[
C = \sum_{k} k_{\mathrm{eff}}^2 \, |A_k|^2
\]

and magnitudes are invariant under complex conjugation,

it follows that:

\[
C_{\text{forward}} = C_{\text{reverse}}
\]

Therefore, the spectral compression functional is time-reversal symmetric.

\subsection*{Interpretation}

The static informational object (the full trajectory) has no preferred time direction.

The apparent arrow of time arises only from the sampling procedure,
which appends new elements to the end of the sequence.

The ontological object itself is symmetric:

\[
\{ x_0, \dots, x_T \}
\quad \text{and} \quad
\{ x_T, \dots, x_0 \}
\]

have identical informational complexity.

\subsection*{Conclusion}

The spectral compression principle:

\[
C \sim \sum k^2 |A_k|^2
\]

is intrinsically time-reversal invariant.

Any emergent arrow of time must therefore arise
from boundary conditions or asymmetric sampling dynamics,
not from the compression functional itself.



\section{Is Time Traveling Possible?}

Jumping back in time is totally different question.

According to Thermodynamics, in an isolated system, total information cannot be created or destroyed—only transformed and redistributed. 
It cannot be proven, but it is generally assumed to hold. 

Now consider what travel to the past would imply. You we in an isolated environment, say inside your living room. A macroscopic object suddenly appearing from the future would
introduce a large amount of mass-energy and information into an otherwise closed system without any causal precursor.

An $80\,\mathrm{kg}$ human corresponds to roughly $7 \times 10^{18}\,\mathrm{J}$ of energy. Such an appearance would violate global information accounting and render thermodynamics meaningless.

How does this translates to our framework?

According to Thermodynamics, in an isolated system, total information cannot be created or destroyed—only transformed and redistributed. 
It cannot be proven, but it is generally assumed to hold. 

In our framework, time is not a fundamental substance that flows.
The universe is a static collection of informational configurations.
An ``observer'' is simply a path through those configurations:

\[
\{x_0, x_1, x_2, \dots, x_T\}.
\]

The index of this sequence plays the role of time.
Nothing is moving in an external sense; the observer simply traverses
configurations in a particular order.

\vspace{1em}

\subsection*{What Would Time Travel Mean?}

Reversing the order of the entire sequence,

\[
\{x_0, x_1, \dots, x_T\}
\quad \longrightarrow \quad
\{x_T, x_{T-1}, \dots, x_0\},
\]

is mathematically trivial and costs nothing.
This corresponds to reading the same path backwards.
The underlying description length remains exactly the same.

But this is not what we usually mean by ``time travel.''

Instead, time travel would mean something like this:

\[
x_{t+1} \approx x_{t_0}
\quad \text{with} \quad t_0 \ll t.
\]

In other words, the observer suddenly jumps from its current configuration
to one that occurred far earlier along its path.

That is a discontinuity.

\vspace{1em}

\subsection*{Why Discontinuities Are Expensive}

In this theory, the probability of a trajectory depends on how
compressible it is.
Smooth trajectories are easy to compress.
Abrupt jumps are not.

Mathematically, we measure compressibility using a spectral decomposition.
If we write the trajectory as a signal $f(t)$, we expand it in Fourier modes:

\[
f(t) = \sum_k A_k e^{i k t}.
\]

The complexity is defined as

\[
C = \sum_k k^2 |A_k|^2.
\]

This formula penalizes high frequencies.
Low-frequency components describe smooth variation.
High-frequency components describe sharp changes.

A smooth trajectory concentrates its weight in small $k$.
A sudden jump behaves like a step function.
Its Fourier coefficients decay slowly:

\[
|A_k| \sim \frac{1}{k}.
\]

Substituting into the complexity measure:

\[
k^2 |A_k|^2 \sim k^2 \cdot \frac{1}{k^2} = \text{constant}.
\]

Thus the total complexity grows roughly as

\[
C \sim \sum_k \text{constant},
\]

which increases with the number of modes included.
In plain language:
a sharp jump excites \emph{all} frequencies.
It is extremely costly in terms of description length.

\vspace{1em}

\subsection*{What About Returning Smoothly?}

There is an important distinction.

If the trajectory gradually curves back,
retracing earlier configurations step by step,
then the motion remains smooth.
Its spectrum remains dominated by low frequencies.
Such a loop would not be heavily penalized.

Therefore, returning to earlier states is not forbidden.
Abrupt teleportation is what becomes prohibitively expensive.

\vspace{1em}

\subsection*{Information and Physical Intuition}

In an isolated system, information is assumed to be conserved.
An $80\,\mathrm{kg}$ human corresponds to roughly

\[
E \approx 7 \times 10^{18}\,\mathrm{J}.
\]

If such a person were suddenly to appear from the future,
this would seem to inject enormous mass-energy and information
into a closed system without causal precursor.

In our framework, however, the deeper conserved quantity
is compressibility.
The real ``cost'' of time travel is not energy,
but the explosion of spectral complexity.

\vspace{1em}

\subsection*{Conclusion}

Is time travel possible?

Within this model:

\begin{itemize}
\item It is not logically forbidden.
\item It does not violate the symmetry of the underlying equations.
\item But abrupt jumps to the past dramatically increase
      the description length of the trajectory.
\item Therefore they are overwhelmingly disfavored.
\end{itemize}

Time travel is not impossible by law.
It is simply non-compressible.

And in a universe governed by minimal description length,
non-compressible histories are extraordinarily unlikely.
