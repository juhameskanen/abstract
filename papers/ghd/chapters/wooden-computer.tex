\chapter{Wooden Computer}

The brain of a computer---its central processing unit (CPU)---consists of a set of electric switches called transistors. The CPU does not need to be an electric device. Equally well, one might implement the required logic using a set of wooden parts, for example.

In theory, one could implement a DNA simulation as a mechanically operating computer consisting of wooden components. Instead of using transistors in a silicon chip to control electrons, one could use wooden parts on a plywood platform to control wooden balls. When such a machine stepped through its logic, a virtual human would take its first steps in its virtual universe.

What is this strange phenomenon that creates consciousness and pain from a jerking pile of wooden pieces?

If a huge number of moving wooden components can create pain, then what does one moving piece of wood create?

Can current physics even describe this action?

No man-made device is perfect, and wooden \texttrademark{} is no exception. Friction, tolerances, and things like that introduce resonances and other unintentional vibrations to the operation of wooden \texttrademark{}. If the actual logic in the wooden \texttrademark{} creates a virtual universe with pain and consciousness, then what do these unintentional side effects create? Do they get reflected in some form into the created virtual world too?

Would the virtual fellow in its virtual universe discover these in the form of strange quantum foam? Would it observe them as strange cosmic background radiation with 2.725\,K temperature?

Maybe so; at least it would be difficult to argue why large movements of wooden components would count, but their small resonances would not.

How would the clock speed of the machine running the simulation appear in the created simulated universe? Would it observe that particles in its universe appear to follow some strange abstract square wave function, whose origin it could not explain, but which it might end up calling Wintel's abstract square wave function?

In addition to Turing machines, it is easy to picture other systems creating virtual universes. One possible source of consciousness could be the surface of a sea. In theory, waves and ripples of a sea could describe a DNA simulation in which conscious observers marvel at the wonderful properties of their universe (like those caused by heavy rain during the annual monsoon season).

One could also use a thermostat to describe any procedure by controlling the temperature. The temperature could vary in time so that the thermostat would go through the binary code of the DNA procedure. Of course, the simulated fellow would be totally unaware of the fact that a trivial thermostat is responsible for the illusion of its existence. Obviously, the thermostat itself cannot be regarded as conscious by any means. Correspondingly, running such DNA simulations on any type of computer does not make the computer itself conscious or pain-sensitive. It is still the very same trivial \texttrademark{} stepping through its symbol tape without any choice. However, the running \texttrademark{} creates a virtual parallel universe in which a conscious human wonders about free will.

The whole universe, with its planets and stars, could be the source of a consciousness.

What do the five billion human beings create when walking along streets and passing through doors on their way to work? For example, a person walking through two subsequent doors implements the logical operation called \texttt{AND}. If one can pass through using either the left or right door, then one gets the \texttt{OR} operation.

Could even a regular pencil and a piece of paper be the source of consciousness? Start writing down the evolution of DNA with pencil and pen, and soon virtual people suffer tooth pain in their virtual universe. Both pencil and ballpoint pen should work equally well. Due to the higher friction, the temperature of the cosmic background radiation in a universe created with pencil might be a bit higher though!

So we can play with electrons, or even wooden components, to create virtual universes that duplicate the structures in our universe. Because the simulated fellow would be a virtual duplicate of its real-world counterpart, it too will start exploring its universe, sooner or later (we did!). It will discover that it can create things like Turing Machines. Sooner or later, it would build one and simulate its own existence. The procedure the virtual fellow uses to simulate its own existence is the same procedure that we real-world humans used to create the first simulation level. In other words:

\begin{verbatim}
virtual_1 = DNA(virtual_0)
\end{verbatim}

And apparently this stack of nested simulations would continue forever, or as long as there is enough information in the sub-simulation to describe yet another sub-simulation. We know that the virtual fellow we created in our simulation, as well as the virtual fellow our simulated fellow created, are both virtual. We know it because we created it all in a Turing Machine whose deterministic operation we know precisely. Therefore, the above two equations can be unified into one equation of the form:

\begin{verbatim}
r_{n+1} = DNA(r_n);
\end{verbatim}

The axiomatic \texttt{DNA()} procedure that maps our real world $r_n$ to a virtual world $r_{n+1}$ is exactly the same procedure that maps the virtual world $r_{n+1}$ to another virtual world $r_{n+2}$. There is no single parameter in the equation that would make us real fellows distinctive from these simulated fellows. The only logical conclusion one can draw from this is that this universe of ours is precisely as virtual by nature as the universes we create in our computers. With recursive procedures, it is very difficult to argue that one recursion level would somehow be more real than the others. Just like with the case of the simulated fellow, we real-world fellows don't see that it is all virtual by nature.

There is apparently one very serious problem with the simulation hypothesis: entropy. It is an extremely difficult job to build a system that can simulate even one human DNA strain, let alone a complete human being. There are six billion of us, not to mention other animals, plants, bacteria, and such. Our Milky Way galaxy consists of hundreds of billions of stars similar to our Sun, and hundreds of billions of galaxies have been observed.

Those sub-simulations would soon run out of information!

Despite their purely virtual nature, these abstract constructions contain very nice features. Pain and joy are very cool properties for us conscious human beings. Could there even be some procedures of feelings that evolution hasn't implemented yet for us? Could we have those when we end up in heaven?

Those unknown human experiences could certainly explain the behavior of my wife; she is obviously running all of the procedures simultaneously.

In one Arnold Schwarzenegger movie \cite{lastactionhero1993}, people switched between the real world and the movie world. I never liked the movie because the idea felt way too absurd to me. Silly me.
