\chapter{Magic}

If one allows even a single paranormal creature to exist, what prevents there being a whole flock of them?

\section{Paranormal Experience}

There is only one somewhat ``paranormal'' incident I can be certain of—one I experienced myself. It was a dark night when, as a young student, I suddenly sensed that someone had entered my room. I tried to get up and turn on the light, but a strange low-frequency sound (about 50 Hz) emerged right behind my head. The harder I tried to move, the louder it became, until even breathing felt impossible. I gave up resisting—and immediately, I could breathe again. Seconds later, the sound (and the ``visitor?'') vanished. I was free to move and found no one in the room. I was certain I wasn't dreaming.

Initially, I might have dismissed the incident as a hallucination, but later I heard an older lady describe the exact same phenomenon on a radio program. The only difference was that she also saw a tunnel with a light at the end. I never saw a tunnel, let alone a light (should I be worried?). Still, the buzzing sound incident stuck with me for years.

Soon after, I found a leaflet from a religious group claiming modern science was a scam. It offered ``proofs,'' such as a case where Carbon-14 dating supposedly showed an animal to be ancient even though it had died yesterday. Naturally, I believed it.

As a boy, I thought my father was a smart man. Despite lacking a formal education due to the war, he understood complex topics—percentage calculus, for example. So when he told me that some people can feel underground water flows, I believed him too. He even suspected those flows could have harmful effects on people sleeping nearby. Indeed, my grandmother was living proof! One day, we ran our own experiment with divining rods. We failed to find a single water flow.

Then there was my best friend, who swore by a certain paranormal phenomenon: two people place their hands over the head of a third, concentrate, and after a few minutes, they can lift them using only their fingertips, ``defying gravity.'' Finally, I thought, here was a chance to prove the paranormal! We gathered a group and tried. We concentrated with all our might, slipped our fingers under the seated person, and... nothing. He stayed firmly in the chair. We even switched roles, suspecting that one of us was subconsciously not concentrating hard enough, but gravity remained annoyingly consistent.

Not even the classic method of altering concentration—drinking lots of beer—made a difference. Gravity was unimpressed. Alcohol, however, had other noticeable effects the next morning.

One of my teachers was also convinced of spiritual creatures, insisting we simply lacked the senses to see them. ``With our tiny human eyes, we can't even see infrared!'' he said. So, during my army service, I finally tried infrared night-vision goggles. To my disappointment: no glowing demons, no invisible spirits, nothing! And what could possibly be more infrared than Satan?

Later, I discovered a university study where 32 dowsers attempted to locate underground water veins in a double-blind test. Not a single success. When I brought this up to a colleague who swore he could dowse, he scoffed. So I blindfolded him and asked him to repeat the trick. Without being able to see, he couldn't even remember the spots he’d pointed out minutes earlier. Apparently, water flows are highly mobile—especially when your eyes are covered.

I was also told that special supplements—up to and including LSD—could ``expand the mind'' to perceive truths beyond reality. After so many failed experiments, I wondered: how would this one be different?

If the brain is an informational processor, then drugs do not ``open a door'' to a hidden dimension; they simply disrupt the local hardware. Think of the brain as a high-resolution camera lens meant to capture a clear image of reality. If you crack the lens or smear it with oil, the resulting image might look ``otherworldly'' or ``trippy,'' but you aren't seeing a hidden world—you are seeing the failure of the equipment. A malfunctioning camera doesn’t reveal ghosts; it just produces artifacts, noise, and chromatic aberration. In the same way, a chemically scrambled brain produces ``information noise'' that we mistake for ``spiritual insight.'' It is a failure of the processing logic, not a breakthrough into new data.

In the end, the only mysterious phenomenon I still cannot explain is that strange 50 Hz buzzing. According to my parents, I was born with bluish skin, likely due to a lack of oxygen during labor. Perhaps the other lady with the tunnel-and-light story was also born blue. That seems more likely than a paranormal visitor buzzing in my bedroom at midnight. Perhaps there is no such thing as magic—just a temporary lack of oxygen.

And then there is James Randi’s famous \emph{One Million Dollar Paranormal Challenge}. Surely a million dollars is motivation enough to demonstrate real magic. But no one has ever collected the prize.

\section{Definition of Magic}

How, then, should we define ``magic''?

By categorizing the natural and the supernatural, one notices that magical creatures all share the same trait: non-physicality. Magic appears to defy the laws of physics—laws based on observation and mathematics. Thus, in the spirit of rigorous definition:

\[
\text{Magic} \neq \text{Physics}
\]

By definition, magic must contradict physics, or else it would simply \emph{be} physics. And since physics rests on observation and axioms, magic must rest on either non-axioms or non-observation. That is, it cannot be observed, and it cannot be explained in terms of axiomatic systems.

Mathematics, an axiomatic system, is the study of logical reasoning. A non-axiomatic system, therefore, is the study of non-logical reasoning. 

\[
\text{Magic} = \text{Non-sense}
\]

The best synonym for non-logical reasoning is perhaps \emph{nonsense}.
