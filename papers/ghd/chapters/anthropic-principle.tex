\chapter{Anthropic Principle}


The ``Anthropic Principle'' states that the universe must be compatible with the existence of intelligent observers. We should not therefore wonder why everything appears to be so delicately adjusted to make our existence possible. If this wasn't the case, then there wouldn't be us either.

According to Stephen Hawking, the universe contains vastly more galaxies than strictly needed for life to develop on one planet, suggesting that a strong anthropic principle (requiring life everywhere) is unnecessary. Only one would be enough to produce the raw materials and get us important humans developed.

One might question this reasoning. If the probability for development of intelligent life happened to be extremely small, then isn't this huge universe precisely what one needs to get intelligent life developed at least on one planet? So it actually boils down to probabilities.

Even if the probability of life turned out to be high, maybe God, or whoever wanted to get us created, is a perfectionist and only good (e.g., sin-free) humans will do—which we apparently are not.

Or maybe God is impatient! God doesn't want to wait 1\,400\,000\,000\,000 billion years. Why not create 14 billion galaxies and wait only one year?

If there is God, then who created God? If God was never created but simply \emph{is}, then why couldn't the universe simply \emph{be} as well? Both constructions would be rather magnificent to exist. However, if God created the universe, then God is apparently more intelligent and complex. Correspondingly, the probability of God is smaller than the probability of the universe.

Can we resist the temptation to play with our genome? Researchers are currently trying hard to create artificial self-aware systems on all possible fronts. Also, it does not really matter whether God \emph{just is}, or whether he was created (possibly by another God), because in both cases there would be God. So we might as well strip all these obvious and false blocks out of the code, which simplifies the diagram to the following form:

\begin{verbatim}
if (God exists) {
    // God creates a man, 
    // for his own image creates he him
} else {
    // man creates virtual man 
    // for his own image creates he him
}
\end{verbatim}

There might soon be God, sorts of, many of them. This happens as soon as we get the first DNA simulations executed, or self-aware conscious AI systems developed. We humans will then play the role of God, and those simulated fellows play the role of sinful humans—trying to turn to science and prove God wrong.

Many with a scientific state of mind dislike the strong version of the anthropic principle. However, if the two DNA assumptions hold, then everything in this universe is due to the anthropic principle.
