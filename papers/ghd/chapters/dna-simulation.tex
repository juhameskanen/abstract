\section{DNA Simulation}

\subsection{Thought Experiment}

The current technology does not allow us to run full scale human DNA simulations yet. However, that does not
prevent us from running it as a thought experiment. So we digitize a human genome and run it inside a sufficiently 
detailed computer simulation. We also simulate a sufficiently large world with it to avoid our simulated human
to develop phsychosis in empty space.  Nine months later, in simulation time, our virtual copy takes its first breath 
in its simulated world.


\subsection{The Nature of the Simulated World}

How would such a simulated human perceive its environment? Would it sense the limited memory space of the 
computer running it? Would it be able to bump its head against the upper boundary of RAM and feel pain? 
Would the flipping of bits tickle its nose, or the rotation speed of a hard drive make it dizzy?

Would it eventually discover that its entire universe is driven by storage devices, memory chips, and an 
overclocked multi-core CPU?

\subsection{Sense of Reality}

The simulated human is not created \emph{within} our universe. It does not consist of real-world particles such 
as electrons or quarks. Instead, it exists entirely within a virtual universe that we simulate alongside it. 
As a result, it has no access to our physical hardware. It cannot observe transistors, memory cells, voltages, 
or processor clocks.

The only thing the simulated human can study is the internal structure of its own virtual world.

Within that world, there are virtual particles, virtual forces, and virtual laws of physics. When the 
simulated human bangs its head against a simulated wall, the simulated particles in the wall respond exactly 
as the laws of that virtual universe dictate - including pain. To the simulated observer, the experience is indistinguishable 
from how real particles behave when we humans bang our heads against real walls.

Every measurement the simulated human performs inside its universe will be internally consistent. 
The outcomes of experiments will match the predictions of the simulated physical laws, just as our 
measurements match the laws of physics in our own universe.

This is because both the real world and the simulated world are \emph{axiomatic systems}. Mathematics does 
not care whether it is applied to apples, bananas, electrons, or bits. The statement ``$2 + 2 = 4$'' holds 
regardless of the physical substrate that implements the system.

For the simulated human, its universe is not an approximation, an illusion, or a shadow of reality. 
It is \emph{reality}. There is no experiment it could perform that would reveal the presence of the computer 
running the simulation, because that computer exists outside the axioms of its universe.

From the inside, the simulated universe would feel precisely as real as our universe feels to us.
