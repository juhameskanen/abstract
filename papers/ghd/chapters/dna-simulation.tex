\chapter{Running DNA Simulations}

\section{Thought Experiment}

The current technology does not allow us to run full scale human DNA simulations yet. However, this does not prevent us from exploring it as a thought experiment.
So we digitize a human genome and run it inside a sufficiently 
detailed computer simulation. We also simulate a sufficiently large world with it to avoid our simulated human
developing phsychosis in empty space.  Nine months later, in simulation time, our virtual copy takes its first breath 
in its simulated world.


\section{The Nature of the Simulated World}

How would such a simulated human perceive its environment? Would it sense the limited memory space of the 
computer running it? Would it be able to bump its head against the upper boundary of RAM and feel pain? 
Would the flipping of bits tickle its nose, or the rotation speed of a hard drive make it dizzy?

Would it eventually discover that its entire universe is driven by storage devices, memory chips, and an 
overclocked multi-core CPU?

\section{Sense of Reality}

The simulated human is not created \emph{within} our universe. It does not consist of real-world particles such 
as electrons or quarks. Instead, it exists entirely within a virtual universe that we simulate alongside it. 
As a result, it has no access to our physical hardware. It cannot observe transistors, memory cells, voltages, 
or processor clocks.

The only thing the simulated human can study is the internal structure of its own virtual world.

Within that world, there are virtual particles, virtual forces, and virtual laws of physics. When the 
simulated human bangs its head against a simulated wall, the simulated particles in the wall respond exactly 
as the laws of that virtual universe dictate, causing pain. To the simulated observer, the experience is indistinguishable 
from how real particles behave when we humans bang our heads against real walls.

Every measurement the simulated human performs inside its universe will be internally consistent. 
The outcomes of experiments will match the predictions of the simulated physical laws, just as our 
measurements match the laws of physics in our own universe.

This is because both the real world and the simulated world are \emph{axiomatic systems}. Mathematics does 
not care whether it is applied to apples, bananas, electrons, or bits. The statement $2 + 2 = 4$ holds 
regardless of the physical substrate that implements the system.

For the simulated human, there is no experiment it could perform that would reveal the presence of the computer 
running the simulation, because that computer exists outside the axioms of its universe.

From the inside, the simulated universe would feel precisely as real as our universe feels to us.



\section{Substrate Independency}

The brain of a computer---its central processing unit (CPU)---consists of a set of electric switches called transistors. The CPU does not need to be an electric device.
Just as $2+2=4$ holds for both apples and bananas, simulations should work regardless of the substrate on which they are implemented.

In theory, one could implement a DNA simulation as a mechanically operating computer consisting of wooden components. Instead of using transistors in a silicon chip to control electrons, one could use wooden parts on a plywood platform to control wooden balls. When such a machine stepped through its logic, a virtual human would take its first steps in its virtual universe.

What is this strange phenomenon that creates consciousness and pain from a jerking pile of wooden pieces?

If a huge number of moving wooden components can create pain, then what does one moving piece of wood create?

Can current physics even describe this action?

No man-made device is perfect, and wooden \texttrademark{} is no exception. Friction, tolerances, and things like that introduce resonances and other unintentional vibrations to the operation of wooden \texttrademark{}.
If the actual logic in the wooden \texttrademark{} creates a virtual universe with pain and consciousness, then what do these unintentional side effects create? Do they get reflected in some form into the created virtual world too?

Would the virtual fellow in its virtual universe discover these in the form of strange quantum foam? Would it observe them as strange cosmic background radiation with 2.725\,K temperature? Maybe that 
indeed explains why we measure quantum foam and cosmic background radiationin our universe. We are being simulated in a wooden Universal Turing Machine!

Maybe not so, but at least it would be difficult to argue why large movements of wooden components would count, but their small resonances would not.

How would the clock speed of the machine running the simulation appear in the created simulated universe?
Would the simulated human  observe that particles in its universe appear to follow some strange abstract square wave function, whose origin it could not explain, but which it might end up calling Godel's (TM) abstract square wave function?
Due to the large number of cores, the simulated human would conclude that the function must be complex-valued with phase coherence, and imaginary numbers would provide a natural formalism.

In addition to Turing machines, it is easy to picture other systems creating virtual universes. One possible source of consciousness could be the surface of a sea. In theory, waves and ripples of a sea could describe a DNA simulation in which conscious observers marvel at the wonderful properties of their universe (like those caused by heavy rain during the annual monsoon season).

Computer software is nothing but a sequence of bits, trivial on/off switches. Correspondingly, one could also use a thermostat to describe any procedure.
Let's say the temperature varied in time so that the thermostat would go through the binary code of the DNA procedure. Of course, the simulated fellow would be totally unaware of the fact that a trivial thermostat is responsible for the illusion of its existence. Obviously, the thermostat itself cannot be regarded as conscious by any means. Correspondingly, running such DNA simulations on any type of computer does not make the computer itself conscious or pain-sensitive. It is still the very same trivial \texttrademark{} stepping through its symbol tape without any choice. However, the running \texttrademark{} creates a virtual parallel universe in which a conscious human wonders about free will.

A person walking through two subsequent doors implements the logical operation called \texttt{AND}. If one can pass through using either the left or right door, then one gets the \texttt{OR} operation.
What do the five billion human beings create when walking along streets and passing through doors on their way to work? 

Could even a regular pencil and a piece of paper be the source of consciousness? Start writing down the evolution of DNA with pencil and pen, and soon virtual people suffer tooth pain in their virtual universe. Both pencil and ballpoint pen should work equally well. Due to the higher friction, the temperature of the cosmic background radiation in a universe created with pencil might be a bit higher though!

The only conclusion one can draw from these is that whatever it is that we call consciousnes and pain must be subtrate independent. The source of pain cannot be any
physical attribute, such as mass, electric charge, elementary particle such as photon, because it is always possible to find an implementation where such a property does not play any role.

The whole universe, with its planets and stars, could be the source of a consciousness.

The only common factor between different implementations appears to be the logic they are running. And logic is ultimately abstract information. 



\section{Recursion and the Ontological Parity of Simulations}

If we can utilize electrons or even macroscopic components to create virtual universes that replicate the biological and structural motifs of our own—such as DNA—we reach a logical crossroads. Because these simulated observers are functional duplicates of their "real-world" counterparts, they will inevitably begin exploring their own substrate. 

They will discover the principles of computation and, eventually, construct their own Turing Machines. The procedure these virtual entities use to simulate their own existence is identical to the procedure we used to create them. We can express this transition mathematically. If our world is $r_n$ and the simulated world is $r_{n+1}$, the mapping is:
\[ r_{n+1} = f_{\text{DNA}}(r_n) \]

This nested stack of simulations continues as long as the host level contains sufficient computational density to support the sub-level. Because we know the deterministic nature of the Turing Machine we used to initiate the first step, we must admit that the relationship is strictly recursive:
\[ r_{n+k} = f_{\text{DNA}}(r_{n+k-1}) \]



In a recursive formula, it is notoriously difficult to argue for "ontological seniority." There is no parameter within the $f_{\text{DNA}}$ function that distinguishes a "real" world from a "virtual" one; the operator remains invariant across all levels of the recursion. The logical conclusion is that our "base" reality is as computationally contingent as the simulations we produce. To an observer inside the recursion, the substrate is always invisible; we perceive our level as "solid" simply because we are defined by the same logic that governs it.

\subsection{The Entropy Barrier}

There is, however, a significant physical constraint to this hypothesis: the \textit{Information Bottleneck}. Simulating the human genome, let alone the consciousness of eight billion humans and the staggering complexity of $10^{22}$ stars in the observable universe, requires an astronomical amount of information. 

According to the Bekenstein Bound, the maximum information $I$ contained in a sphere of radius $R$ is limited by its energy $E$:
\[ I \le \frac{2\pi RE}{\hbar c \ln 2} \]
A sub-simulation, by definition, must be contained within a subset of the host's resources. Unless the "Parent" universe is infinitely more complex than our own, each successive level of recursion must be lossy—a lower-resolution "sketch" of its predecessor. 

\subsection{Subjective Experience and Unknown Procedures}

Despite their virtual nature, these abstract constructions possess very "real" emergent properties. Pain, joy, and consciousness are the qualitative outputs of these recursive procedures. This raises an intriguing possibility: are there "emotional procedures" or modes of consciousness that evolution has simply not yet "compiled" for our species? 

Perhaps these undiscovered human experiences could explain the more baffling behaviors we encounter in our daily lives—some individuals seem to be running all possible emotional procedures simultaneously. 

In the film \textit{Last Action Hero} \cite{lastactionhero1993}, characters transition between the "real" world and a cinematic one. While the premise was presented as a slapstick fantasy, the underlying logic is less absurd than it appears. If reality is recursive, the boundary between the "actor" and the "script" is merely a matter of which level of the function you currently occupy.
