\chapter{Understanding the $D-\psi-G$ Framework}

\section{Motivation and Empirical Starting Points}

Empirical facts motivate the $D-\psi-G$ framework. Despite decades of effort, no theory has successfully unified the quantum wavefunction ($\psi$) with spacetime geometry ($G$) at a fundamental level.
Based on conclusions from Paper-1, time is an internal property of observers, and observer states (e.g., pain, reasoning, decision-making) have measurable causal effects on future observer histories.

These facts strongly suggest that neither $\psi$ nor $G$ is ontologically fundamental. Instead, they point to a deeper substrate in which observer execution, quantum amplitudes, and geometry appear as complementary descriptions.

\textbf{Conclusion.} The discrete observer ($D$), the quantum description ($\psi$), and geometry ($G$) are mutually orthogonal projections of a single underlying informational structure.


\section{The Three Projections}

We distinguish three irreducible but related descriptions:

\begin{itemize}
    \item \textbf{$D$ --- Discrete observer description.} A concrete, finite informational realization: execution traces, bitstrings, Turing-machine states, memory updates, and reasoning steps. This is what is directly \textit{experienced}.
    \item \textbf{$\psi$ --- Quantum (spectral) description.} A compressed, probabilistic encoding of ensembles of $D$-compatible execution traces. $\psi$ is efficient, lossy, and non-ontic by itself.
    \item \textbf{$G$ --- Geometric description.} A large-scale, coarse-grained spacetime representation capturing stable regularities across ensembles of histories.
\end{itemize}

No single description is complete. However, \textbf{any two constrain the third, up to gauge freedom}.

\section{Orthographic Projection Analogy}

In technical drawing, a 3D object cannot be reconstructed from a single 2D projection. Each view is lossy but consistent; orthogonality ensures each captures unique information.

\begin{table}[h]
\centering
\begin{tabular}{|l|l|}
\hline
\textbf{Technical drawing} & \textbf{$D-\psi-G$ framework} \\ \hline
3D object & Abstract informational structure \\ \hline
Top view & $D$ (observer execution trace) \\ \hline
Front view & $\psi$ (interference / amplitude structure) \\ \hline
Side view & $G$ (geometric / curvature structure) \\ \hline
\end{tabular}
\end{table}



Crucially, \textbf{no projection contains the object}. Each provides constraints on the same underlying informational entity.

\section{The Role of $D$ as a Dimension}

$D$ is not a state variable within $\psi$ or $G$. It is a \textit{selection functional} that singles out a single realized history from a space of admissible histories $\Omega$:
\[
D : \Omega \to \omega_{\text{experienced}},
\]
where $\omega_{\text{experienced}}$ is the realized execution trace. This makes $D$ orthogonal to $\psi$ (superpositions) and $G$ (ensemble regularities).

\section{Probability Measure over Observer Histories}

Observers condition on their own existence. We define an observer-conditioned probability measure:
\[
\mathbb{P}(\gamma \mid O) = \frac{1}{Z_O} \exp\big(-\lambda\, \mathcal{C}_O[\gamma]\big), \qquad \gamma \in \Gamma_O.
\]

\section{Why Kolmogorov Complexity Is the Wrong Object}

Kolmogorov complexity is unsuitable because it is non-computable and machine-dependent. Physical laws must be expressed using representations that are computable and amenable to continuous optimization.

\section{$\psi$ as a Compression Codec}

\textbf{The complex-valued wavefunction is the compression scheme.} $\psi$ is a spectral encoding. We replace Kolmogorov complexity with a \textit{spectral coding cost}:
\[
\mathcal{C}_\psi[\gamma] \sim \text{description length of } \psi_\gamma \text{ in a fixed continuous basis}.
\]

\section{Geometry as a Required Decoding Layer}

A persistent observer must decode the substrate geometrically to maintain informational closure and predict boundary violations ($\partial O \cap \partial X = \emptyset$). Geometry is a survival-critical, error-tolerant compression codec.

\section{Why General Relativity Emerges}

General Relativity is the optimal large-scale geometric compression because it satisfies locality, continuity, compressibility, and predictive stability.

\section{Joint Complexity Minimization}

The realized world minimizes \textit{joint} representational cost:
\[
\mathcal{C}_{\text{total}}[\gamma] = \mathcal{C}_\psi[\gamma] + \mathcal{C}_G[\gamma].
\]
Among $\psi$-compatible geometries, GR-like structures minimize $\mathcal{C}_G$.

\section{Summary}

The observer's realized history is selected by:
\[
\boxed{\mathbb{P}(\gamma \mid O) = \frac{1}{Z_O} \exp\big(-\lambda \, \mathcal{C}_O[\gamma]\big)}
\]


Quantum mechanics and general relativity are unified not by adding structure, but by removing ontological commitments.
Each is the minimal encoding required to make an observer-compatible universe executable. GR and QM are incompatible only if both are treated as fundamental.
When treated as minimal, orthogonal encodings of information, they not only coexist — they uniquely select each other.

\begin{center}
\begin{tikzpicture}[scale=2]

% Axes
\draw[thick,->] (0,0,0) -- (2,0,0) node[anchor=north east] {$D$};
\draw[thick,->] (0,0,0) -- (0,2,0) node[anchor=south west] {$\psi$};
\draw[thick,->] (0,0,0) -- (0,0,2) node[anchor=south] {$G$};

% Optional: origin label
\node at (0,0,0) [anchor=north east] {$\mathcal{I}_n$};

% Optional: dotted projections
\draw[dashed] (2,0,0) -- (2,2,0) -- (0,2,0);
\draw[dashed] (2,0,0) -- (2,0,2) -- (0,0,2);
\draw[dashed] (0,2,0) -- (0,2,2) -- (0,0,2);

\end{tikzpicture}
\end{center}
