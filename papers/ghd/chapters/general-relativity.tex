\chapter{General Relativity}

\section{Curved Spacetime and Gravity}

Unlike the abstract Hilbert space of quantum mechanics, we experience a three-dimensional spatial world evolving in time. 
What we perceive as gravity is, in fact, the manifestation of spacetime curvature.

According to General Relativity, gravity is not a conventional force. 
Instead, matter and energy curve spacetime, and objects follow paths dictated by that curvature. 
In Einstein’s equation:

\begin{equation}
G_{\mu\nu} = \frac{8\pi G}{c^4} T_{\mu\nu},
\end{equation}

the distribution of mass–energy $T_{\mu\nu}$ determines the curvature $G_{\mu\nu}$, 
which in turn governs the motion of all objects.

Space and time are not absolute backgrounds. Each observer measures intervals 
according to their local trajectory through curved spacetime. 
The future is embedded in the four-dimensional geometry, though not all events are 
causally accessible to a given observer.

A thought experiment illustrates this. Imagine an observer in orbit around a black hole, 
and consider an object falling toward the event horizon. 
From the distant observer's perspective, the object slows as it approaches the horizon, 
never appearing to cross it. To the falling object, however, the rest of the universe 
appears to accelerate: billions of years may pass outside while only seconds elapse locally.

This difference is not a quirk. It is a structural feature of spacetime. 
Time is observer-dependent, and gravitational fields stretch or compress 
the rate at which events are experienced.

General Relativity replaces force with geometry. 
But this geometric reformulation raises a deeper question:

Why geometry at all?

\section{Why Curvature?}

Why does gravity manifest as curvature? Why does spacetime possess this structure?

One answer is efficiency.

General Relativity can be interpreted as an extreme data compression scheme. Take any amount of energy and matter, 
and the Einstein field equations map the enormously complex distribution of microscopic degrees of freedom 
into a single geometric object: the metric tensor $g_{\mu\nu}$.

Instead of tracking all individual particle interactions, the theory encodes their collective informational content 
into curvature.

Symbolically:

\[
\text{Information density} \longrightarrow \text{Curvature}.
\]

Many microstates correspond to the same macroscopic geometry. 
The metric thus acts as a minimal sufficient statistic of stress–energy.

\section{Singularities}

A fundamental limitation of General Relativity arises under extreme conditions. 
Spacetime can curve so intensely that curvature invariants diverge—a phenomenon known as a \textbf{singularity}. 
In classical GR, singularities signal a breakdown of the spacetime manifold: 
the equations predict them, yet fail to provide a valid description at those points.

Notable examples include the centers of black holes and the \emph{initial singularity} 
associated with the Big Bang. In these regions, classical notions of distance and time cease to apply, 
and the standard geometric description of spacetime loses predictive power.

This breakdown is not a mere curiosity; it reflects a structural limitation of the classical theory. 
Extreme curvature reveals where the classical metric cannot fully encode the underlying degrees of freedom.

From an informational perspective, singularities may indicate points where the geometric description becomes inefficient 
or incomplete. Many microstates may correspond to the same macroscopic geometry, 
but in singular regions even this coarse-grained encoding fails. 

To approach the ultimate nature of reality, we must understand the fundamental constituents 
that underlie classical spacetime and the informational substrate from which geometry emerges. 
Black hole and cosmological singularities serve as natural guideposts in this exploration.

