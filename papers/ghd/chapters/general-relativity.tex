\chapter{General Relativity}

\section{The Geometric Nature of Gravity}

General Relativity is extraordinarily successful at describing gravity.
Its predictions have been confirmed across vastly different scales,
from planetary motion to gravitational waves and black hole mergers.

Yet, like quantum mechanics, it demands a profound departure from classical intuition.

Gravity is not a force acting within space and time.
It is a statement about the structure of space and time themselves.

Matter does not move \emph{through} spacetime.
Matter moves \emph{as constrained by} spacetime geometry.

\subsection{Spacetime as a Geometric Object}

In General Relativity, the universe is modeled as a four-dimensional differentiable manifold
$\mathcal{M}$ equipped with a metric tensor $g_{\mu\nu}$.

The metric determines:
\begin{itemize}
\item Distances
\item Time intervals
\item Angles
\item Causal structure
\end{itemize}

The infinitesimal spacetime interval is given by:
\[
ds^2 = g_{\mu\nu} \, dx^\mu dx^\nu.
\]

This single object replaces:
\begin{itemize}
\item Newtonian gravitational potential
\item Absolute space
\item Absolute time
\end{itemize}

The metric is not a passive background.
It is dynamical.

Spacetime geometry responds to the distribution of matter and energy.

\subsection{Free Fall as Geometry}

Consider a freely falling particle.
In Newtonian mechanics, its acceleration is caused by a force.

In General Relativity, no force acts.

Instead, the particle follows a \emph{geodesic}:
\[
\frac{d^2 x^\mu}{d\tau^2}
+
\Gamma^\mu_{\nu\rho}
\frac{dx^\nu}{d\tau}
\frac{dx^\rho}{d\tau}
= 0,
\]
where $\Gamma^\mu_{\nu\rho}$ are the Christoffel symbols constructed from the metric.

This equation expresses inertial motion.
Gravity disappears locally.

What we perceive as gravitational attraction is the convergence of nearby geodesics.

\subsection{Curvature}

The curvature of spacetime is encoded in the Riemann curvature tensor:
\[
R^\mu_{\;\nu\rho\sigma}.
\]

It measures the failure of vectors to return unchanged after parallel transport around infinitesimal loops.

Contractions of the Riemann tensor yield:
\begin{itemize}
\item The Ricci tensor $R_{\mu\nu}$
\item The Ricci scalar $R$
\end{itemize}

These quantities summarize curvature relevant to volume distortion and geodesic convergence.

Curvature is not optional.
It is the fundamental dynamical degree of freedom of the theory.

\subsection{The Einstein Field Equations}

The dynamics of spacetime are governed by the Einstein field equations:
\[
G_{\mu\nu} = 8\pi G \, T_{\mu\nu},
\]
where:
\[
G_{\mu\nu} = R_{\mu\nu} - \frac{1}{2} g_{\mu\nu} R
\]
is the Einstein tensor, and $T_{\mu\nu}$ is the stress–energy tensor.

The stress–energy tensor encodes:
\begin{itemize}
\item Energy density
\item Momentum density
\item Pressure
\item Stress
\end{itemize}

These equations equate geometry with matter.

They are:
\begin{itemize}
\item Nonlinear
\item Local
\item Tensorial
\item Coordinate-independent
\end{itemize}

Spacetime tells matter how to move.
Matter tells spacetime how to curve.

\subsection{No Preferred Coordinates}

General Relativity is invariant under arbitrary smooth coordinate transformations.

There is no preferred notion of:
\begin{itemize}
\item Absolute rest
\item Absolute simultaneity
\item Global time slicing
\end{itemize}

Only geometric invariants have physical meaning.

Coordinates are bookkeeping devices, not physical entities.

This removes vast amounts of redundancy.
Many coordinate descriptions correspond to the same physical spacetime.

\subsection{Gravitational Time Dilation}

Time itself is geometry-dependent.

For a stationary observer:
\[
d\tau = \sqrt{-g_{00}} \, dt.
\]

Clocks at different gravitational potentials tick at different rates.

This is not a dynamical effect.
It is a direct consequence of metric structure.

Time is local.

\subsection{Horizons and Information Loss}

Strong curvature can produce horizons.

At an event horizon:
\begin{itemize}
\item Light cannot escape
\item Time coordinates exchange roles
\item External observers lose access to interior information
\end{itemize}

Horizons are not physical barriers.
They are geometric boundaries of causal accessibility.

Information loss here is observational, not necessarily fundamental.

\subsection{Singularities}

Under broad conditions, General Relativity predicts singularities.

At singularities:
\begin{itemize}
\item Curvature scalars diverge
\item Geodesics terminate
\item The manifold description breaks down
\end{itemize}

The theory does not fail mathematically.
It predicts its own domain of invalidity.

This signals that spacetime geometry is not fundamental at arbitrarily small scales.


\section{General Relativity as Constraint}

Unlike quantum mechanics, General Relativity is not a theory of states evolving in time.

It is a theory of \emph{consistent four-dimensional configurations}.

Given suitable boundary conditions, the Einstein equations constrain the allowed geometries.

Time evolution is not fundamental.
It is a slicing of a four-dimensional structure.

The equations are elliptic-hyperbolic constraints on geometry.

This is why:
\begin{itemize}
\item The initial value problem is subtle
\item Global solutions are rare
\item Exact solutions are highly symmetric
\end{itemize}

Spacetime is not computed step-by-step.
It exists as a self-consistent whole.

\section{Geometry as Compression}

Why does gravity take this geometric form?

Because geometry is an efficient representation.

Instead of tracking:
\begin{itemize}
\item All particle trajectories
\item All interactions
\item All forces
\end{itemize}

General Relativity encodes gravitational influence into:
\[
g_{\mu\nu}(x)
\]

This single object summarizes how all matter moves.

Curvature replaces bookkeeping.

Just as Hilbert space compresses exponentially many classical alternatives,
spacetime geometry compresses the gravitational influence of matter
into a smooth tensor field.

Different microscopic configurations can produce the same stress–energy tensor.
Different stress–energy tensors can admit multiple geometries.

The mapping is many-to-many.


