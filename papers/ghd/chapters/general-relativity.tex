\chapter{Why General Relativity?}

\section{Curved Spacetime and Gravity}

According to General Relativity, gravity is not a force in the traditional sense. 
Instead, matter and energy curve spacetime, and objects follow the paths dictated by that curvature. 
In Einstein’s equation:

\begin{equation}
G_{\mu\nu} = \frac{8\pi G}{c^4} T_{\mu\nu},
\end{equation}

the distribution of mass–energy $T_{\mu\nu}$ determines the curvature $G_{\mu\nu}$, 
which in turn governs the motion of all objects.

Space and time are not absolute backgrounds. Each observer measures intervals 
according to their local trajectory through curved spacetime. 
The future is embedded in the four-dimensional geometry, though not all events are 
accessible to a given observer.

A thought experiment makes this concrete. Imagine an observer in orbit around a black hole, 
and consider a hypothetical object falling toward the event horizon. 
From the distant observer's perspective, the object slows as it approaches the horizon, 
never seeming to cross it. To the falling object, however, the rest of the universe 
appears to accelerate: billions of years may pass outside while only seconds elapse locally.

This difference is not a quirk. It is a structural feature of spacetime. 
Time is observer-dependent, and gravitational fields stretch or compress 
the rate at which events are experienced.

General Relativity replaces force with geometry. 
But this geometric reformulation raises a deeper question:

Why geometry at all?

\section{Singularities}

Highly concentrated mass–energy can curve spacetime so intensely that curvature formally diverges—a singularity. 
In classical GR, such singularities signal breakdown of the spacetime manifold. 
The equations predict them, yet cease to describe them.

From a purely geometric standpoint, this appears pathological.

From an information-theoretic standpoint, it may not be.

Let us investigate black hole singularities by treating a computer simulation 
and the simulated black hole as two equivalent descriptions of the same informational structure.

The question is not whether GR predicts singularities—it does. 
The question is whether the informational description clarifies what they represent.

\section{Execution Trace}

In software engineering, an execution trace is the chronological record of all machine states during program execution. 
If source code is a map, the execution trace is the complete GPS log of every step taken.

Building on the equivalence between physical and informational configurations established earlier, 
we treat spacetime geometry and execution trace as two interpretations of the same underlying structure.

Where GR becomes numerically unstable near singularity formation, 
the execution trace remains well-defined. 
This provides complementary access to the collapse process.

Let $\mathcal{M}$ denote the set of all memory locations in a deterministic computing system. 
A \emph{machine state} $\mathcal{S} \in \mathcal{M}$ is a complete assignment of values to all memory elements. 
Let $\mathcal{P} = (I_1, I_2, \dots, I_n)$ be a finite sequence of deterministic instructions, 
with $I_k : \mathcal{S} \rightarrow \mathcal{S}$. 

The execution trace $\mathcal{T}$ is the ordered sequence:

\[
\mathcal{T} = (s_0, s_1, \dots, s_n), \quad s_{k+1} = I_{k+1}(s_k).
\]

The trace encodes the full informational evolution of the simulated geometry.

\subsection{Mapping Geometry to Bitstring}

Let a geometric configuration be encoded as a bitstring $b \in \{0,1\}^L$, 
representing discretized particle positions in a collapsing dust cloud. 
Define the configuration space:

\[
C = \{0,1\}^{3k}.
\]

Let

\[
f : C \to \mathbb{Z}^3,
\]

decode fixed-length binary segments into integer coordinates. 
Quantization ensures that entropy reflects genuine positional distinctions.

\subsection{Shannon Entropy}

The Shannon entropy of the bitstring is

\[
H(b) = -p_0 \log_2 p_0 - p_1 \log_2 p_1,
\]

where $p_0$ and $p_1$ are empirical bit frequencies.

Across the execution trace, entropy decreases monotonically as particles converge.

Although the geometric simulation crashes numerically at the classical singularity 
due to divergences, the entropy trajectory extrapolates smoothly:

\[
H(\mathcal{B}_t) \to 0.
\]

Zero entropy implies $W = 1$: only one distinguishable configuration remains.

Any faithful geometric decoding of a zero-entropy bitstring yields the same object: a point.

Thus the singularity is not an infinite curvature region, but a state of informational exhaustion. 
It is the simplest possible geometric configuration.

\section{General Relativity as Information Compression}

We now return to the broader question.

Why does gravity manifest as curvature? Why does spacetime possess this structure?

One answer is efficiency.

General Relativity acts as an extreme data compression scheme.

The Einstein field equations map the enormously complex distribution of 
microscopic degrees of freedom into a single geometric object: the metric tensor $g_{\mu\nu}$.

Instead of tracking all particle interactions, the theory encodes their collective 
informational content into curvature.

Symbolically:

\[
\text{Information density} \longrightarrow \text{Curvature}.
\]

Many microstates correspond to the same macroscopic geometry. 
The metric acts as a minimal sufficient statistic of stress–energy.

This is compression.

Quantum mechanics performs a different compression: it encodes 
all classical possibilities into a state vector in Hilbert space.

But Hilbert space is vast and nonlocal.

Spacetime geometry is compact and local.

Observers require locality.

\section{Why Three Spatial Dimensions?}

We now address the central question:

Why do we experience three spatial dimensions instead of 
an abstract high-dimensional Hilbert space?

An observer is a self-maintaining information structure. 
To persist, three conditions must hold:

\begin{enumerate}
\item Internal information must remain stable.
\item External interaction must be regulated.
\item Informational boundaries must exist.
\end{enumerate}

Boundaries require volume.

In one spatial dimension, closed boundaries cannot enclose anything.  
In two dimensions, enclosed regions exist but physical fields decay too slowly 
(for example, gravitational potential scales logarithmically rather than as $1/r$), 
making stable isolation fragile.

In three spatial dimensions, several crucial properties emerge:

\begin{itemize}
\item Inverse-square force laws ($F \propto 1/r^2$),
\item Stable bound orbits,
\item Finite-energy localized field configurations,
\item Closed surfaces that enclose volume.
\end{itemize}

A 2-sphere embedded in $\mathbb{R}^3$ encloses volume. 
That volume separates interior from exterior.

Without such separation, information would diffuse without constraint.

Dimensional analysis of gravity supports this. 
In $D$ spacetime dimensions, Newtonian potential scales as

\[
\Phi(r) \propto \frac{1}{r^{D-3}}.
\]

Stable planetary orbits exist only in $D = 4$ spacetime dimensions (3 spatial).  
For $D > 4$, forces decay too rapidly to produce stable closed orbits.  
For $D < 4$, gravitational dynamics lack propagating degrees of freedom.

Thus $3+1$ dimensions represent the minimal setting in which:

\begin{itemize}
\item Gravity propagates dynamically,
\item Bound systems are stable,
\item Information structures can remain isolated.
\end{itemize}

We do not experience Hilbert space directly because it lacks intrinsic 
geometric boundaries. It does not define interior versus exterior.

Spacetime does.

\section{Geometry and Survival}

Biological observers possess sharply defined boundaries. 
Even minor disruptions threaten survival. 
Pain itself functions as an informational protection system.

If we were diffuse wave-like structures without geometric containment, 
persistent intelligence would be impossible.

Well-defined boundaries prevent destructive mixing of relevant and irrelevant information.

Geometry is therefore not aesthetic—it is protective.

Spacetime geometry provides:

\begin{itemize}
\item Locality,
\item Causal order,
\item Boundary formation,
\item Controlled information exchange.
\end{itemize}

These are prerequisites for observers.

\section{Conclusions}

We may summarize:

\begin{itemize}
\item General Relativity encodes mass–energy into curvature. It is a geometric compression scheme.
\item Singularities correspond to zero informational entropy states.
\item Three spatial dimensions are selected by the requirements of stable boundaries and localized information.
\item Hilbert space may describe total informational possibility, but geometry provides the minimal compression compatible with observer survival.
\end{itemize}

Spacetime is not merely a stage on which physics unfolds.

It is the simplest geometric encoding of information that allows 
persistent informational structures to exist.

Gravity is geometry.

Geometry is compression.

And compression is what makes observers possible.
