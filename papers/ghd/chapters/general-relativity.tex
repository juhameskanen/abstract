\chapter{Why General Relativity?}

\section{Curved Spacetime and Gravity}

According to General Relativity gravity is not a force in the traditional sense. Instead, matter and energy curve spacetime, and objects follow the paths dictated by that curvature. In Einstein’s equation:

\begin{equation}
G_{\mu\nu} = \frac{8\pi G}{c^4} T_{\mu\nu},
\end{equation}

the distribution of mass-energy $T_{\mu\nu}$ determines the curvature $G_{\mu\nu}$, which in turn governs the motion of all objects. Highly concentrated mass-energy can curve spacetime so intensely that the curvature formally becomes infinite—a singularity.

Time and space are no longer absolute. Each observer measures intervals according to their local trajectory through curved spacetime. The future is already embedded in the four-dimensional geometry, though not all events are accessible to a given observer.


\section{Time Dilation and the Observer’s Perspective}

A thought experiment makes this concrete. Imagine a programmer in orbit around a black hole, and consider a hypothetical object falling toward the event horizon. From the distant programmer’s perspective, the object slows as it approaches the horizon, never seeming to cross it. To the falling object, however, the rest of the universe appears to speed by: billions of years can pass outside while only seconds elapse locally.  

This difference is not a quirk but a fundamental property of spacetime itself. Time is a **personal, observer-dependent dimension**, and gravitational fields stretch or compress the rate at which events are experienced.



\section{Why Does Spacetime Have This Structure?}

But why does the universe operate in this way? Why is spacetime curved, time relative, and gravity geometric rather than a simple force?  

There is some analogy from computer graphics. In a ray-tracing renderer:

\begin{itemize}
    \item Regions with complex geometry take longer to render, requiring multiple intersections and “is-inside” tests.  
    \item Empty regions render quickly, as no details need evaluation.  
\end{itemize}

Could curvature and time dilation in the universe emerge from similar mechanism? Regions of high mass-energy, where many states are concentrated, experience slower local “processing” (time dilation), empty regions can evolve more freely.  

From this viewpoint, GR is not only a description of motion and gravity—it is a **computational infrastructure for the universe**, ensuring that observers and informational structures survive in a consistent, persistent environment. However, where is the computer running the ray tracer?

\section{Sense of Geometric Space}

A conscious intelligent observer (without naming anyone in particular!) has a natural sense of 3D space. But why is it that we perceive ourselves living in a world of solid geometric objects?
Why do we lack a "wavelike self" defined in the vastness of Hilbert space?

Both are equally real—it just takes a good microscope and a double-slit experiment to prove it. So, why do we require these two vastly different descriptions just to define who we are?

