\chapter{The Nature of Black Hole Singularities}

\section{Introduction}

In classical general relativity, black hole singularities can take two distinct forms depending on their angular momentum. 
Non-rotating (Schwarzschild) black holes possess a point-like singularity at the center, whereas rotating (Kerr) black holes 
feature a ring-shaped singularity lying in the equatorial plane.

In this work, we investigate both singularities from an information-theoretic perspective. The guiding hypothesis, 
which we term the Methodological Epilogue, is that geometric spacetime and computational simulation can be viewed as 
two equivalent interpretations of a single underlying informational structure. 
Within this framework, entropy serves as the unifying quantity, allowing geometric and computational descriptions to be 
mapped onto one another.

The central question we pursue is whether this reformulation yields predictions beyond those of GR. Specifically, 
we ask whether approaching singularities as information-theoretic phenomena—rather than merely as geodesic 
endpoints—provides new insights into their structure, dynamics, or physical interpretation. To explore this, 
we develop both analytic reasoning and numerical simulations of collapsing dust clouds 
in Schwarzschild spacetime, interpreted through entropy and information flow.


\section{Methods}

We model black hole evolution through computational simulation. 
Instead of measuring thermodynamic entropy, we compute the Shannon entropy of the 
\emph{bitstrings} that encode the discretized particle positions and spacetime geometry in memory.  
This bitstring entropy directly quantifies the informational variability of the simulated spacetime image, 
and is the central observable linking computation and geometry.


\subsection*{Definition: Execution Trace}

\begin{quote}
\textbf{Execution Trace.}  
Let \( \mathcal{M} \) denote the set of all memory locations in a deterministic computing system, including registers and control flags. 
A \emph{machine state} \( s \in \mathcal{S} \) is a complete assignment of values to all elements of \( \mathcal{M} \).

Let \( P = (I_1, I_2, \dots, I_n) \) be a finite sequence of deterministic instructions, where each instruction induces a transition:
\[
I_k : \mathcal{S} \rightarrow \mathcal{S}
\]

Then the \emph{execution trace} \( T \) of program \( P \) is the ordered sequence of states:
\[
T = (s_0, s_1, \dots, s_n), \quad \text{where } s_{k+1} = I_{k+1}(s_k).
\]

The execution trace captures the total informational content of the computation.
Any simulated phenomenon is necessarily embedded in this trace. 
In practice, we restrict to the \emph{bitstring sub-trace} that encodes the discretized spacetime geometry.
\end{quote}



\subsection{Mapping Bitstrings to Geometry}

Let a geometric state be encoded as a bitstring $b \in \{0,1\}^L$. Each such bitstring represents 
the discretized positions of all particles in the simulated dust cloud.

Let:
\[
  \mathcal{C} = \{0,1\}^{3k}
\]
be the space of 3D geometric configurations. Define:
\[
  f : \mathcal{C} \to \mathbb{Z}^3, \quad f(b) = (\phi(b_1), \phi(b_2), \phi(b_3)),
\]
where $\phi : \{0,1\}^k \to \mathbb{Z}$ decodes fixed-length binary segments into integers. 
This discretization introduces finite resolution but suffices to define entropy across state sequences. 


\section{Entropy Collapse and Geometric Singularity}

We define the Shannon entropy of a bitstring $b \in \{0,1\}^L$ as:
\[
  H(b) = -p_0 \log_2 p_0 - p_1 \log_2 p_1,
\]
where $p_0$ and $p_1$ are the empirical frequencies of $0$- and $1$-bits in $b$.

Over a sequence of states (the bitstring sub-trace), the entropy is computed for each frame, yielding 
an \emph{entropy trajectory} that reflects the collapse of information variability as particles fall toward the singularity. 


\section{Simulations}

Two different metrics were studied: Schwarzchild and Kerr. Particle trajectories were tracked 
through the event horizon as far as the numerically reasonable, and then frozen.

\begin{figure}[h!]
  \centering
  \includegraphics[width=0.8\textwidth]{figures/schwarzschild.png}
  \caption{Entropy signature of bitstrings derived from trajectories integrated with Painlevé–Gullstrand coordinates.}
  \label{fig:schwarzschild_metric}
\end{figure}

\begin{figure}[h!]
  \centering
  \includegraphics[width=0.8\textwidth]{figures/kerr.png}
  \caption{Entropy signature of bitstrings derived from trajectories integrated with Kerr metric.}
  \label{fig:ker_metric}
\end{figure}


All the simulations show a monotonic decrease in bitwise entropy as particles fall inward. 
The extrapolated entropy approaches zero as the particles approach the singularity.


We define the following lemma:

\section{Lemma: Entropy-Singularity}

\begin{quote}
  \textbf{Entropy--Singularity Lemma (Fundamental Statement):} \\
  Let $b_t \in \{0,1\}^L$ be the bitstring representation of the simulated geometry at time $t$. 
  If $H(b_t) \to 0$, then the decoded geometry $f(b_t)$ degenerates to a singleton set. That is:
  \[
    H(b_t) \to 0 \quad \Rightarrow \quad \exists p \in \mathbb{Z}^3 \text{ such that } f(b_t) = p \ \ \forall t.
  \]
\end{quote}



\subsection*{Proof of Entropy-Singularity Lemma}

\begin{proof}
If $H(b_t)=0$, then the bitstring consists entirely of repeated bits. 
This corresponds to a single repeated encoded state $b^\ast$. 
For any mapping $f:\{0,1\}^L \to \mathbb{Z}^3$, we have 
$f(b_t)=(f(b^\ast),f(b^\ast),\dots)$, so the image set is $\{f(b^\ast)\}$. 
Thus the geometric representation degenerates to a single point, regardless of the choice of $f$.
\end{proof}


This result is representation-invariant: any dataset of zero entropy corresponds to a single repeated state, 
and any geometric mapping of that state must yield a single point. 
Thus, zero informational variability implies geometric collapse, independently of coordinate system, 
numerical encoding, or representation scheme.

section*{Conclusion}

In general relativity, a singularity is traditionally interpreted as a boundary of the spacetime manifold, where 
curvature invariants diverge and geodesics terminate incompletely. By applying information-theoretic methods, 
we have shown that singularities can be reinterpreted as the geometric manifestation of an information state of zero entropy.

At zero entropy, no geometric structures can emerge, and consequently, no particle-like degrees of freedom exist. 
Following from this, singularities do not correspond to points of infinite density; rather, they are points where gravity 
itself vanishes, as no information exists to encode spacetime or particles.

Thus, rather than representing points of infinite density, singularities can be understood as regions of zero density. 


\section{Discussion}

We have considered both static (Schwarzschild) and rotating (Kerr) black holes. 
The Entropy–Singularity Lemma is expected to hold in general, dynamic settings, 
since it depends only on the information-theoretic property of vanishing entropy, 
independent of the details of spacetime evolution.

Our simulations confirm the Entropy–Singularity Lemma in a computational context. 
Although the global thermodynamic entropy of the black hole increases (consistent with the second law), 
the local bitstring entropy encoding spatial geometry decreases toward the singularity. 
This reflects a narrowing of accessible informational states and growing determinism in geometry.

This distinction does not contradict thermodynamics. Rather, it separates global physical entropy 
from local representational entropy used in encoding spacetime. 
Our findings suggest that entropy provides a fundamental informational scale, 
with zero entropy corresponding to a collapse of space.

From this perspective, gravitational singularities can be interpreted not as edges of spacetime, 
but as well-defined geometric objects: point-like singularities in non-rotating black holes and 
ring-shaped singularities in rotating black holes.


\section{Bitstring Encoding}


To serialize a geometric state into a bitstring $b \in \{0,1\}^L$:
\begin{enumerate}
  \item Quantize continuous coordinates: $\mathbb{R} \to \mathbb{Z}$
  \item Convert integers to binary strings of fixed width $k$
  \item Concatenate binary segments into a full state bitstring
\end{enumerate}

To compute entropy:
\[
  H(b) = -p_0 \log_2 p_0 - p_1 \log_2 p_1,
\]
where $p_0$ and $p_1$ are empirical frequencies of bits in $b$.

