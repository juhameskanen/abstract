\documentclass[11pt]{article}
\usepackage{enumitem}
\usepackage{xcolor}
\usepackage{hyperref}
\usepackage{tcolorbox}
\usepackage{lipsum} % for placeholder text
% -- Common definitions for all papers

% --- Packages ---
\usepackage[utf8]{inputenc}
\usepackage[T1]{fontenc}
\usepackage{amsmath, amssymb, amsthm}
\usepackage{geometry}
\usepackage{hyperref}
\usepackage{cite}
\newcommand{\pdflink}[1]{} % By default, do nothing

% --- Page Setup ---
\geometry{letterpaper, margin=1in}
\setlength{\parindent}{0pt}
\setlength{\parskip}{1em}


% --- Author ---
\author{Abstract Researcher}

% Glossary
\newcommand{\SSP}{Spectral Selection Principle (SSP)}
\newcommand{\Lcost}{$\mathcal{L}$} % Added $ $ here
\newcommand{\Tobs}{$\mathcal{T}_{\mathrm{obs}}$} % Added $ $ here

\newcommand{\PrintGlossary}{
    \section*{Glossary of the Abstract Universe}
    \begin{description}
    \item[SSP] \SSP: The foundational principle asserting that reality is the subset of informational paths that admit the most efficient representation in the frequency domain. It replaces "laws of physics" with a statistical preference for compressibility.
    
    \item[\Tobs] Observer Filter

    \item[$\mathcal{L}$ (Spectral Encoding Length)] The measure of information required to represent a state or path $\gamma$. In this framework, $\mathcal{L}$ replaces the classical concept of \textit{Action}. Minimizing $\mathcal{L}$ is equivalent to the Principle of Least Action.
    
    \item[$\mathcal{T}_{\mathrm{obs}}$ (Observer Filter)] The subset of all possible configuration sequences capable of describing a subjective experience.
    
    \item[MSM (Microstructure Motif)] Recurring, compressible bit-patterns within configuration space (e.g., particles, fields). The density of these motifs determines the local spectral cost, manifesting as mass-energy.
    
    \item[Nyquist Horizon] The "Planck Scale" of the theory. It is the maximum frequency supportable by the discretization of configuration space. Any structure below this limit is mathematically "aliased" and cannot be rendered.
    
    \item[Induced Time] The sequential ordering of states along a path $\gamma$. An emergent property of the observer's trajectory.
    
    \item[Phase-Coherence] The informational alignment between disparate parts of the spectral encoding. This provides the mathematical basis for what is traditionally called \textit{quantum entanglement}.

    \end{description}
}



\usepackage[utf8]{inputenc}
\usepackage{tcolorbox}
\usepackage{hyperref}
\usepackage{enumitem}
\usepackage[backend=biber]{biblatex} % your bibliography
\addbibresource{../../references.bib}

%----------------------------------------
% FAQ box styles
%----------------------------------------
\tcbset{
  faqquestion/.style={
    colback=gray!15, colframe=gray!50,
    fonttitle=\bfseries, boxrule=0.5pt, arc=4pt
  },
  faqanswer/.style={
    colback=blue!5, colframe=blue!50,
    fonttitle=\bfseries, boxrule=0.5pt, arc=4pt
  }
}

% Answer box command
\newcommand{\answer}[1]{%
  \begin{tcolorbox}[title=A:]#1\end{tcolorbox}%
}

% FAQ environment using real sections for TOC
\newenvironment{faq}[2][]{%
  \refstepcounter{section}% use section counter for TOC
  \section*{Q\thesection: #2}%
  \addcontentsline{toc}{section}{Q\thesection: #2}%
  \ifx&#1&\else\textit{#1}\par\medskip\fi
  \begin{tcolorbox}[faqquestion]%
}{%
  \end{tcolorbox}%
}

\addbibresource{../../references.bib} 

\title{Frequently Asked Questions}
\date{2001 -- 2025}

\begin{document}
\maketitle

% Automatic PDF TOC at top
\tableofcontents
\bigskip

\begin{faq}{Where did all the matter in the universe come from?}
\answer{
This question implicitly assumes a temporal ordering and an external origin, but both time and causality are emergent properties of observers within the universe (as established in Paper~1). In the underlying informational substrate, the universe is abstract and timeless. It exists because there are no constraints forbidding it. 
}
\end{faq}


\begin{faq}{Why something rather than nothing?}
Asking why there is something rather than nothing is like asking why “heads” rather than “tails” — both possibilities exist, but only one can be observed. The observer necessarily finds itself in the branch that exists.
\end{faq}

\begin{faq}{Why do particles follow a complex-valued wavefunction?}
  The wavefunction is not a physical object in the classical sense; it is the optimal \textbf{compression algorithm}
  for the underlying informational structure of reality. We perceive the universe as "waving" because we are observing compressed information.

\noindent This process follows a deterministic logical funnel:
\begin{center}
    $\boxed{\text{Maximal Compression}} \rightarrow \boxed{\text{Maximal Probability}} \rightarrow \boxed{\text{Maximal Predictability}}$
\end{center}


According to the Gibbs measure $P(\gamma | O) \propto \exp(-\lambda C_O[\gamma])$, configurations with the lowest complexity ($C_O$) are the most
statistically likely to exist. The "laws" of physics are simply the most probable survivors of this informational selection.
\end{faq}    


\begin{faq}{Why does the universe expand? Is there a link to entropy?}
\answer{
The universe expands because its entropy increases; space and entropy are two sides of the same coin. 
The trajectory of the universal informational configuration toward higher entropy is formally analogous to the Inflationary Epoch followed by gradual expansion. 
Space is the \textbf{geometric projection} of total entropy: as the underlying information becomes more disordered, the corresponding geometry stretches, producing the observed expansion of the universe.}
\end{faq}


\begin{faq}{Why did the universe start with zero entropy?}
\answer{
The observer-conditioned measure overwhelmingly favors **histories that minimize total complexity**:

\[
\mathbb{P}(\gamma \mid O) \propto \exp[-\lambda \, \mathcal{C}_O[\gamma]].
\]

A zero-entropy initial state is **maximally compressible** and hence most probable. 

\[
0 \;\rightarrow\; 1 \;\rightarrow\; \dots \;\rightarrow\; \text{current state}.
\]

Starting from high entropy or decreasing entropy requires extra information to embed the observer, increasing \(\mathcal{C}_O[\gamma]\) and suppressing probability:

\[
\mathbb{P}(\gamma_{\rm high-entropy} \mid O) \ll \mathbb{P}(\gamma_{\rm zero-entropy} \mid O).
\]

Thus, the **arrow of time** and increasing total entropy naturally emerge:  

\[
\text{Zero entropy} \;\Rightarrow\; \text{Maximal compressibility} \;\Rightarrow\; \text{Observer-compatible universe}.
\]
}
\end{faq}



\begin{faq}{If there is no metaphysical signals or interaction, then how does the observer ``see'' anything?}
  \answer{The universe we “see” is not transmitted to us—it is already part of the set that define us.

The observer’s wavefunction can be split as:

\[
\Psi_\gamma = \Psi_{\rm self} \otimes \Psi_{\rm env},
\]

where \(\Psi_{\rm self}\) is the geometrically persistent part defining the observer, and \(\Psi_{\rm env}\) is the remaining “outside” world.

Temporal extrapolation of \(\Psi_{\rm self}\) has a **limited predictive horizon**:

\[
\Psi_{\rm self}(t+1) \sim \arg\min_{\gamma_{t+1}} \mathcal{C}*O[\gamma*{0:t+1}],
\]

so the observer cannot see far into the future. Spatial extrapolation generates the compressible environment:

\[
\Psi_{\rm env} \sim \text{extensions of } \Psi_{\rm self} \text{ into surrounding bits}.
\]

The perceived universe emerges because **low-complexity histories dominate the measure**:

\[
\mathbb{P}(\gamma \mid O) \propto \exp[-\lambda , \mathcal{C}_O[\gamma]].
\]

In short: the observer does not receive signals from outside; the “world” is a compressible extrapolation of the self-wavefunction, and apparent laws and structures emerge statistically.
}
\end{faq}

  
\begin{faq}{Does \IaMe support QBism?}{Does the theory support the QBism view that our observations shape quantum phenomena?}
\answer{Yes. In \IaMe, the observer is an informational structure and is most likely to find themselves in configurations that preserve their reasoning and decision-making capabilities. Observers with reasoning have a higher probability of persistence than those without. As a result, reasoning statistically biases which configurations an observer experiences next. Quantum mechanics emerges from these informational rules, so it may appear that observation or decisions “change” the quantum state, even though all evolution is fully contained within the underlying substrate.}
\end{faq}


\begin{faq}{How does Gödel incompleteness affect the theory?}
\answer{Considering how highly respected authors, such as Stephen Hawking, have noted that a Theory of Everything may be forever constrained by Gödelian limits, we approach this with necessary humility. Gödel’s incompleteness theorems constrain formal systems capable of arithmetic, but our framework suggests a shift in focus: a ToE aims for ontological closure rather than formal deductive completeness. While an \textbf{embedded observer} will inevitably encounter Gödel-style limits in their internal formal reasoning, these might be constraints on the \textit{compression and representation} of the data, not necessarily a limitation on the underlying informational substrate itself.}
\end{faq}

\begin{faq}{Is a singularity a point of infinite density?}
  \answer{No. If singularities are taken as points of infinite curvature, their informational correspondence would compress poorly and thus be suppressed in statistical measure. 
  A geometric point maps to a \textbf{zero-entropy state}, which possesses no degrees of freedom to encode microstructure (such as particles). 
  Therefore, in neither case does the theory imply that "singularities" are points of infinite curvature.}
\end{faq}

\begin{faq}{Are we living in a simulation?}
\answer{
  In the popular formulation, the simulation hypothesis posits an external simulator operating in a higher-level physical reality. Recent work has argued that such hypotheses are not fully consistent with observed features of our universe, placing strong constraints on externally run simulations \cite{faizal2025}. \cite{Vazza2025SimulationConstraints} approaches the simulation hypothesis from a very different perspective, with astrophysical constraints. Regardless of the technical details, this approach merely shifts the explanatory burden: one must then explain the origin and laws of the simulator itself. In this framework, such meta-theories are rejected. A genuine Theory of Everything cannot rely on an external system, as a ``theory of everything minus the simulator'' is not a theory of everything. Instead, we assume informational ontological equivalence: simulated and non-simulated realities are not ontologically distinct. The universe is not running \emph{on} something else; rather, all structures, including observers and apparent physical laws, arise within a single abstract informational substrate. In this sense, the question of whether we are ``in a simulation'' is ill-posed.}
\end{faq}


\begin{faq}{Is the recursive simulation hypothesis supported within the theory?}
\answer{
Yes, but only in a restricted and precisely defined sense.

Within the theory, observers are informational structures. Any such structure capable of reasoning can, in principle, construct formal systems, Turing machines, and simulations of itself. This has already occurred: humans have defined universal computation and have begun simulating simplified versions of biological, cognitive, and physical processes. Since biological information (including DNA) can be encoded digitally without loss of information, simulated humans can be axiomatic copies of real humans at the informational level.

These sub-simulations do not require a separate ontological substrate. They correspond simply to different configurations within the same underlying informational space (e.g., subsets of $2^n$). In this limited sense, recursive simulation is supported.

However, the theory places a fundamental constraint on unbounded recursive simulation: \textbf{entropy}.

While it is logically possible to define a simulation of observers, it is informationally expensive to \emph{realize} one. Extracting, encoding, and maintaining the information required to faithfully describe even a single human genome is already nontrivial. Scaling this to billions of humans, each composed of billions of cells, embedded in a universe containing astronomical numbers of particles and interactions, rapidly exhausts available informational resources.

As a result, recursive simulations necessarily degrade. They must either, omit detail, compress aggressively (thereby losing fidelity), restrict scope, or terminate.

While recursive simulation is internally consistent with the theory, its informational cost is high. Consequently, deep, high-fidelity nested simulations are unlikely to persist. The theory favors shallow or short-lived recursions, yet this contradicts our current empirical evidence.}
\end{faq}


\begin{faq}{Why do we die?}
\answer{
Death can be understood in terms of the spectral complexity of an observer's wavefunction.
Let $\psi_O$ denote the linear encoding of all correlations defining observer $O$.
The observer's probability of continued existence along a future history $\gamma \in \Gamma_O$ is given by the Spectral Selection Principle:
\[
\mathbb{P}(\gamma \mid O) \propto \exp\big(-\alpha \,\Sigma[\gamma]\big),
\]
where $\Sigma[\gamma]$ is the number of independent frequency--phase components required to encode $\gamma$ at the observer's resolution, and $\alpha$ is a scale parameter determined by observer bandwidth.

As the observer accumulates information—memories, learned skills, and internal structure—the effective spectral complexity $\Sigma[\gamma]$ of future continuations necessarily increases. Each additional independent component in $\psi_O$ reduces the exponential weight of compatible histories. Formally, if the observer requires $N$ additional frequencies to represent future states, the probability of survival along those histories scales as
\[
\mathbb{P}_\text{future} \sim \exp(-\alpha N),
\]
leading to a rapid, exponential suppression of persistence.

In this framework, death is the inevitable consequence of combinatorial exhaustion: as the observer's informational content grows, the set of future histories that preserve identity becomes vanishingly small. Survival probability approaches zero, corresponding to the cessation of the observer's experiential existence. Hence, death emerges naturally from the finite and exponentially constrained structure of complex informational systems.
}
\end{faq}


\begin{faq}{Why do we age? Why do we develop wrinkles?}
\answer{
Wrinkles and aging are the geometric manifestation of increasing spectral complexity in the observer wavefunction. 
As \(\psi_O\) accumulates more independent frequencies over time, the number of compatible configurations in \(\Gamma_O\) decreases exponentially. 
The corresponding geometric view (\(G\)) reflects this scarcity: imperfect skin, structural deterioration, and reduced regenerative capacity are simply the observer’s “finding itself” in suboptimal configurations. 
Aging is therefore the visible trace of spectral overcomplexification and the inevitable vanishing of survival probability.}
\end{faq}

\begin{faq}{Does the theory explain why the expansion rate of the universe is so close to the dividing line}
\answer{Conditional on entropy increasing and on typical emergence, observers are overwhelmingly likely to find themselves at the entropy value where microstructure count is maximal — which corresponds to a near-critical expansion regime.}
\end{faq}


\begin{faq}{Is this just Solipsism? Am I the only one who exists?}
\answer{The theory adopts Cartesian Certainty—the observer’s existence—as its sole a priori datum. However, it does not default to Solipsism; instead, the number of observers is treated as a probabilistic variable.

Statistically, it is difficult to justify a "singular" outcome in a system of high complexity. Unless it can be mathematically demonstrated that the probability of many is lower than probabilty of one. The theory follows the Highest Probability Rule: "Others" exist if they are a natural, high-likelihood consequence of a reality emerging without predefined constraints.}
\end{faq}

\begin{faq}{Does the claim that compression implies maximal probability depend on Kolmogorov complexity?}
\answer{
No. We explicitly reject Kolmogorov complexity as a foundational quantity. Kolmogorov complexity is neither continuous nor computable and therefore cannot serve as a physical observable.

In the $IaM^e$ framework, description length is instead defined through the spectral content of the wavefunction. A physical state is characterized by a well-defined set of frequencies and phases, and its informational cost is given by the minimal spectral length required to represent it under finite resolution. This quantity is continuous, measurable, and can be mapped to a discrete bit representation.

The apparent wave-like behavior of the microphysical world is not postulated but selected: among all admissible informational configurations, those admitting maximal spectral compression dominate the measure. Observers are therefore overwhelmingly likely to be embedded in universes whose dynamics admit compact spectral representations. In this sense, we do not observe a compressed universe because it is assumed, but because compressed configurations are overwhelmingly more probable within the space of physically realizable descriptions.}
\end{faq}

\printbibliography

\end{document}
