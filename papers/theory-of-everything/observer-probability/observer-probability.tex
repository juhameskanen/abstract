\documentclass[11pt]{article}
% -- Common definitions for all papers

% --- Packages ---
\usepackage[utf8]{inputenc}
\usepackage[T1]{fontenc}
\usepackage{amsmath, amssymb, amsthm}
\usepackage{geometry}
\usepackage{hyperref}
\usepackage{cite}
\newcommand{\pdflink}[1]{} % By default, do nothing

% --- Page Setup ---
\geometry{letterpaper, margin=1in}
\setlength{\parindent}{0pt}
\setlength{\parskip}{1em}


% --- Author ---
\author{Abstract Researcher}

% Glossary
\newcommand{\SSP}{Spectral Selection Principle (SSP)}
\newcommand{\Lcost}{$\mathcal{L}$} % Added $ $ here
\newcommand{\Tobs}{$\mathcal{T}_{\mathrm{obs}}$} % Added $ $ here

\newcommand{\PrintGlossary}{
    \section*{Glossary of the Abstract Universe}
    \begin{description}
    \item[SSP] \SSP: The foundational principle asserting that reality is the subset of informational paths that admit the most efficient representation in the frequency domain. It replaces "laws of physics" with a statistical preference for compressibility.
    
    \item[\Tobs] Observer Filter

    \item[$\mathcal{L}$ (Spectral Encoding Length)] The measure of information required to represent a state or path $\gamma$. In this framework, $\mathcal{L}$ replaces the classical concept of \textit{Action}. Minimizing $\mathcal{L}$ is equivalent to the Principle of Least Action.
    
    \item[$\mathcal{T}_{\mathrm{obs}}$ (Observer Filter)] The subset of all possible configuration sequences capable of describing a subjective experience.
    
    \item[MSM (Microstructure Motif)] Recurring, compressible bit-patterns within configuration space (e.g., particles, fields). The density of these motifs determines the local spectral cost, manifesting as mass-energy.
    
    \item[Nyquist Horizon] The "Planck Scale" of the theory. It is the maximum frequency supportable by the discretization of configuration space. Any structure below this limit is mathematically "aliased" and cannot be rendered.
    
    \item[Induced Time] The sequential ordering of states along a path $\gamma$. An emergent property of the observer's trajectory.
    
    \item[Phase-Coherence] The informational alignment between disparate parts of the spectral encoding. This provides the mathematical basis for what is traditionally called \textit{quantum entanglement}.

    \end{description}
}


\addbibresource{../../references.bib} 

\title{Observer Probability}

\date{2023}

\begin{document}
\maketitle

\section{Motivation}

We show that the dominance of compressed observer descriptions follows from a purely combinatorial argument.

\subsection{Setup}

Let the total information budget of the universe be fixed at $N$ bits. Consider an observer $O$ whose identity can be fully specified by a minimal description of $k$ bits, with $k \leq N$. The remaining $N-k$ bits correspond to degrees of freedom that do not affect the observer's identity.

We assume a uniform measure over the space of all $2^N$ possible bitstrings of length $N$.

\subsection{Counting Observer Realizations}

An observer $O$ is realized whenever its $k$ defining bits are embedded into an $N$-bit string, regardless of the values of the remaining $N-k$ bits. The total number of distinct realizations of $O$ is therefore

\begin{equation}
\Omega(O \mid N, k)
= \binom{N}{k} \, 2^{\,N-k},
\end{equation}

where:
\begin{itemize}
    \item $\binom{N}{k}$ counts the number of ways to choose the locations of the $k$ defining bits,
    \item $2^{N-k}$ counts the number of possible assignments of the remaining unconstrained bits.
\end{itemize}

\subsection{Probability Measure}

Given the uniform measure over all $2^N$ bitstrings, the probability that a randomly selected $N$-bit configuration instantiates observer $O$ is

\begin{equation}
P(O \mid N, k)
= \frac{\Omega(O \mid N, k)}{2^N}
= \binom{N}{k} \, 2^{-k}.
\end{equation}

This expression is exact and involves no uncomputable quantities.

\subsection{Compression Implies Maximal Probability}

For fixed $N$, consider two observers $O_1$ and $O_2$ with minimal description lengths $k_1 < k_2$. The ratio of their probabilities is

\begin{equation}
\frac{P(O_1)}{P(O_2)}
= \frac{\binom{N}{k_1} \, 2^{-k_1}}{\binom{N}{k_2} \, 2^{-k_2}}.
\end{equation}

In the regime $k \ll N$, we may use the approximation

\begin{equation}
\binom{N}{k} \approx \frac{N^k}{k!},
\end{equation}

yielding

\begin{equation}
P(O \mid N, k)
\approx \frac{N^k}{k!} \, 2^{-k}.
\end{equation}

This probability decreases rapidly with increasing $k$. Consequently, observers admitting shorter descriptions occupy an exponentially larger fraction of the configuration space.


\begin{figure}[h!]
\centering
\includegraphics[width=0.75\textwidth]{figures/compression_max_probability.png}
\caption{Scatter Plot (MSL vs. Observer Probability):
Observer probability as a function of minimal spectral length (MSL) for 5,000 random bitstrings. Lower MSL corresponds to more compressible configurations, which are exponentially more probable.}
\end{figure}

\begin{figure}[h!]
\centering
\includegraphics[width=0.75\textwidth]{figures/msl.png}
\caption{Smoothed Trend (Binned Average Probability):
Smoothed trend of observer probability across MSL bins, showing the inverse relationship between spectral complexity and combinatorial likelihood.}
\end{figure}

\begin{figure}[h!]
\centering
\includegraphics[width=0.75\textwidth]{figures/distribution.png}
\caption{Histogram of Probabilities:
Distribution of observer probabilities P(O) across all sampled bitstrings, illustrating that highly compressible (low MSL) configurations dominate the probability landscape.}
\end{figure}



\subsection{Interpretation}

This result establishes that compression implies maximal probability as a direct consequence of combinatorics. No appeal is made to Kolmogorov complexity, Solomonoff induction, or heuristic notions of simplicity.
The minimal description length $k$ may be continuous and physically defined (e.g., via spectral support of a wavefunction) and only mapped to discrete bits after coarse-graining.

In the $IaM^e$ framework, $k$ corresponds to the Minimal Spectral Length required to encode the observer. The wavefunction provides the analytic representation achieving this minimal description,
and probabilistic weighting arises from the number of compatible microscopic realizations.

\subsection{Conclusion}

For a fixed information budget, observer configurations with minimal description length dominate the measure by necessity.
The emergence of compressed, wave-like physical laws therefore follows from exact counting, not heuristic principles or uncomputable complexity measures.


\subsection{Proof of Concept}

\begin{itemize}
\item \href{simulation/observer_probability.py}{observer\_probability.py: Python program serving as proof-of-concept}
\end{itemize}

\printbibliography

\end{document}
