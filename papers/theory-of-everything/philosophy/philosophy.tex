\documentclass[11pt]{article}
% -- Common definitions for all papers

% --- Packages ---
\usepackage[utf8]{inputenc}
\usepackage[T1]{fontenc}
\usepackage{amsmath, amssymb, amsthm}
\usepackage{geometry}
\usepackage{hyperref}
\usepackage{cite}
\newcommand{\pdflink}[1]{} % By default, do nothing

% --- Page Setup ---
\geometry{letterpaper, margin=1in}
\setlength{\parindent}{0pt}
\setlength{\parskip}{1em}


% --- Author ---
\author{Abstract Researcher}

% Glossary
\newcommand{\SSP}{Spectral Selection Principle (SSP)}
\newcommand{\Lcost}{$\mathcal{L}$} % Added $ $ here
\newcommand{\Tobs}{$\mathcal{T}_{\mathrm{obs}}$} % Added $ $ here

\newcommand{\PrintGlossary}{
    \section*{Glossary of the Abstract Universe}
    \begin{description}
    \item[SSP] \SSP: The foundational principle asserting that reality is the subset of informational paths that admit the most efficient representation in the frequency domain. It replaces "laws of physics" with a statistical preference for compressibility.
    
    \item[\Tobs] Observer Filter

    \item[$\mathcal{L}$ (Spectral Encoding Length)] The measure of information required to represent a state or path $\gamma$. In this framework, $\mathcal{L}$ replaces the classical concept of \textit{Action}. Minimizing $\mathcal{L}$ is equivalent to the Principle of Least Action.
    
    \item[$\mathcal{T}_{\mathrm{obs}}$ (Observer Filter)] The subset of all possible configuration sequences capable of describing a subjective experience.
    
    \item[MSM (Microstructure Motif)] Recurring, compressible bit-patterns within configuration space (e.g., particles, fields). The density of these motifs determines the local spectral cost, manifesting as mass-energy.
    
    \item[Nyquist Horizon] The "Planck Scale" of the theory. It is the maximum frequency supportable by the discretization of configuration space. Any structure below this limit is mathematically "aliased" and cannot be rendered.
    
    \item[Induced Time] The sequential ordering of states along a path $\gamma$. An emergent property of the observer's trajectory.
    
    \item[Phase-Coherence] The informational alignment between disparate parts of the spectral encoding. This provides the mathematical basis for what is traditionally called \textit{quantum entanglement}.

    \end{description}
}


\addbibresource{../../references.bib} 

\title{Philosophy}
\date{2024}

\begin{document}
\maketitle

\section*{Philosophical Grounding of a Theory of Everything}

Physical theories must ultimately be tested against observation; without falsifiable consequences, a framework belongs more to philosophy than physics.

Many mainstream physical theories treat the observer as external or auxiliary rather than as a central element of the formalism.
Yet observation and experience are the only way the universe is ever known.

If a theory \(T\) requires an external structure \(S\) to fix its constants, boundary conditions, or domain of validity, then \(S\) constitutes information not contained within \(T\).
Likewise, if a theory fails to account for the existence of conscious observers \(O\), it omits a central empirical fact: that observations occur at all.

\[
T + S \neq T
\]
\[
T - O \neq T
\]

A genuine Theory of Everything must therefore be self-contained. It must not only describe observed regularities, but also explain why observation itself is possible.


\section{The Nature of Mathematics}

We take as a starting point that reality emerges from an abstract, unconstrained informational substrate—an infinite potential encompassing all possibilities.
Within this substrate, information, mathematics, and ultimately the laws of physics arise as emergent structures: they exist because there are no predefined constraints forbidding them.

However, reasoning about mathematics using mathematics encounters inherent limitations, as demonstrated by Gödel's incompleteness theorems.
No formal system sufficiently expressive to encode arithmetic can be both complete and self-verifying. At some level, any argument about the origin or necessity of mathematics cannot itself be fully formalized within mathematics.

For the purposes of a self-contained Theory of Everything, we adopt the minimal operational assumption that all entities, structures, laws, and observers arise as emergent features of the underlying informational substrate. No additional axioms or external parameters are required. This assumption allows mathematics and logic to be treated as emergent tools rather than pre-given absolutes, while acknowledging that ultimate meta-mathematical questions cannot be resolved purely within formal systems.


\section{Requirements for a Fully Self-Contained Theory of Everything}

We expect a genuine Theory of Everything (ToE) not to be limited to a catalog of dynamical laws governing pre-assumed entities. Such a theory would merely shift the explanatory burden to its primitives and leave unanswered the most fundamental questions concerning existence, structure, and observation.
A fully self-contained ToE must therefore satisfy a stronger criterion. It must account for the total explanatory closure of reality without appeal to external meta-theories.

At minimum, such a theory is expected to describe:

\begin{enumerate}
\item \textbf{What exists.}  
  The theory must specify its fundamental ontological substrate. These primitives must not be defined implicitly by higher-level structures (such as spacetime, particles, or fields), but instead give rise to them. Any reliance on externally interpreted mathematical objects undermines self-containment.
  

\item \textbf{Why structure exists.}  
  Beyond enumerating entities, the theory must explain why the universe exhibits regularities in form of laws of physics, rather than being maximally disordered or trivial.
  This includes explaining why stable laws, symmetries, and persistent patterns arise at all.

\item \textbf{Why the observed structure has the form it does.}  
  A ToE must justify the emergence of specific structural features—such as dimensionality, locality, causality, and temporal ordering—rather than assuming them. 
  
\item \textbf{The Observer Exists}  
  Observation cannot be treated as an external process. Observers must be describable as structures internal to the theory, arising from the same principles as all other phenomena. 
  
\item \textbf{Why observation is limited.}  
  Equally important, the theory must account for the fact that observers have bounded access to truth, predictive power, and global structure. These limitations must not be imposed ad hoc, but emerge from the same principles that allow observers to exist in the first place.
  
\end{enumerate}

A theory satisfying these conditions is self-contained in the strong sense: it explains not only the universe, but also the possibility and limitations of explanation itself.


\section{Gödel Incompleteness and Self-Contained Theories}

Gödel’s incompleteness theorems apply to formal axiomatic systems that are consistent, effectively axiomatized, and sufficiently expressive to encode arithmetic. Such systems cannot be both complete and self-verifying. These results place constraints on what can be proven \emph{within} a given formalism.

A fully self-contained Theory of Everything is not required to be formally complete in the Gödelian sense. Its aim is ontological closure rather than deductive closure: it must explain what exists and why observers arise within it, not prove every truth expressible in a fixed language. 

However, if observers are modeled as embedded subsystems that reason using finite, effectively specifiable formalisms, then Gödel-style incompleteness may arise at the level of observer-accessible description. In this case, incompleteness is an internal limitation of observer reasoning rather than a limitation of the underlying theory itself. Whether such limitations arise depends on how observers and reasoning are realized within the theory and is not assumed a priori.

Thus, Gödel incompleteness is neither an obstacle nor a guarantee for a Theory of Everything; it is a conditional result concerning formal reasoning by embedded observers.


\section*{The Source of Matter and Energy}

Why is there something rather than nothing? Rather than treating existence as a metaphysical given, we adopt the minimal assumption that all logically possible configurations are permitted unless explicitly constrained.
Under this view, ``nothingness'' corresponds to a single, highly specific configuration, while ``something'' encompasses an unbounded set of possible states. An empty set exists no less than any other set.
It is like asking why ``head'' rather than ``tail''. Both are abstract by nature. 


\section*{The Fine-Tuning Problem}

The apparent fine-tuning of physical constants poses a long-standing problem: why do the laws of physics take values so precisely compatible with complex structures and conscious life?

If we reject metaphysical selection principles and require a fully self-contained ToE, then no external agent or mechanism may choose these values.
The only remaining possibility is that the fundamental nature of reality is not defined by fixed laws, but by an underlying space of possibilities from which lawful, predictable universes emerge as typical observational outcomes.


\section*{Finiteness of the Observed Universe}

Is the universe finite or infinite? A finite universe would require a specific bound: for example, a maximum amount of information, energy, or degrees of freedom.
Any such bound immediately raises the question of its origin: why this value rather than another?

Answering that question would require a meta-theory, leading to an infinite regress of explanations. To avoid this regress, the most parsimonious conclusion is that the fundamental substrate of reality is unbounded.

The finiteness we observe—limited horizons, finite entropy, and bounded causal domains—is therefore not fundamental. It is an emergent, observer-relative property arising from local structure within an infinite underlying reality.


\section*{The Existence of an Observer}

Any theory intended to describe reality is necessarily conditioned on the existence of at least one observer, since observation is the means by which the theory is evaluated. This requirement is not an additional ontological assumption, but a minimal consistency condition on admissible models.

The existence of an observer is therefore taken as a given boundary condition: the theory must permit observer-realizing structures. However, no further assumptions are made regarding the number, identity, or distribution of observers. In particular, postulating a single privileged observer would introduce an unexplained parameter and merely displace the explanatory burden.

Instead, observer multiplicity, persistence, and typicality are treated as emergent properties to be derived from the theory’s underlying informational structure. Whether a universe admits one observer, many observers, or none at all is a prediction of the theory, not an axiom.



\end{document}
