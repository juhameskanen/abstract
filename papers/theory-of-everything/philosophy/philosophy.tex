\documentclass[11pt]{article}
% -- Common definitions for all papers

% --- Packages ---
\usepackage[utf8]{inputenc}
\usepackage[T1]{fontenc}
\usepackage{amsmath, amssymb, amsthm}
\usepackage{geometry}
\usepackage{hyperref}
\usepackage{cite}
\newcommand{\pdflink}[1]{} % By default, do nothing

% --- Page Setup ---
\geometry{letterpaper, margin=1in}
\setlength{\parindent}{0pt}
\setlength{\parskip}{1em}


% --- Author ---
\author{Abstract Researcher}

% Glossary
\newcommand{\SSP}{Spectral Selection Principle (SSP)}
\newcommand{\Lcost}{$\mathcal{L}$} % Added $ $ here
\newcommand{\Tobs}{$\mathcal{T}_{\mathrm{obs}}$} % Added $ $ here

\newcommand{\PrintGlossary}{
    \section*{Glossary of the Abstract Universe}
    \begin{description}
    \item[SSP] \SSP: The foundational principle asserting that reality is the subset of informational paths that admit the most efficient representation in the frequency domain. It replaces "laws of physics" with a statistical preference for compressibility.
    
    \item[\Tobs] Observer Filter

    \item[$\mathcal{L}$ (Spectral Encoding Length)] The measure of information required to represent a state or path $\gamma$. In this framework, $\mathcal{L}$ replaces the classical concept of \textit{Action}. Minimizing $\mathcal{L}$ is equivalent to the Principle of Least Action.
    
    \item[$\mathcal{T}_{\mathrm{obs}}$ (Observer Filter)] The subset of all possible configuration sequences capable of describing a subjective experience.
    
    \item[MSM (Microstructure Motif)] Recurring, compressible bit-patterns within configuration space (e.g., particles, fields). The density of these motifs determines the local spectral cost, manifesting as mass-energy.
    
    \item[Nyquist Horizon] The "Planck Scale" of the theory. It is the maximum frequency supportable by the discretization of configuration space. Any structure below this limit is mathematically "aliased" and cannot be rendered.
    
    \item[Induced Time] The sequential ordering of states along a path $\gamma$. An emergent property of the observer's trajectory.
    
    \item[Phase-Coherence] The informational alignment between disparate parts of the spectral encoding. This provides the mathematical basis for what is traditionally called \textit{quantum entanglement}.

    \end{description}
}


\addbibresource{../../references.bib} 

\title{Philosophy}
\date{2024}

\begin{document}
\maketitle

\section*{Philosophical Grounding of a Theory of Everything}

All existing candidates for a Theory of Everything (ToE) rely on pre-assumed structures, constraints, or constants that are not themselves derived from the theory.
These assumptions are typically accepted as primitive and lie outside the explanatory scope of the framework.

If a theory \(T\) requires an external structure \(S\) to fix its constants, boundary conditions, or domain of validity, then \(S\) constitutes information not contained within \(T\).
Likewise, if a theory fails to account for the existence of conscious observers \(O\), it omits a central empirical fact: that observations occur at all.

\[
T + S \neq T
\]
\[
T - O \neq T
\]

A genuine Theory of Everything must therefore be self-contained. It must not only describe observed regularities, but also explain why observation itself is possible.
In particular, it must clarify what is being observed, why it has the structure it does, and why observers arise within it.

\section*{The Source of Matter and Energy}

Why is there something rather than nothing? Rather than treating existence as a metaphysical given, we adopt the minimal assumption that all logically possible configurations are permitted unless explicitly constrained.

Under this view, ``nothingness'' corresponds to a single, highly specific configuration, while ``something'' encompasses an unbounded set of possible states.
Within an infinite or unconstrained space of possibilities, the relative measure of absolute non-existence is zero. Existence is therefore not exceptional; it is generic.

Matter and energy need not be postulated as fundamental substances. Instead, they emerge as stable, self-consistent structures within the space of possible configurations that support persistence and interaction.

\section*{The Fine-Tuning Problem}

The apparent fine-tuning of physical constants poses a long-standing problem: why do the laws of physics take values so precisely compatible with complex structures and conscious life?

If we reject metaphysical selection principles and require a fully self-contained ToE, then no external agent or mechanism may choose these values.
The only remaining possibility is that the fundamental nature of reality is not defined by fixed laws, but by an underlying space of possibilities from which lawful, predictable universes emerge as typical observational outcomes.


\section*{Finiteness of the Observed Universe}

Is the universe finite or infinite? A finite universe would require a specific bound: for example, a maximum amount of information, energy, or degrees of freedom.
Any such bound immediately raises the question of its origin: why this value rather than another?

Answering that question would require a meta-theory, leading to an infinite regress of explanations. To avoid this regress, the most parsimonious conclusion is that the fundamental substrate of reality is unbounded.

The finiteness we observe—limited horizons, finite entropy, and bounded causal domains—is therefore not fundamental. It is an emergent, observer-relative property arising from local structure within an infinite underlying reality.

\end{document}
