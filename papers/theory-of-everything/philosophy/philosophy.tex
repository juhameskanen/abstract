\documentclass[11pt]{article}
% -- Common definitions for all papers

% --- Packages ---
\usepackage[utf8]{inputenc}
\usepackage[T1]{fontenc}
\usepackage{amsmath, amssymb, amsthm}
\usepackage{geometry}
\usepackage{hyperref}
\usepackage{cite}
\newcommand{\pdflink}[1]{} % By default, do nothing

% --- Page Setup ---
\geometry{letterpaper, margin=1in}
\setlength{\parindent}{0pt}
\setlength{\parskip}{1em}


% --- Author ---
\author{Abstract Researcher}

% Glossary
\newcommand{\SSP}{Spectral Selection Principle (SSP)}
\newcommand{\Lcost}{$\mathcal{L}$} % Added $ $ here
\newcommand{\Tobs}{$\mathcal{T}_{\mathrm{obs}}$} % Added $ $ here

\newcommand{\PrintGlossary}{
    \section*{Glossary of the Abstract Universe}
    \begin{description}
    \item[SSP] \SSP: The foundational principle asserting that reality is the subset of informational paths that admit the most efficient representation in the frequency domain. It replaces "laws of physics" with a statistical preference for compressibility.
    
    \item[\Tobs] Observer Filter

    \item[$\mathcal{L}$ (Spectral Encoding Length)] The measure of information required to represent a state or path $\gamma$. In this framework, $\mathcal{L}$ replaces the classical concept of \textit{Action}. Minimizing $\mathcal{L}$ is equivalent to the Principle of Least Action.
    
    \item[$\mathcal{T}_{\mathrm{obs}}$ (Observer Filter)] The subset of all possible configuration sequences capable of describing a subjective experience.
    
    \item[MSM (Microstructure Motif)] Recurring, compressible bit-patterns within configuration space (e.g., particles, fields). The density of these motifs determines the local spectral cost, manifesting as mass-energy.
    
    \item[Nyquist Horizon] The "Planck Scale" of the theory. It is the maximum frequency supportable by the discretization of configuration space. Any structure below this limit is mathematically "aliased" and cannot be rendered.
    
    \item[Induced Time] The sequential ordering of states along a path $\gamma$. An emergent property of the observer's trajectory.
    
    \item[Phase-Coherence] The informational alignment between disparate parts of the spectral encoding. This provides the mathematical basis for what is traditionally called \textit{quantum entanglement}.

    \end{description}
}


\addbibresource{../../references.bib} 


\title{Philosophy}

\date{2024}

\begin{document}
\maketitle



\section*{Philosophical Grounding}

A Theory of Everything must be self-contained; any theory predicated upon empirical assumptions is merely a subset.

If a theory $T$ requires an external constraint $S$ to define its constants or boundaries, or if it fails to account for $O$ us—conscious observers—it remains incomplete.

\[
T + S \neq T
\]
\[
T - O \neq T
\]

Is the universe finite or infinite? If it were finite, the source and specific value of that constraint (e.g., "Why N bits and not N+1?") would require a meta-theory to explain.
To avoid an infinite regress of constraints, we must conclude that the fundamental nature of reality is unbounded and infinite.

Why is there something rather than nothing? The most rational answer is that "nothingness" is but one unique configuration among an infinite ensemble of possibilities.
Within an infinite potential, the mathematical probability of absolute non-existence is zero.

Therefore, the fundamental nature of reality is probabilistic and random.
The lawful predictable universe is the most probable outcome in which conscious observers are likely to find themselves.


\end{document}
