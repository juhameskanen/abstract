\documentclass[11pt]{article}
% -- Common definitions for all papers

% --- Packages ---
\usepackage[utf8]{inputenc}
\usepackage[T1]{fontenc}
\usepackage{amsmath, amssymb, amsthm}
\usepackage{geometry}
\usepackage{hyperref}
\usepackage{cite}
\newcommand{\pdflink}[1]{} % By default, do nothing

% --- Page Setup ---
\geometry{letterpaper, margin=1in}
\setlength{\parindent}{0pt}
\setlength{\parskip}{1em}


% --- Author ---
\author{Abstract Researcher}

% Glossary
\newcommand{\SSP}{Spectral Selection Principle (SSP)}
\newcommand{\Lcost}{$\mathcal{L}$} % Added $ $ here
\newcommand{\Tobs}{$\mathcal{T}_{\mathrm{obs}}$} % Added $ $ here

\newcommand{\PrintGlossary}{
    \section*{Glossary of the Abstract Universe}
    \begin{description}
    \item[SSP] \SSP: The foundational principle asserting that reality is the subset of informational paths that admit the most efficient representation in the frequency domain. It replaces "laws of physics" with a statistical preference for compressibility.
    
    \item[\Tobs] Observer Filter

    \item[$\mathcal{L}$ (Spectral Encoding Length)] The measure of information required to represent a state or path $\gamma$. In this framework, $\mathcal{L}$ replaces the classical concept of \textit{Action}. Minimizing $\mathcal{L}$ is equivalent to the Principle of Least Action.
    
    \item[$\mathcal{T}_{\mathrm{obs}}$ (Observer Filter)] The subset of all possible configuration sequences capable of describing a subjective experience.
    
    \item[MSM (Microstructure Motif)] Recurring, compressible bit-patterns within configuration space (e.g., particles, fields). The density of these motifs determines the local spectral cost, manifesting as mass-energy.
    
    \item[Nyquist Horizon] The "Planck Scale" of the theory. It is the maximum frequency supportable by the discretization of configuration space. Any structure below this limit is mathematically "aliased" and cannot be rendered.
    
    \item[Induced Time] The sequential ordering of states along a path $\gamma$. An emergent property of the observer's trajectory.
    
    \item[Phase-Coherence] The informational alignment between disparate parts of the spectral encoding. This provides the mathematical basis for what is traditionally called \textit{quantum entanglement}.

    \end{description}
}


\addbibresource{../../references.bib} 

\title{Dimensional Optimality of Observer Boundaries}

\date{2024}

\begin{document}
\maketitle

\section{Motivation}

We treat geometry as the means of enforcing a well defined inside–outside separation and preventing uncontrolled information mixing with the environment.
Given that geometry is required, we apply a minimum description length (MDL) principle to evaluate which geometric realizations are most probable.
The central claim is that three spatial dimensions are not necessary, but are optimal for realizing bounded, persistent observers with well-defined interior and exterior regions.

The central claim is that three spatial dimensions are not necessary, but are \emph{optimal} for encoding bounded, persistent observers with well-defined interior/exterior separation.

\section{Setup}

Let $D$ denote an execution trace representing a bounded, persistent observer. By an observer we mean an information-processing system satisfying:
\begin{enumerate}
    \item Persistence over time,
    \item A well-defined interior state,
    \item A boundary separating internal from external information,
    \item Multiple concurrent internal processes (e.g.\ signaling, energy intake, waste removal).
\end{enumerate}

Let $G_d$ denote a geometric encoding of $D$ in $d$ spatial dimensions, and let $L(G_d)$ denote the minimal description length required to encode:
\begin{itemize}
    \item The observer interior,
    \item The observer boundary $\partial \Omega_d$,
    \item The routing of internal processes within the geometry.
\end{itemize}

We consider the MDL-induced measure:
\[
\mu(G_d) \propto e^{-L(G_d)}.
\]

The question is which spatial dimension $d$ minimizes $L(G_d)$.

\section{Main Claim}

\textbf{Claim.} For execution traces corresponding to bounded, persistent observers,
\[
L(G_3) = \min_d L(G_d),
\]
and therefore three-dimensional geometric encodings dominate the MDL-induced measure.

Lower-dimensional encodings exist but incur prohibitively large description length, while higher-dimensional encodings introduce superfluous geometric degrees of freedom without compensating reduction in encoding cost.

\section{Sketch of Argument}

\subsection{Boundary and routing costs}

In $d$ spatial dimensions, the observer boundary $\partial \Omega_d$ is $(d-1)$-dimensional. For an observer of characteristic linear size $R$, the boundary encoding cost scales as:
\[
L_{\text{boundary}}(d) \sim C_1 R^{\,d-1}.
\]

However, boundary size alone does not determine optimality. A critical additional cost arises from the routing of multiple independent internal processes within the observer.

Let $k$ denote the number of concurrent internal flows (e.g.\ circulation, signaling, energy transport). Let $C_d(k)$ denote the minimal description length required to embed $k$ disjoint routing channels in $d$ dimensions without mutual interference.

\subsection{One-dimensional encodings}

In $d=1$, all internal processes are totally ordered. No two independent channels can bypass each other without intersection. As a result:
\[
C_1(k) = \infty \quad \text{for } k > 1.
\]
Thus, one-dimensional encodings cannot support complex observers.

\subsection{Two-dimensional encodings}

In $d=2$, boundaries are one-dimensional curves, and internal routing is constrained by planar topology. While disjoint routing is possible in principle, each additional internal channel forces global coordination to avoid intersections.

As $k$ grows, the description length required to specify non-intersecting routes grows superlinearly:
\[
C_2(k) \sim \exp(k),
\]
due to unavoidable crossings, global constraints, and topological fragility. Small perturbations require large-scale boundary and routing updates, dramatically increasing MDL cost.

Thus, while two-dimensional observers are not impossible, their geometric encodings are overwhelmingly inefficient.

\subsection{Three-dimensional encodings}

In $d=3$, boundaries are two-dimensional surfaces, and volumetric separation becomes possible. Independent internal processes can be routed through disjoint tunnels, layers, and cavities with purely local specification.

As a result:
\[
C_3(k) \sim O(k),
\]
and the total description length is minimized:
\[
L(G_3) \approx L_{\text{interior}} + C_1 R^2 + C_2 k.
\]

Three dimensions are the minimal spatial setting in which:
\begin{itemize}
    \item Stable boundaries exist,
    \item Independent internal flows can bypass each other locally,
    \item Boundary modifications remain local rather than global.
\end{itemize}

\subsection{Higher-dimensional encodings}

For $d>3$, routing complexity does not improve further: $C_d(k) \sim O(k)$. However, boundary encoding costs grow rapidly:
\[
L_{\text{boundary}}(d) \sim R^{\,d-1},
\]
and additional geometric degrees of freedom must be explicitly specified, increasing MDL without providing functional benefit.

Thus:
\[
L(G_d) > L(G_3) \quad \text{for } d > 3.
\]

\section{Measure Concentration}

Under the MDL-induced measure:
\[
\mu(G_d) \propto e^{-L(G_d)},
\]
even modest differences in description length lead to exponential suppression. Since $L(G_3)$ is minimal, three-dimensional encodings overwhelmingly dominate the measure.

This explains why observers most likely find themselves embedded in three spatial dimensions, without invoking anthropic selection or special physical laws.


\section{Related Work and Novelty}

The question of why observers find themselves embedded in three spatial dimensions has been approached from multiple perspectives, but none capture the information-theoretic, MDL-based argument presented here.

In physics, dimensional constraints have been considered in the context of planetary stability, inverse-square laws, and the propagation of forces \cite{penrose2010, hawking1996nature, hawking1988}. Such arguments are contingent on the dynamics of specific physical laws and do not generalize to abstract observers or to the informational structure of existence.

In computational models, one- and two-dimensional cellular automata have been used to investigate the emergence of localized persistent structures \cite{zuse1970calculating, lloyd2006, deutsch1997}. It is well-known that one-dimensional automata cannot support multiple disjoint information channels without interference, and that two-dimensional systems face rapidly growing coordination complexity for multi-channel routing. Three-dimensional automata allow volumetric separation and more efficient routing, but prior work has largely remained confined to specific CA rules and has not formalized an MDL-based measure over geometric embeddings.

From the perspective of theoretical computer science, the embedding of graphs and routing of disjoint channels in low-dimensional spaces has been studied extensively \cite{LiVitanyi, Solomonoff1964, Chaitin1975}. These results highlight that 1D and 2D topologies incur high or even unbounded cost for independent flows, while 3D embeddings permit linear-cost local routing. However, previous work treats these results in the context of abstract networks or circuits, rather than as a general principle governing observer existence.

Our contribution extends these insights by framing observers as bounded execution traces with multiple internal processes and a required inside–outside separation. Geometry is treated not as an optional representational choice but as a **necessary structural vertex** in a trinity of analytic (wavefunction), discrete (particles), and geometric components, each of which is essential to the existence and stability of the observer. Using a minimum description length (MDL) measure over possible geometric embeddings, we show that three spatial dimensions are optimal: they minimize the total description length required to encode the interior, boundary, and routing of internal processes. Lower-dimensional embeddings incur exponentially higher description length, while higher-dimensional embeddings introduce redundant degrees of freedom without reducing routing complexity. This combination of information-theoretic formalism, MDL measure, and the trinary observer structure is, to our knowledge, not present in prior literature and establishes a new explanatory framework for the emergence of three-dimensional space for observers.




\section{Conclusion}

Three spatial dimensions are not required for observers to exist, but they are optimal for encoding bounded, persistent observers with well-defined interior boundaries and multiple concurrent internal processes.

Under an MDL-based measure over geometric encodings, three-dimensional space emerges as the overwhelmingly probable setting for observers.




\end{document}
