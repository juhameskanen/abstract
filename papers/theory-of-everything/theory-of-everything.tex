\documentclass[11pt]{article}
\usepackage{amsmath, amssymb, amsfonts}
\usepackage{tikz-cd}
\usepackage{hyperref}
\usepackage{graphicx}
\usepackage{enumitem}
\usepackage{xcolor}
\usepackage{lipsum}
\usepackage{biblatex}

\addbibresource{../references.bib}

\title{Informational Ontology and Observer-Conditioned Typicality}
\author{Juha Meskanen}
\date{2025}

\begin{document}

\maketitle

\begin{abstract}
We present an informational ontology in which physical laws emerge from observer-conditioned typicality. Modeling the universe as an abstract, pre-geometric configuration space, we show that quantum mechanics and general relativity arise as complementary compression channels—spectral and geometric—optimized for the persistence of observers. We formalize this via the D--$\psi$--G decomposition, which partitions the complexity of observer-compatible paths and identifies fundamental constants $\hbar$, $G$ and $c$ as scaling weights for spectral and geometric encoding. Our framework naturally resolves the Boltzmann Brain paradox: low-complexity, observer-rich universes dominate the measure, while isolated high-entropy fluctuations are exponentially suppressed. Simulations demonstrate that smooth spacetime and wave-like dynamics emerge from maximization of description-length-based probability. These results suggest that physical laws are not primitive, but are the statistically dominant regularities of a maximally compressible informational substrate, providing a unified, observer-centric perspective on quantum mechanics, gravity, and cosmology.
\end{abstract}

\noindent \textbf{Keywords:} Information Theory, Observer-Conditioned Typicality, Emergent Gravity, Quantum Foundations, D--$\psi$--G Decomposition, Algorithmic Complexity, Spectral Compression, Pre-Geometric Universe.


\tableofcontents
\bigskip

\section{Introduction}

The nature of reality can be formalized as a collection of abstract information without intrinsic geometry, ordering, or dynamics. Observers are structured informational entities, and their existence selects for configurations exhibiting regularity and stability. This paper develops a formal, stand-alone framework in which quantum mechanics and general relativity emerge as complementary compression channels acting on the same informational substrate. We introduce an observer-conditioned probability measure and the D--$\psi$--G decomposition to unify discrete, spectral, and geometric representations. This formalism reproduces the familiar macroscopic laws of physics while remaining fully consistent with a timeless, pre-geometric ontology.

\section{Ontological Premises}

\subsection{Fundamental Information Space}

For definiteness, we adopt a binary representation of configurations. This choice is purely formal and convenient. Nothing in the framework depends on the universe being fundamentally binary; any discrete or symbolic representation would suffice. 

Let $\mathcal{C} = \{0,1\}^n$ denote the complete set of binary configurations of $n$ bits. Each element $s \in \mathcal{C}$ is a static configuration of the universe. At this fundamental level:
\begin{itemize}
    \item configurations are unordered,
    \item all configurations exist equally,
    \item there is no external time or dynamics.
\end{itemize}

A permutation $\sigma: \mathcal{C} \to \mathcal{C}$ defines a total ordering of all $2^n$ configurations; conceptually, this can be viewed as a ``movie'' containing all possible states. While mathematically valid, most permutations are maximally incoherent, containing no stable structure. Observers interact only with structured \emph{paths}, finite sequences $\gamma = (s_1, s_2, \dots, s_T)$ preserving informational continuity.

\subsection{Observers and Induced Time}

Observers are subpatterns instantiated along paths through $\mathcal{C}$. They require:
\begin{itemize}
    \item continuity and memory,
    \item the ability to model and predict internal states,
    \item well-defined boundaries to preserve identity.
\end{itemize}

Time is not fundamental; it is induced as the ordering along the observer path. Observed dynamics, cosmology, and causality are models of path structure rather than external processes.

\section{Compression and Spectral Complexity}

\subsection{Spectral Complexity}

We define \emph{spectral complexity} $C_{\rm spec}[\gamma]$ as the minimal representation of a path $\gamma$ in terms of frequencies, phases, and amplitudes necessary to capture observer-relevant information. Low $C_{\rm spec}$ corresponds to smooth, compressible worlds; high $C_{\rm spec}$ corresponds to chaotic, uncompressible worlds. Observer-compatible histories overwhelmingly arise from paths minimizing $C_{\rm spec}$.

\subsection{Wavefunction as Optimal Compression}

The quantum wavefunction encodes an ensemble of observer-compatible paths:
\begin{itemize}
    \item interference reflects shared substructure across paths,
    \item probabilities correspond to relative description weights,
    \item linearity and complex amplitudes emerge from minimal spectral encoding.
\end{itemize}

\section{D--$\psi$--G Decomposition and Unified Probability}

\subsection{Observer-Conditioned Probability}

We formalize observer-typicality through a probability measure over histories $\gamma \in \Gamma_O$, the set of all paths sustaining observer $O$:
\begin{equation}
\mathbb{P}(\gamma \mid O) = \frac{1}{Z_O} \exp\Big[-\lambda \mathcal C_O[\gamma]\Big],
\end{equation}
where $\mathcal{C}_O[\gamma]$ is a description-length-based complexity functional, $\lambda$ is a fidelity parameter, and $Z_O$ normalizes the measure. Histories minimizing $\mathcal{C}_O[\gamma]$ dominate probability, yielding emergent laws of physics.

\subsection{D--$\psi$--G Decomposition}

We decompose $\mathcal C_O[\gamma]$ additively into three orthogonal channels:
\begin{equation}
\mathcal C_O[\gamma] = \alpha_D \mathcal C_D[\gamma] + \alpha_\psi \mathcal C_\psi[\gamma] + \alpha_G \mathcal C_G[\gamma],
\end{equation}
where:
\begin{itemize}
    \item $\mathcal C_D[\gamma]$: discrete description of particle-like states,
    \item $\mathcal C_\psi[\gamma]$: spectral (wavefunction) representation,
    \item $\mathcal C_G[\gamma]$: geometric (spacetime) encoding enforcing observer boundaries.
\end{itemize}

\subsubsection{Illustration}

\begin{center}
\begin{tikzcd}[row sep=5em, column sep=5em]
& \psi(\mathcal{I}_n) \arrow[dl, "Sampling"] & \\
D(\mathcal{I}_n) \arrow[rr, "Geometric"] & & G(\mathcal{I}_n) \arrow[ul, "Compression"]
\end{tikzcd}
\end{center}

This diagram illustrates the mutually reinforcing relationship between discrete instantiation $D$, spectral compression $\psi$, and geometric encoding $G$. Together, they generate observer-compatible, high-probability histories.

\subsection{Emergent Physics}

The additive decomposition implies that quantum behavior dominates at scales where $\mathcal C_\psi$ is significant, while classical spacetime emerges where $\mathcal C_G$ dominates. The chain of emergence is thus:
\begin{equation}
\text{Maximal Compression} \;\Rightarrow\; \text{Maximal Probability} \;\Rightarrow\; \text{Maximal Predictability}.
\end{equation}

\section{Interpretation}

\subsection{Smooth Worlds and Typicality}

Although chaotic observer paths exist, the measure overwhelmingly favors smooth, compressible configurations. Observers find themselves in worlds with predictable physical laws not by necessity, but due to combinatorial dominance under compression-weighted probability.

\subsection{Geometry as Boundary Enforcement}

Persistent observer identity requires well-defined internal-external distinctions. Geometry enforces these boundaries, producing the familiar three-dimensional structures observed by intelligent entities.

\section{Simulations}

Demonstrations of emergent physics in this framework include:

\begin{itemize}
    \item \texttt{simulations/emergent\_atom.py} -- discrete and spectral emergence of particle-like structures
    \item \texttt{simulations/emergent\_gravity.py} -- emergent geometric and gravitational features
\end{itemize}


\subsubsection{Spectral Complexity via Entropy}

The spectral component $\mathcal C_\psi[\gamma]$ can be formalized using the Shannon or Rényi entropy of the path's spectral decomposition. Let $\gamma = (s_1, s_2, \dots, s_T)$ denote the observer path in the discrete configuration space $\mathcal{C}$. Define $\hat \gamma(\omega)$ as the discrete Fourier transform (DFT) of $\gamma$ (alternatively, a discrete cosine transform can be used for real-valued paths):

\[
\hat \gamma(\omega) = \sum_{t=1}^{T} s_t \, e^{-2\pi i \omega t/T}, \quad \omega = 0,1,\dots,T-1.
\]

Then the spectral complexity is
\[
\mathcal C_\psi[\gamma] \;\sim\; H_\alpha(\hat \gamma(\omega)) = \frac{1}{1-\alpha} \log \sum_\omega |\hat \gamma(\omega)|^{2\alpha},
\]
or in the Shannon case ($\alpha \to 1$),
\[
\mathcal C_\psi[\gamma] \;\sim\; H(\hat \gamma(\omega)) = - \sum_\omega |\hat \gamma(\omega)|^2 \log |\hat \gamma(\omega)|^2.
\]

Low-entropy paths require few non-negligible spectral coefficients, minimizing $\mathcal C_\psi$ and thereby dominating the observer-conditioned measure. This formalizes the emergence of wave-like dynamics as a statistical consequence of spectral compression.


\subsection{Emergent Fundamental Constants}

Within the D--$\psi$--G framework, the observed fundamental constants $\hbar$, $G$, and $c$ naturally arise as scaling weights associated with distinct informational compression channels. Specifically, $\hbar$ quantifies the cost of spectral encoding in the $\psi$ channel, governing the minimal action per unit of spectral complexity; $G$ quantifies the geometric encoding cost in the $G$ channel, controlling the effective curvature of low-complexity spacetime; and $c$ emerges as the maximal rate at which information can propagate along low-complexity geometric paths, setting the causal speed limit.  

These constants are not postulated a priori, but are unavoidable parameters linking the abstract informational substrate to empirically observed physics. While their qualitative roles are clear, deriving their numerical values from first principles within this framework remains an open problem. We therefore treat $\hbar$, $G$, and $c$ as emergent scaling factors, and leave the precise quantitative derivation as a subject for future work. Simulations indicate that typical low-complexity paths reproduce smooth spacetime and wave-like dynamics consistent with effective $\hbar$, $G$, and $c$, providing empirical support for this interpretation.



\section{Conclusion}

We have presented a observer-centric informational ontology in which quantum mechanics and general relativity emerge as complementary compression schemes. Observers induce typicality constraints that select for smooth, compressible, and geometrically bounded worlds. This framework explains the apparent laws of physics as statistical consequences of minimal-description-length selection and provides a coherent basis for simulations of emergent phenomena.



\section*{Acknowledgments and Foundational References}

The framework presented here draws directly on the formal foundations of algorithmic information theory and inductive inference:

\begin{itemize}
    \item \textbf{Kolmogorov Complexity} \cite{Kolmogorov1965,LiVitanyi}: formalizes the minimal description length of finite strings, providing the rigorous basis for observer-conditioned path complexity.
    \item \textbf{Solomonoff Induction} \cite{Solomonoff1964}: defines a universal prior over all computable sequences, motivating the statistical weighting of observer-compatible paths.
    \item \textbf{Chaitin's Algorithmic Information Theory} \cite{Chaitin1975}: demonstrates the correspondence between program-size complexity and probabilistic occurrence of structures, directly inspiring the $D$--$\psi$--$G$ additive decomposition.
\end{itemize}

Our derivation of emergent physics is an application of these well-established theoretical principles to observer-centered informational universes.


\printbibliography

\end{document}
