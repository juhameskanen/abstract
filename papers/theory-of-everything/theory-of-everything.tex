\documentclass[11pt]{article}
% -- Common definitions for all papers

% --- Packages ---
\usepackage[utf8]{inputenc}
\usepackage[T1]{fontenc}
\usepackage{amsmath, amssymb, amsthm}
\usepackage{geometry}
\usepackage{hyperref}
\usepackage{cite}
\newcommand{\pdflink}[1]{} % By default, do nothing

% --- Page Setup ---
\geometry{letterpaper, margin=1in}
\setlength{\parindent}{0pt}
\setlength{\parskip}{1em}


% --- Author ---
\author{Abstract Researcher}

% Glossary
\newcommand{\SSP}{Spectral Selection Principle (SSP)}
\newcommand{\Lcost}{$\mathcal{L}$} % Added $ $ here
\newcommand{\Tobs}{$\mathcal{T}_{\mathrm{obs}}$} % Added $ $ here

\newcommand{\PrintGlossary}{
    \section*{Glossary of the Abstract Universe}
    \begin{description}
    \item[SSP] \SSP: The foundational principle asserting that reality is the subset of informational paths that admit the most efficient representation in the frequency domain. It replaces "laws of physics" with a statistical preference for compressibility.
    
    \item[\Tobs] Observer Filter

    \item[$\mathcal{L}$ (Spectral Encoding Length)] The measure of information required to represent a state or path $\gamma$. In this framework, $\mathcal{L}$ replaces the classical concept of \textit{Action}. Minimizing $\mathcal{L}$ is equivalent to the Principle of Least Action.
    
    \item[$\mathcal{T}_{\mathrm{obs}}$ (Observer Filter)] The subset of all possible configuration sequences capable of describing a subjective experience.
    
    \item[MSM (Microstructure Motif)] Recurring, compressible bit-patterns within configuration space (e.g., particles, fields). The density of these motifs determines the local spectral cost, manifesting as mass-energy.
    
    \item[Nyquist Horizon] The "Planck Scale" of the theory. It is the maximum frequency supportable by the discretization of configuration space. Any structure below this limit is mathematically "aliased" and cannot be rendered.
    
    \item[Induced Time] The sequential ordering of states along a path $\gamma$. An emergent property of the observer's trajectory.
    
    \item[Phase-Coherence] The informational alignment between disparate parts of the spectral encoding. This provides the mathematical basis for what is traditionally called \textit{quantum entanglement}.

    \end{description}
}



\addbibresource{../references.bib}

\title{Informational Ontology and Observer-Conditioned Typicality}
\date{2025}

\begin{document}

\maketitle


\ifdefined\ishtml
\begin{center}
\small \href{theory-of-everything-theory-of-everything.pdf}{Download PDF Version}
\end{center}
\fi

\section{Abstract}

Building on the conclusions derived in Paper 1 - 3, we propose new approach to unify GR and QM in which the nature of everything is fundamentally unstructured and prior-free.
The universe emerges as the most probable configuration of information describing the observer. Time, Quantum Mechanics and General Relativity
are two ortogonal descriptions of the most typical universe in which an intelligent observer can expect to find themselves.
We present simulation that demonstrate fully emergent gravity and faminilar quantum mechanical features.


\section{Axiomatic Premises}

We begin with two axioms that serve as the premises for the entire derivation.

\subsection*{Axiom 1: Wavefunction as Compression}

Quantum Mechanics is not a fundamental property of the universe, but a compression system.

This assumption is based on observational evidence; the universe, at its smallest scale, appear to wave
according to the determistic wave function. Interference arises from representational overlap in the compressed encoding of observer-compatible histories.


\subsection*{Axiom 2: Geometry as Optimal Model for defining Informational Boundaries}

Spacetime (GR) is not fundamental property of the universe, but optimal model for defining independent
informational entities with well defined ``inside'' property.

Geometry emerges as the unique stable representational framework capable of enforcing persistent informational boundaries under long observer histories.


\subsection{Core Thesis}

Quantum Mechanics General Relativity serve as two mutually orthogonal, optimal compression systems resulting maximal probability of existence of the observer. 

Maximal compression -> maximal probability -> maximal predictability and smoothness 


\begin{tikzcd}[row sep=6em, column sep=6em]
& \psi(\mathcal{I}_n) \arrow[dl, "Sampling"] & \\
D(\mathcal{I}_n) \arrow[rr, "Geometric"] & & G(\mathcal{I}_n) \arrow[ul, "Compression"]
\end{tikzcd}
\textif{D-\(\psi\)-G - Discrete, analytic and geometric perspectives}




\section{Derivation of $IaM^e$}

We begin from a minimal and representation-independent premise: reality is constituted by abstract information alone.
There is no privileged encoding, alphabet, ordering, geometry, probability measure, or temporal structure at the fundamental level.
Any concrete representation is a matter of convenience rather than ontology.


\subsection{Timeless Configuration Space}

For definiteness, we fix a binary representation.
With $n$ bits of information, the configuration space is
\[
\mathcal{C} = \{0,1\}^n,
\]
with cardinality $|\mathcal{C}| = 2^n$.
Each element $s \in \mathcal{C}$ is a complete, static configuration.

These configurations:
\begin{itemize}
    \item are unordered,
    \item all exist equally,
    \item are ontologically symmetric.
\end{itemize}

Most configurations resemble unstructured noise; only a vanishing fraction encode observers or observer-like structures.


\subsection{Permutations and Movies}

One may conceptually arrange the configurations into complete orderings.
A permutation is a bijection
\[
\sigma : \mathcal{C} \to \mathcal{C},
\]
equivalently, a total ordering of all $2^n$ configurations.
The number of such permutations is $(2^n)!$.

Each permutation may be viewed as a complete ``movie'' containing every possible configuration exactly once.
This construction is mathematically valid, but observationally excessive:
almost all permutations are maximally incoherent and devoid of stable structure.
Permutations therefore serve as a conceptual upper bound, not as the fundamental object of experience.

\subsection{Observer Paths}

What observers actually experience are not permutations but \emph{paths}.
An observer path is a finite or semi-infinite sequence
\[
\gamma = (s_1, s_2, \dots, s_T), \quad s_i \in \mathcal{C},
\]
with no requirement to visit all configurations, and with possible repetition.

Observer paths:
\begin{itemize}
    \item traverse only a tiny subset of $\mathcal{C}$,
    \item preserve informational continuity,
    \item support memory, self-modeling, and prediction.
\end{itemize}

Every observer path is a subsequence of many permutations, but permutations are not fundamental.
The correct containment relation is:
\[
\text{Observer paths}
\;\subset\;
\text{finite sequences over } \mathcal{C}
\;\subset\;
\text{permutations of } \mathcal{C}.
\]


\subsection{Time as Induced Ordering}

The configuration space $\mathcal{C}$ is timeless.
Time is not fundamental; it is induced as the ordering along an observer path.

Observers traverse short, coherent subpaths through $\mathcal{C}$.
From regularities along these paths, they infer a past and extrapolate a future.
Cosmology, dynamics, and causality are therefore models of the path structure, not of the full configuration space.

This resolves the Wheeler--DeWitt timelessness without introducing external time or dynamics.


\subsection{Observer Filtering and Measure}

An observer corresponds to a structured subpattern instantiated along a path through $\mathcal{C}$.

Among all possible paths, only a vanishing subset include the observer.
These form the observer-compatible set $\mathcal{T}_{\mathrm{obs}}$.

Observer paths with high stepwise novelty---such as Boltzmann-brain-like trajectories involving random jumps across configuration space---are logically and ontologically possible.
However, they carry exponentially suppressed measure.

The distinction is not one of possibility, but of typicality.
Paths that do not compress well have negligible weight under any reasonable description-length-based measure.


\section{Spectral Complexity}

We define \emph{Spectral Complexity} $C_{\rm spec}$ as a physically grounded measure of description length, computed from the spectral decomposition of observer-relevant configurations. 
Spectral Complexity is one admissible instantiation of the general information functional under assumptions of approximate locality and smoothness.
Low $C_{\rm spec}$ corresponds to smooth, low-bandwidth, compressible worlds; high $C_{\rm spec}$ corresponds to chaotic, high-frequency, uncompressible worlds. 
The Spectral Minimum Description Length principle states that observer-compatible worlds are overwhelmingly drawn from configurations minimizing $C_{\rm spec}$, subject to discrete instantiation (D) and geometric filtering (G). 
This provides a computable, continuous, and physically meaningful replacement for algorithmic Kolmogorov complexity, grounding the emergence of lawful physics in spectral compressibility.


\subsection{Wavefunction as Compression}

The quantum wavefunction is not a physical medium, nor a dynamical law.
It plays the role of an optimal compression of the ensemble of observer-compatible paths.

Much like a video codec compresses a family of movies by exploiting shared structure, the wavefunction encodes:
\begin{itemize}
    \item large ensembles of compatible paths,
    \item interference as shared substructure in the compressed description,
    \item probabilities as relative description weight.
\end{itemize}

There is no external player, no execution substrate, and no need for temporal unfolding.
The existence of the information itself is sufficient for lived experience.


\subsection{Why Smooth Compressed Worlds Dominate}

Although chaotic and non-compressed observer paths exist, smooth, compressed, predictable paths dominate measure.
Given any fixed amount of information, that information can be arranged to form smooth observer movies in compressed format in vastly more ways than it can be arranged to form highly irregular ones without compression.

A random draw from the ensemble of observer paths is therefore overwhelmingly likely to yield a lawful, continuous world with compression artifacts (QM).
Our experience of smooth physics is not imposed by law, but emerges from combinatorial dominance under compression-weighted selection.

Geometry emerges as geometric interpretation of information. Information of the observer, from survivals point of view
must not leak out. We must have well defined boundary of information, and that's why we have a sense of
3D body.


\subsection{Constraints, Spectral Complexity and Induction}

In this framework, both quantum behavior and spacetime geometry emerge naturally from **observer-centric informational constraints**.
These constraints are captured by a trinity of interdependent principles:

\begin{itemize}
\item \textbf{Maximal compression ($\psi$):} 
  Observer histories are encoded in a form of minimal description length, ensuring that their evolution is both highly probable and effectively law-like.
  The universal wavefunction serves as the optimal compression algorithm for all underlying degrees of freedom.

\item \textbf{Minimal discrete instantiation ($D$):} 
  Observer histories must be realized at finite resolution.
  Discrete entities—such as particles—provide the minimal sufficient representation that faithfully approximates the underlying continuous wavefunction.

\item \textbf{Stable boundaries ($G$):} 
  Observers require a persistent inside–outside distinction to preserve identity and prevent information leakage.
  Geometry enforces these boundaries, supplying the only known mechanism to maintain coherent observer structure across spacetime.
\end{itemize}

Together, these principles define the structure of observer-compatible universes: histories are highly compressible, discretely instantiated, and geometrically bounded, yielding emergent laws and spacetime from purely informational constraints.


\subsection{Proof of Concept}

\begin{itemize}
\item simulations/emergent\_atom.py
\item simulations/emergent\_gravity.py
\end{itemize}

\printbibliography

\end{document}

