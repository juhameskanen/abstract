\documentclass[11pt]{article}
% -- Common definitions for all papers

% --- Packages ---
\usepackage[utf8]{inputenc}
\usepackage[T1]{fontenc}
\usepackage{amsmath, amssymb, amsthm}
\usepackage{geometry}
\usepackage{hyperref}
\usepackage{cite}
\newcommand{\pdflink}[1]{} % By default, do nothing

% --- Page Setup ---
\geometry{letterpaper, margin=1in}
\setlength{\parindent}{0pt}
\setlength{\parskip}{1em}


% --- Author ---
\author{Abstract Researcher}

% Glossary
\newcommand{\SSP}{Spectral Selection Principle (SSP)}
\newcommand{\Lcost}{$\mathcal{L}$} % Added $ $ here
\newcommand{\Tobs}{$\mathcal{T}_{\mathrm{obs}}$} % Added $ $ here

\newcommand{\PrintGlossary}{
    \section*{Glossary of the Abstract Universe}
    \begin{description}
    \item[SSP] \SSP: The foundational principle asserting that reality is the subset of informational paths that admit the most efficient representation in the frequency domain. It replaces "laws of physics" with a statistical preference for compressibility.
    
    \item[\Tobs] Observer Filter

    \item[$\mathcal{L}$ (Spectral Encoding Length)] The measure of information required to represent a state or path $\gamma$. In this framework, $\mathcal{L}$ replaces the classical concept of \textit{Action}. Minimizing $\mathcal{L}$ is equivalent to the Principle of Least Action.
    
    \item[$\mathcal{T}_{\mathrm{obs}}$ (Observer Filter)] The subset of all possible configuration sequences capable of describing a subjective experience.
    
    \item[MSM (Microstructure Motif)] Recurring, compressible bit-patterns within configuration space (e.g., particles, fields). The density of these motifs determines the local spectral cost, manifesting as mass-energy.
    
    \item[Nyquist Horizon] The "Planck Scale" of the theory. It is the maximum frequency supportable by the discretization of configuration space. Any structure below this limit is mathematically "aliased" and cannot be rendered.
    
    \item[Induced Time] The sequential ordering of states along a path $\gamma$. An emergent property of the observer's trajectory.
    
    \item[Phase-Coherence] The informational alignment between disparate parts of the spectral encoding. This provides the mathematical basis for what is traditionally called \textit{quantum entanglement}.

    \end{description}
}


\addbibresource{../references.bib}

\title{The Informational Derivation of Reality: Consciousness, Time, and Ontological Equivalence}

\date{2001}

\begin{document}

\maketitle

\begin{abstract}
  We show that humans must be formal axiomatic systems if only four assumptions hold.
  From this, we proof that simulated humans must instantiate subjective experience, and time is not a fundamental property of the universe
  but an emergent, internal property of the observer. These results establish a strict \textbf{ontological equivalence}:
  the computer and the virtual simulated universe are not causally related, but merely different arrangements of the same finite information set.
  Reality is therefore fundamentally abstract and substrate-independent, emerging from an ensemble of informational configurations rather than from a physical substrate.
  The resulting claims are falsifiable.
\end{abstract}

  
\ifdefined\ishtml
\begin{center}
\small \href{humans-as-axiomatic-systems.pdf}{Download PDF Version}
\end{center}
\fi

\section{Axiomatic Premises}
The argument begins with four axioms that serve as the premises for the entire derivation.

\subsection*{Axiom 1: Genetic Encoding of Subjective Experience}
The human genome encodes sufficient information to construct a conscious, pain-sensitive human.
\[
A_1: D \rightarrow H \rightarrow S
\]
(Where $D$ is the DNA/Genome, $H$ is the human organism, and $S$ is subjective experience.)

\subsection*{Axiom 2: Physicality and Axiomatic Law}
DNA and the human organism are composed solely of ordinary physical matter governed by physical laws ($P$).

\[
A_2: H \in \text{Axiomatic System}(P, D)
\]

\subsection*{Axiom 3: Generalized Church-Turing Thesis Holds}

\[
A_3: P \text{ is Axiomatic} \implies \exists H' \text{ s.t. } \text{Trace}(H') \cong \text{Trace}(H)
\]
(Where $\cong$ denotes an isomorphism of causal/relational structures.)

\subsection*{Axiom 4: Causal Efficacy of Pain}
Subjective experience ($S$, e.g., pain) is a \textbf{causally efficacious} property of the human system. It is not an epiphenomenon, and thus has measurable, behavioral effects.
\[
A_4: S \text{ is causally efficacious.}
\]
If $H' \equiv H$, then $H'$ must have $S$ to maintain identical behavior.

\section{Deduction 1: Substrate Independence and Time as an Internal Property of the Observer}

\subsection{The Optimization Argument}

Consider a DNA simulation running on a computer, denoted by $T_{\text{alg}}$, consisting of code (the laws of physics) and data (the state of the simulated universe).
One can gradually optimize the code by introducing lookup tables, eventually replacing all computation with a static, precomputed dataset, denoted by $T_{\text{data}}$.

\begin{itemize}
    \item Let $E_{\text{int}}$ be Alice's experience of time and pain (the internal state transitions).
    \item Let $E_{\text{ext}}$ be the computer's external runtime (number of CPU cycles).
\end{itemize}

\textbf{Premise:} Code optimization changes $E_{\text{ext}}$ but preserves $E_{\text{int}}$.
\[
\text{Optimization}(T_{\text{alg}}) \rightarrow T_{\text{data}} \implies E_{\text{int}}(T_{\text{alg}}) \equiv E_{\text{int}}(T_{\text{data}})
\]

In this limit, the external runtime \(E_{\text{ext}}\) becomes zero in the sense that no state transitions are executed; the complete execution trace exists as static data.
One can therefore ask: does Alice's consciousness still persist in $T_{\text{data}}$?

If consciousness were to cease in $T_{\text{data}}$, it would imply that $E_{\text{int}}$ depends on $E_{\text{ext}}$, which necessitates a minimum code/data ratio for subjective experience.
This minimum ratio would be a \textbf{new, non-physical constant} imposed on $A_2$, leading to a contradiction.

\begin{quote}
  \textbf{Conclusion:} Consciousness can emerge from pure static data.
  Time and subjective experience ($S$) must emerge solely from the relationships among informational states, not from the external runtime.
\end{quote}

\subsection{The Multi-threaded Argument}

Consider a multi-threaded computer running two DNA simulations, $\tau_A$ (Alice) and $\tau_B$ (Bob), concurrently, with a minimal time slice of one CPU cycle per thread.
The execution trace of the computer is then an interleaved sequence of segments drawn from both $\tau_A$ and $\tau_B$.
As the number of threads (simulated observers) increases without bound, the resulting execution trace asymptotically approaches white noise.
Consequently, bits originating from $\tau_A$ are separated by increasingly large intervals of unrelated data.

\subsubsection*{The Threshold of Subjective Continuity}
Is Alice still conscious in this limit? If one maintains that $E_{\text{int}}$ vanishes as the interleaving becomes extreme, one must define a specific \textbf{thread density} or \textbf{bit-contiguity threshold} required for subjectivity. Such a threshold would constitute a new physical constant, which contradicts the completeness of $A_2$.

\begin{quote}
\textbf{Conclusion:} Conscious experience can arise from static white noise, without requiring any external computation or observation.
\end{quote}

\section{The Functionalist Proof by Contradiction}

\begin{enumerate}
    \item \textbf{Assumption (Objection):} A simulation $H'$ exists such that $H' \equiv H$ (physical/behavioral equivalence) but $S(H') = \emptyset$ (lacks consciousness/sense of pain) \cite{Chalmers1996}.
    \[
    \text{Behavior}(H') = \text{Behavior}(H) \land S(H') \neq S(H)
    \]
    \item \textbf{Premise:} From $A_4$, the behavior of $H$ is a function of its physical inputs \textbf{and} its subjective experience: $\text{Behavior}(H) = f(\text{Inputs}, S)$.
    \item \textbf{Contradiction:} If the behaviors are identical despite the difference in $S$, then $S$ must not be a necessary input to the function $f$.
    \item \textbf{Violation of Axiom:} If $S$ is not necessary to produce the behavior, then $S$ is \textbf{epiphenomenal} (causally inert). This directly contradicts $A_4$.
\end{enumerate}

\begin{quote}
\textbf{Conclusion:} To maintain the integrity of $A_4$ within the axiomatic system, the simulation $H'$ must experience subjective time and pain.
\end{quote}

\section{Ontological Equivalence of Configurations}

Let $I$ be a simulation encoding a complete observer history (e.g., Alice). 
Let $R_A$, of length $n$ bits, denote the static execution trace of $I$ as observed externally, 
and let $R_B$ denote the same sequence as internally experienced as spacetime and subjective states. 
Static data does not encode an intrinsic temporal ordering; therefore, one cannot argue that one gives rise to the other. 
The relationship between $R_A$ and $I$ must thus be representational rather than causal. 
$R_A$ and $I$ are two arrangements of the same information, consisting of $n$ bits, 
and are therefore capable of describing up to $2^n$ distinct configurations.

From Axioms 2--4 and the optimization argument, no physical property of the simulating substrate, nor the ordering of bits in $R_A$, is ontologically privileged.
Any arrangement of the bits that preserves the relational structure of $I$ encodes the same observer. Formally, there exists a bijective mapping:
\[
\phi : R_A \leftrightarrow R_B
\]
preserving all causally relevant relations within $I$.

It follows that the existence and experiences of the observer depend solely on the internal relational structure of $I$, not on the substrate or external runtime.
Any claim that one substrate or arrangement is ``more real'' than another would require introducing a new, non-physical constant, contradicting the axioms.

\begin{quote}
  \textbf{Conclusion:} The external execution trace and the internal experienced universe are two complementary, equally valid representations of the same underlying information.
\end{quote}

\section{Falsifiability of the Hypothesis}
The hypothesis is falsifiable in the future when technology advances and DNA simulations can be run with sufficient accuracy for DNA-based organisms.
The effect of pain can be measured just like an effect of physical forces can be measured. If a DNA simulation $H'$ is constructed and shown to lack $S$, then $A_4$ is invalidated, and the axioms 1--3 collapse.



\section*{Discussion}

\subsection{The Hard Problem of Consciousness and the Hollow Simulation Objection}

A common objection to functionalist and computational accounts of mind is the so-called \emph{hard problem of consciousness} \cite{Chalmers1996}.
In this context, it is often claimed that a simulated human, even if behaviorally and functionally identical to a biological human, could nevertheless be ``hollow''—that is, it could lack subjective experience while still producing identical outward behavior.

Formally, this objection asserts the possibility of a system $H'$ such that
\[
\text{Behavior}(H') = \text{Behavior}(H) \quad \text{and} \quad S(H') = \emptyset,
\]
where $H$ is a biological human and $S$ denotes subjective experience.

Within the present axiomatic framework, this possibility is excluded. By Axiom~2, the biological human $H$ is a formal axiomatic system whose state transitions are fully determined by physical law and initial conditions. By Axiom~3, there exists a formal representation $H'$ whose execution trace is isomorphic to that of $H$, preserving all causally relevant internal relations. By Axiom~4, subjective experience is a causally efficacious component of the system, contributing to observable behavior.

If $H'$ were to lack subjective experience while remaining behaviorally identical to $H$, then subjective experience would not be a necessary input to the causal function generating behavior. This would render subjective experience epiphenomenal, directly contradicting Axiom~4. 

\subsection{Quantum Computers}

We have considered two thought experiments run on classical computers. If conscious experience can be duplicated on classical computers, we argue it can also be
duplicated on quantum computers.

This follows directly from substrate independence: quantum computation represents an alternative physical realization of the same formal state-transition structure.



\section{Conclusions}

By treating subjective experience ($S$) as a causally efficacious property of an axiomatic system, we move from metaphysical speculation to a rigorous informational mechanics.

\begin{itemize}
    \item \textbf{Substrate Irrelevance:} The universe is fundamentally informational and abstract rather than substrate-dependent.
    \item \textbf{Time:} Time is not a fundamental background parameter ($E_{\text{ext}}$) but emerges solely as an internal property ($E_{\text{int}}$) of the observer. 
    \item \textbf{Emergence from White Noise:} Information that appears random can encode a universe containing conscious observers.
    \item \textbf{Ontological Parity:} There is no privileged ``base reality.'' Every arrangement of bits that preserves the relational structure $\phi$ possesses equal ontological status. The computer and the virtual universe are not causally related, but are different arrangements of the same finite information set. What appears to a global, external observer as a static execution trace—a frozen bit-string—is experienced by the internal structure (Alice) as a dynamic progression and lived experience.
\end{itemize}

\printbibliography
\end{document}
