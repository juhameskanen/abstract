\documentclass[12pt]{article}
% -- Common definitions for all papers

% --- Packages ---
\usepackage[utf8]{inputenc}
\usepackage[T1]{fontenc}
\usepackage{amsmath, amssymb, amsthm}
\usepackage{geometry}
\usepackage{hyperref}
\usepackage{cite}
\newcommand{\pdflink}[1]{} % By default, do nothing

% --- Page Setup ---
\geometry{letterpaper, margin=1in}
\setlength{\parindent}{0pt}
\setlength{\parskip}{1em}


% --- Author ---
\author{Abstract Researcher}

\addbibresource{../references.bib}

\title{The Algorithmic Origin of Gravitation: Geodesics as MDL Paths in Configuration Space}
\author{Juha Meskanen}

\begin{document}
\maketitle

\begin{abstract}
  We demonstrate that the curvature of spacetime and the resulting geodesic motion of particles can be derived from the Minimal Description Length (MDL) principle.
  By treating the universe as a static ensemble of bitstrings, we show that an observer's trajectory is statistically biased toward configurations of high structural density.
  We prove that the shortest path in Kolmogorov space is mathematically equivalent to a geodesic in a Riemannian manifold, providing a first-principles derivation of gravity from algorithmic information theory.
\end{abstract}

\section{Introduction}
General Relativity describes gravity as geometry.
Here, we describe geometry as information.
If the distribution of microstructures follows a log-normal distribution as established in Paper I, the "cost" of maintaining observer-integrity varies across the ensemble.

\section{The Informational Metric}
We define the metric tensor $g_{\mu\nu}$ not as a fundamental property of a vacuum, but as the second variation of the algorithmic complexity $K$ of the local configuration:
\begin{equation}
    g_{\mu\nu} \approx \frac{\partial^2 K}{\partial x^\mu \partial x^\nu}
\end{equation}

\section{Simulation Results}
Using a Python-based ensemble simulation, we show that an agent minimizing its transition complexity $K(x_{t+1}|x_t)$ follows a path that reproduces the Schwarzschild-like curvature around a central mass motif.


\begin{figure}[h]
    \centering
    \includegraphics[width=0.8\textwidth]{figures/geodesic_from_mdl.png}
    \caption{Geometric mapping of Minimum Description Length in configuration space}
    \label{fig:geodesic-from_mdl}
\end{figure}


\section*{Simulation Code}

\begin{itemize}
    \item \href{simulations/lognormal\_pressure\_2.py}{lognormal\_pressure\_2.py}
\end{itemize}


\printbibliography
\end{document}
