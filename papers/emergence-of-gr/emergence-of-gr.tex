\documentclass[11pt]{article}
% -- Common definitions for all papers

% --- Packages ---
\usepackage[utf8]{inputenc}
\usepackage[T1]{fontenc}
\usepackage{amsmath, amssymb, amsthm}
\usepackage{geometry}
\usepackage{hyperref}
\usepackage{cite}
\newcommand{\pdflink}[1]{} % By default, do nothing

% --- Page Setup ---
\geometry{letterpaper, margin=1in}
\setlength{\parindent}{0pt}
\setlength{\parskip}{1em}


% --- Author ---
\author{Abstract Researcher}

\addbibresource{../references.bib}

\title{The Spectral Origin of Gravitation: Geodesics as Minimal Spectral Paths in Configuration Space}
\author{Juha Meskanen}

\begin{document}
\maketitle

\begin{abstract}
We demonstrate that spacetime curvature and particle geodesics can be derived from a spectral encoding of the universe.
By treating the universe as a static ensemble of informational configurations, an observer's trajectory is biased toward paths that admit efficient spectral compression.
The shortest spectral paths correspond to geodesics in a Riemannian manifold, providing a first-principles derivation of gravity from spectral information theory.
\end{abstract}

\section{Introduction}
General Relativity describes gravity as geometry; in our framework, geometry emerges from information. 
Configurations of matter and microstructures correspond to patterns in an abstract configuration space $\mathcal{C}$.
Paths through $\mathcal{C}$ are observer-experienced histories; paths with efficient spectral compression dominate the measure, leading to emergent classical-like physics.
We show that minimizing spectral encoding length along observer-compatible paths reproduces geodesic motion.

\section{The Informational Metric}
The metric tensor $g_{\mu\nu}$ is not fundamental but arises from variations in spectral encoding length:
\begin{equation}
    g_{\mu\nu} \approx \frac{\partial^2 \mathcal{L}}{\partial x^\mu \partial x^\nu},
\end{equation}
where $\mathcal{L}$ is the Minimal Spectral Description of the local configuration.

\subsection{Equivalence of Minimal Spectral Paths and Geodesics}
Define the infinitesimal spectral distance $d\mathcal{L}$ between consecutive states along an observer path $\gamma$:
\begin{equation}
    S_{\text{Spectral}} = \int_{\gamma} \mathcal{L}(\text{state}_{t+dt} | \text{state}_t).
\end{equation}
Using the correspondence between second derivatives of spectral encoding and the Fisher information metric $g_{ij}$:
\begin{equation}
    S_{\text{Spectral}} \approx \int \sqrt{g_{\mu\nu} \dot{x}^\mu \dot{x}^\nu} dt.
\end{equation}
The path that minimizes spectral description is mathematically identical to a geodesic in Riemannian geometry:
\begin{equation}
    \frac{d^2 x^\mu}{ds^2} + \Gamma^\mu_{\alpha\beta} \frac{dx^\alpha}{ds} \frac{dx^\beta}{ds} = 0,
\end{equation}
with $\Gamma$ representing gradients in microstructure density.

\section{Simulation Results}
A Python-based ensemble simulation demonstrates that agents minimizing $\mathcal{L}(x_{t+1}|x_t)$ follow Schwarzschild-like trajectories around central motif clusters.

\begin{figure}[h]
    \centering
    \includegraphics[width=0.8\textwidth]{figures/geodesic_from_spectral.png}
    \caption{Minimal Spectral Path mapping in configuration space}
    \label{fig:geodesic-from-spectral}
\end{figure}

\begin{figure}[h]
    \centering
    \includegraphics[width=0.8\textwidth]{figures/geodesic_from_spectral_memory.png}
    \caption{Observer with memory minimizing spectral description of state transitions}
    \label{fig:geodesic-from-spectral-memory}
\end{figure}

\section*{Simulation Code}
\begin{itemize}
    \item \href{simulations/geodesic_from_spectral.py}{geodesic\_from\_spectral.py}
    \item \href{simulations/geodesic_from_spectral_memory.py}{geodesic\_from\_spectral\_memory.py} 
\end{itemize}

\section{Planck Scale and the Nyquist Analogy}
\subsection{The Planck Scale as the Nyquist Frequency}
In signal processing, the Nyquist Frequency is the maximum frequency resolvable for a given sampling rate. Analogously:

\begin{itemize}
    \item \textbf{Sampling Rate:} Rate of bit-flips in configuration space $\mathcal{C}$.
    \item \textbf{Planck Limit:} Maximum frequency representable in the spectral wavefunction $\Psi$.
    \item \textbf{Implication:} Features smaller than Planck length or faster than Planck time are aliased and cannot exist for observers.
\end{itemize}

\subsection{Discrete Time from Spectral Updates}
Configurations contain $n$ bits. Transitions $s_i \to s_{i+1}$ in a path $\gamma$ involve bit-flips. Spectral encoding $\mathcal{L}$ only updates when significant motif changes occur. Observer time emerges from these discrete spectral updates.

\subsection{Spectral Density and the Inverse-Square Law}
The log-normal distribution of microstructure motifs naturally produces $1/r^2$ scaling:

\begin{itemize}
    \item \textbf{Source:} Clusters of reusable motifs represent masses.
    \item \textbf{Radial Distribution:} Configurations that reference the cluster increase with sphere surface area: $A=4\pi r^2$.
    \item \textbf{Consequence:} Probability pressure toward motifs falls as $1/r^2$, reproducing classical gravitational scaling.
\end{itemize}

\section*{Supplementary Materials}

\begin{itemize}
\item \href{supplementary-plank-scale-and-spectral-gravity.pdf}{Plank scale and Spectral Gravity}
\end{itemize}



\printbibliography
\end{document}
