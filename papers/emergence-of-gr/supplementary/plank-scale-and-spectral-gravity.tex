\documentclass[11pt]{article}
% -- Common definitions for all papers

% --- Packages ---
\usepackage[utf8]{inputenc}
\usepackage[T1]{fontenc}
\usepackage{amsmath, amssymb, amsthm}
\usepackage{geometry}
\usepackage{hyperref}
\usepackage{cite}
\newcommand{\pdflink}[1]{} % By default, do nothing

% --- Page Setup ---
\geometry{letterpaper, margin=1in}
\setlength{\parindent}{0pt}
\setlength{\parskip}{1em}


% --- Author ---
\author{Abstract Researcher}

% Glossary
\newcommand{\SSP}{Spectral Selection Principle (SSP)}
\newcommand{\Lcost}{$\mathcal{L}$} % Added $ $ here
\newcommand{\Tobs}{$\mathcal{T}_{\mathrm{obs}}$} % Added $ $ here

\newcommand{\PrintGlossary}{
    \section*{Glossary of the Abstract Universe}
    \begin{description}
    \item[SSP] \SSP: The foundational principle asserting that reality is the subset of informational paths that admit the most efficient representation in the frequency domain. It replaces "laws of physics" with a statistical preference for compressibility.
    
    \item[\Tobs] Observer Filter

    \item[$\mathcal{L}$ (Spectral Encoding Length)] The measure of information required to represent a state or path $\gamma$. In this framework, $\mathcal{L}$ replaces the classical concept of \textit{Action}. Minimizing $\mathcal{L}$ is equivalent to the Principle of Least Action.
    
    \item[$\mathcal{T}_{\mathrm{obs}}$ (Observer Filter)] The subset of all possible configuration sequences capable of describing a subjective experience.
    
    \item[MSM (Microstructure Motif)] Recurring, compressible bit-patterns within configuration space (e.g., particles, fields). The density of these motifs determines the local spectral cost, manifesting as mass-energy.
    
    \item[Nyquist Horizon] The "Planck Scale" of the theory. It is the maximum frequency supportable by the discretization of configuration space. Any structure below this limit is mathematically "aliased" and cannot be rendered.
    
    \item[Induced Time] The sequential ordering of states along a path $\gamma$. An emergent property of the observer's trajectory.
    
    \item[Phase-Coherence] The informational alignment between disparate parts of the spectral encoding. This provides the mathematical basis for what is traditionally called \textit{quantum entanglement}.

    \end{description}
}


\addbibresource{../../references.bib}

\title{Planck Scale and Spectral Gravity}
\author{Juha Meskanen}

\begin{document}
\maketitle


\section{The Planck Scale as the Nyquist Frequency}  
In signal processing, the Nyquist frequency \(f_\mathrm{Nyquist} = f_s/2\) is the highest frequency that can be resolved given a sampling rate \(f_s\). Frequencies above \(f_\mathrm{Nyquist}\) are aliased and indistinguishable.  

In the Abstract Universe:

\begin{itemize}
    \item \textbf{Sampling Rate:} The rate of bit-flips along observer-compatible paths in configuration space \(\mathcal{C}\).  
    \item \textbf{Planck Limit:} The highest frequency resolvable in the spectral wavefunction \(\Psi(\gamma)\).  
    \item \textbf{Consequence:} Structures smaller than the Planck length or evolving faster than the Planck time cannot be resolved by \(\Psi\); they are mathematically unrepresentable and effectively non-existent for any observer.
\end{itemize}

\section{Discrete Time from Spectral Updates}  
Let \(\gamma = (s_1, s_2, \dots)\) be an observer path through \(\mathcal{C}\), with each \(s_i\) an \(n\)-bit configuration.  

Define a spectral encoding length \(\mathcal{L}(\gamma)\). Updates to \(\mathcal{L}\) occur only when significant motifs (observer-relevant patterns) change along the path.  

\[
t_{i+1} - t_i = 
\begin{cases}
1, & \text{if a motif update occurs} \\
0, & \text{otherwise}
\end{cases}
\]

Thus, \textbf{time emerges discretely}, with “ticks” corresponding to spectral transitions along observer-compatible paths.  

\section{Spectral Density and the Inverse-Square Law}  
Let a “mass” be a localized cluster of reusable microstructure motifs in configuration space. The number of configurations at distance \(r\) that can reference these motifs grows with the surface area of a 3D sphere:

\[
A(r) = 4 \pi r^2
\]

Consequently, the density of spectral-compatible configurations falls as:

\[
\rho(r) \propto \frac{1}{r^2} \, L(d)
\]

where \(L(d)\) is the log-normal distribution of motif distances \(d\).  

Observer paths minimize spectral cost \(\mathcal{L}(\gamma)\) and are biased toward regions of high \(\rho(r)\). In the continuum limit, the variational principle

\[
\delta \int_\gamma \mathcal{L}(\text{state}) \, ds = 0
\]

reproduces classical geodesics in 3D space, giving rise to an effective inverse-square law of interaction.  

\section{Connection to Quantum Gravity and Path Integrals}  
Summing over all observer-compatible paths weighted by their spectral amplitude:

\[
\langle \mathcal{O} \rangle = \sum_{\gamma \in \mathcal{T}_{\mathrm{obs}}} \mathcal{O}(\gamma) \, 2^{-\mathcal{L}(\gamma)}
\]

reproduces a path-integral formalism. In this framework:

\begin{itemize}
    \item \textbf{Everettian view:} All observer-compatible paths exist simultaneously; probabilities emerge from spectral weighting.  
    \item \textbf{Wheeler–DeWitt view:} Configuration space is timeless; time is induced along the observer path \(\gamma\).  
    \item \textbf{Classical limit:} Low-\(\mathcal{L}\) paths dominate, yielding smooth geodesics in agreement with General Relativity.
\end{itemize}

\section{Summary}  
The Planck scale, discrete time, and gravity emerge naturally from:

\begin{enumerate}
    \item The maximal resolvable spectral frequency (Nyquist-Planck analogy).  
    \item Discrete spectral updates along observer paths.  
    \item Statistical bias of observer paths toward high-density microstructure regions, producing effective inverse-square laws.  
\end{enumerate}

In this view, gravitation is not a fundamental force but a statistical-geometric phenomenon rooted in the spectral structure of observer-compatible paths in configuration space.

\printbibliography
\end{document}
