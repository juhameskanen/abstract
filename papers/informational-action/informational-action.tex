\documentclass[11pt]{article}
% -- Common definitions for all papers

% --- Packages ---
\usepackage[utf8]{inputenc}
\usepackage[T1]{fontenc}
\usepackage{amsmath, amssymb, amsthm}
\usepackage{geometry}
\usepackage{hyperref}
\usepackage{cite}
\newcommand{\pdflink}[1]{} % By default, do nothing

% --- Page Setup ---
\geometry{letterpaper, margin=1in}
\setlength{\parindent}{0pt}
\setlength{\parskip}{1em}


% --- Author ---
\author{Abstract Researcher}

% Glossary
\newcommand{\SSP}{Spectral Selection Principle (SSP)}
\newcommand{\Lcost}{$\mathcal{L}$} % Added $ $ here
\newcommand{\Tobs}{$\mathcal{T}_{\mathrm{obs}}$} % Added $ $ here

\newcommand{\PrintGlossary}{
    \section*{Glossary of the Abstract Universe}
    \begin{description}
    \item[SSP] \SSP: The foundational principle asserting that reality is the subset of informational paths that admit the most efficient representation in the frequency domain. It replaces "laws of physics" with a statistical preference for compressibility.
    
    \item[\Tobs] Observer Filter

    \item[$\mathcal{L}$ (Spectral Encoding Length)] The measure of information required to represent a state or path $\gamma$. In this framework, $\mathcal{L}$ replaces the classical concept of \textit{Action}. Minimizing $\mathcal{L}$ is equivalent to the Principle of Least Action.
    
    \item[$\mathcal{T}_{\mathrm{obs}}$ (Observer Filter)] The subset of all possible configuration sequences capable of describing a subjective experience.
    
    \item[MSM (Microstructure Motif)] Recurring, compressible bit-patterns within configuration space (e.g., particles, fields). The density of these motifs determines the local spectral cost, manifesting as mass-energy.
    
    \item[Nyquist Horizon] The "Planck Scale" of the theory. It is the maximum frequency supportable by the discretization of configuration space. Any structure below this limit is mathematically "aliased" and cannot be rendered.
    
    \item[Induced Time] The sequential ordering of states along a path $\gamma$. An emergent property of the observer's trajectory.
    
    \item[Phase-Coherence] The informational alignment between disparate parts of the spectral encoding. This provides the mathematical basis for what is traditionally called \textit{quantum entanglement}.

    \end{description}
}


\addbibresource{../references.bib}

\title{Recovering Classical Geodesics from Spectral Entropy Minimization}

\date{2001}

\begin{document}

\maketitle

\begin{abstract}
We present a constructive proof-of-concept demonstrating that classical geodesic paths on a curved surface can be recovered by minimizing an informational cost functional defined over discrete execution traces.
Specifically, we compare Euclidean action minimization with minimization of spectral entropy computed from local bit-encoded wavefunction segments.
Without assuming any prior equivalence between geometric and informational principles, we show that both methods select nearly identical paths across a range of noisy, nontrivial geometries.
This result supports the hypothesis that classical spacetime dynamics may emerge from an underlying informational principle.
\end{abstract}

\section{Introduction}
The principle of least action occupies a foundational role in classical and quantum physics.
In Euclidean quantum gravity, the action functional governs the weighting of paths in a path integral formulation, while in classical general relativity geodesics extremize proper distance or action.
In parallel, information-theoretic approaches to physics have long suggested that entropy, complexity, or information content may play a fundamental role in physical law.
However, most prior work draws conceptual or variational analogies rather than providing explicit constructive equivalence.

In this work, we investigate whether a purely informational cost functional, defined independently of geometry, can recover classical geodesic behavior.
Our approach is algorithmic and discrete: paths are selected step-by-step by minimizing either Euclidean action or spectral entropy over local execution traces.
Remarkably, we find strong agreement between the two resulting paths.

\section{Geometric Setup}
We consider a two-dimensional spatial domain embedded in a three-dimensional height field, representing a curved metric surface.
The surface is constructed as a superposition of Gaussian hills and valleys with added stochastic noise.
Let $\phi(x,y)$ denote the surface height.
The classical Euclidean action for a path $\gamma$ proceeding in the $x$ direction is approximated locally by
\begin{equation}
I_E \approx \sum_x \left( \nabla_x \phi \right)^2,
\end{equation}
where the gradient is discretized on a grid.

At each step, the next path position is chosen from a local neighborhood to minimize the incremental contribution to $I_E$.

\section{Informational Cost Functional}
Independently, we define an informational cost based on spectral entropy.
For each candidate path segment, the local surface values are encoded into a binary sequence by thresholding relative to a local mean.
This bitstring is interpreted as a discrete wavefunction, from which a power spectral density is computed via a real Fourier transform.

The spectral entropy is then defined as
\begin{equation}
H = -\sum_k p_k \log_2 p_k,
\end{equation}
where $p_k$ is the normalized power in frequency mode $k$. Lower entropy corresponds to more compressible, smoother waveforms.

To ensure continuity between steps, we introduce a coherence penalty based on overlap between consecutive spectral amplitude vectors.
The total informational cost is the sum of spectral entropy and a weighted coherence term.

\section{Path-Finding Algorithm}
Both the Euclidean and informational methods employ the same local path-finding structure.
Starting from a fixed boundary condition, the path advances column by column.
At each step, three candidate positions are evaluated, and the one minimizing the respective cost functional is selected.

Importantly, no geometric information is used in the informational method beyond the raw surface samples; likewise, no informational quantities appear in the Euclidean action.

\section{Results}
Across multiple randomly generated surfaces and noise realizations, we observe that the paths selected by Euclidean action minimization and spectral entropy minimization are nearly identical.
Minor deviations occur primarily near regions of high curvature or noise, but the global path structure remains consistent.

Heatmap visualizations of the accumulated action and entropy further reveal correlated cost landscapes, with minima aligned along similar valleys of the surface.

\begin{figure}[h!]
\centering
\includegraphics[width=0.7\textwidth]{figures/path-comparison.png}
\caption{Paths}
\end{figure}

\begin{figure}[h!]
\centering
\includegraphics[width=0.7\textwidth]{figures/fields.png}
\caption{Heatmaps }
\end{figure}


\section{Discussion}
The observed correspondence suggests that classical geodesics may be understood as paths of minimum informational cost, where information is quantified in terms of spectral complexity of local execution traces.
Unlike prior entropic or informational interpretations of gravity, this result is constructive and algorithmic.

While this work does not constitute a theory of quantum gravity or a unification framework, it provides concrete evidence that geometric action and
informational measures can coincide in nontrivial settings. This supports the broader hypothesis that spacetime dynamics may emerge from informational principles.

\section{Limitations and Future Work}
The present study is limited to two-dimensional surfaces and heuristic discretizations.
Future work should extend the method to full two-dimensional geodesics, higher-dimensional manifolds, and continuous limits.
A formal analytical relationship between Euclidean action and spectral entropy also remains an open question.

\section{Conclusion}
We have demonstrated that minimizing spectral entropy over discrete execution traces can recover classical geodesic paths on curved surfaces.
This result strengthens the case for information-theoretic foundations of physical law and motivates further investigation into informational formulations of action principles.

\section*{Acknowledgments}
The author thanks collaborators and open-source contributors whose tools made this work possible.


\section*{Supplementary Materials}

\begin{itemize}
\item \href{simulations/phi4\_qbitwave\_mcmc\_2.py}{phi4\_qbitwave\_mcmc\_2.py: UV suppression in Quantum Field Theory}
\item \href{simulations/entropy\_action\_comparison.py}{entropy\_action\_comparison.py}
\item \href{supplementary-qbitwave.pdf}{QBitwave supplementary paper}
\end{itemize}


\end{document}
