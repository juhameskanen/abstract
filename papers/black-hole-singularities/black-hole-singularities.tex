\documentclass[11pt]{article}
% -- Common definitions for all papers

% --- Packages ---
\usepackage[utf8]{inputenc}
\usepackage[T1]{fontenc}
\usepackage{amsmath, amssymb, amsthm}
\usepackage{geometry}
\usepackage{hyperref}
\usepackage{cite}
\newcommand{\pdflink}[1]{} % By default, do nothing

% --- Page Setup ---
\geometry{letterpaper, margin=1in}
\setlength{\parindent}{0pt}
\setlength{\parskip}{1em}


% --- Author ---
\author{Abstract Researcher}

\addbibresource{../references.bib}

\title{Black Hole Singularities as Zero Entropy States}

\date{2010}

\begin{document}
\maketitle


\begin{abstract}
Building on the ontological equivalence of physical and informational configurations established in Paper I, we investigate black hole singularities from an information-theoretic perspective, viewing geometric spacetime and computational simulation as two equivalent interpretations of an underlying informational structure.

Standard General Relativity (GR) cannot be executed to the point of singularity formation due to numerical divergences and the breakdown of the manifold description. However, by analyzing the simulation’s execution trace of infalling dust particles, we observe a monotonic drop in Shannon entropy as they approach the center. While the geometric description of GR fails, the informational trend remains continuous. Extrapolating this entropy decay suggests that the singularity is a state of zero Shannon entropy, corresponding to a single, well-defined informational state. We propose that the "singularity" is not a point of infinite curvature, but a state of informational exhaustion where the lack of internal variety prevents the emergence of the microstructures we interpret as matter and energy. This perspective allows the singularity to be treated as a smooth, finite limit within an informational manifold, bypassing the limitations of standard GR.
\end{abstract}

\ifdefined\ishtml
\begin{center}
\small \href{black-hole-singularities.pdf}{Download PDF Version}
\end{center}
\fi


\section{Introduction}

We apply a computational and information-theoretic framework to black hole singularities. 
The guiding hypothesis is that spacetime geometry and the execution of a simulation 
can be viewed as complementary descriptions of the same underlying informational structure.

Our central question is whether approaching singularities as information-theoretic phenomena 
provides new insights into their structure and physical interpretation. To explore this, 
we develop both analytic reasoning and numerical simulations of collapsing dust clouds.

\section{Methods}

We model black hole evolution through computational simulation. 
We compute the \emph{execution-trace entropy} of the discretized particle positions, 
which captures the number of distinguishable geometric configurations encoded in memory.

\subsection{Execution Trace}

Let \(\mathcal{M}\) denote the set of all memory locations in a deterministic computing system. 
A \emph{machine state} \(\mathcal{S} \in \mathcal{M}\) is a complete assignment of values 
to all memory elements. Let \(\mathcal{P} = (I_1, I_2, \dots, I_n)\) be a finite sequence of deterministic instructions, 
with \(I_k : \mathcal{S} \rightarrow \mathcal{S}\). The \emph{execution trace} \(\mathcal{T}\) of program \(\mathcal{P}\) 
is the ordered sequence of states
\[
\mathcal{T} = (s_0, s_1, \dots, s_n), \quad s_{k+1} = I_{k+1}(s_k).
\]

The execution trace encodes all information about the simulated phenomenon. 
In practice, we focus on the \emph{bitstring sub-trace} that encodes spatial geometry.

\subsection{Mapping Geometry to Bitstrings}

Let a geometric state be encoded as a bitstring \(b \in \{0,1\}^L\), representing the discretized positions 
of all particles in the dust cloud. Let
\[
C = \{0,1\}^{3k}
\]
be the space of 3D configurations. Define
\[
f : C \to \mathbb{Z}^3, \quad f(b) = (\phi(b_1), \phi(b_2), \phi(b_3)),
\]
where \(\phi : \{0,1\}^k \to \mathbb{Z}\) decodes fixed-length binary segments. 
Quantization ensures that the encoding reflects true positional differences, 
minimizing spurious entropy from floating-point representations.

\subsection{Execution-Trace Entropy}

To quantify distinguishable configurations, we compute the Shannon entropy of the bitstring representation:

\[
H(b) = -p_0 \log_2 p_0 - p_1 \log_2 p_1,
\]

where \(p_0\) and \(p_1\) are the empirical frequencies of 0- and 1-bits in \(b\). 

Over a sequence of states (the bitstring sub-trace), the entropy trajectory reflects the collapse of distinguishable geometric states as particles fall toward the singularity.

\section{Simulations}

We simulated both Schwarzschild (non-rotating) and Kerr (rotating) black holes. 
All simulations show a monotonic decrease in execution-trace entropy as particles fall inward. 
Extrapolation indicates that the singularity corresponds to zero entropy state.

\begin{figure}[h!]
\centering
\includegraphics[width=0.7\textwidth]{figures/schwarzschild.png}
\caption{Execution-trace entropy of particles falling into a Schwarzschild black hole.}
\end{figure}

\begin{figure}[h!]
\centering
\includegraphics[width=0.7\textwidth]{figures/kerr.png}
\caption{Execution-trace entropy of particles falling into a Kerr black hole.}
\end{figure}

\subsection{Lemma: Vanishing Execution-Trace Entropy Implies Geometric Singularity}

Let \(\mathcal{B}_t \subset \{0,1\}^L\) denote the set of bitstrings encoding geometry over a finite time window near \(t\). 
If
\[
H(\mathcal{B}_t) \to 0,
\]
then the set of distinguishable geometric configurations collapses to a single equivalence class under \(f\). That is, the geometry degenerates to a unique configuration.

\section{Discussion: The Divergence of Geometry vs. The Convergence of Information}

Our simulations confirm the Entropy–Singularity Lemma in a computational context. 
Although global black hole thermodynamic entropy increases, the local execution-trace entropy encoding spatial geometry decreases toward the singularity. 
This reflects the narrowing of accessible informational states and growing determinism in geometry.

The primary conflict between the results presented here and the standard consensus of General Relativity (GR) lies in the interpretation of the singularity.
In the geometric framework of GR, the singularity is characterized by infinite curvature---a mathematical divergence where the manifold itself ceases to exist.
However, our analysis of the simulation's execution trace reveals an alternative, non-pathological end-state.

\subsection{Divergence as a Modeling Artifact}

In standard physics, a divergence is rarely treated as a physical reality; rather, it is viewed as a signal that the underlying model has reached the limit of its applicability.
We argue that the "infinite curvature" predicted by GR is a consequence of treating spacetime as a continuous, differentiable manifold down to zero volume.
By contrast, the informational mapping used in our simulation treats the system as a collection of discrete states. 

As the infalling particles converge, the number of possible microstates $W$ decreases. In the limit, the system reaches a singular state where $W=1$.
At this point, the Shannon entropy, defined as:

\begin{equation}
    H = -\sum_{i=1}^{n} p_i \log p_i
\end{equation}

collapses to exactly zero. 

\subsection{The Faithfulness of the Execution Trace}
A common critique of computational models is that they are merely approximations of continuous physics. However, we contend that the \textit{execution trace} of a simulation
is a faithful representation of the system's logic. Every bit in the simulation represents a physical degree of freedom, and every instruction cycle represents an
ordinal step in the system's evolution. 

Because the simulation can successfully map the transition from high-entropy (distributed particles) to zero-entropy (a single geometric point) without numerical
overflow or logical breakdown, it provides a more robust description of the "singularity" than the equations that diverge.
The simulation does not break; it simply reaches its most compressed, minimal-information state.

\subsection{Redefining the Singularity as Informational Exhaustion}
We propose that the singularity is not a point of infinite gravity, but a state of \textbf{informational exhaustion}. In this state, the absence of internal
variety means there is no information left to describe curvature, matter, or time.
This results in a smooth end-state: a singular, well-defined point in the informational ensemble that is perfectly consistent with the rest of the manifold,
but lacks the bit-depth to manifest as physical complexity. 

This suggests that the "breakdown" of spacetime is not a physical catastrophe, but a phase transition into a state of absolute informational simplicity,
providing a finite limit where GR erroneously predicts infinity.


\subsection{Bitstring Encoding Procedure}

To encode geometry:

\begin{enumerate}
    \item Quantize continuous coordinates: \(\mathbb{R} \to \mathbb{Z}\)
    \item Convert integers to fixed-width binary strings
    \item Concatenate binary segments into a full state bitstring
\end{enumerate}

Entropy is then computed as above. The collapse of this entropy directly reflects the loss of distinguishable structure.

\section{Conclusion}

Singularities might not be pathological or broken regions of spacetime. Instead, they are perfectly well-defined, smooth geometric
objects—points—precisely because the system has exhausted the informational capacity required to support any complex structure, volume, or
fluctuation. By viewing the breakdown of spacetime through an information-theoretic lens, we replace physical divergence with
informational convergence, offering a bridge between computational reality and gravitational collapse.

\section*{Simulation Code}

\begin{itemize}
    \item \href{simulations/schwarzschild.py}{schwarzschild.py: Schwarzschild black hole simulation}
    \item \href{simulations/kerr.py}{kerr.py: Kerr black hole simulation}
    \item \href{simulations/blackhole.py}{blackhole.py: Common base classes}
\end{itemize}
   
\printbibliography
\end{document}
