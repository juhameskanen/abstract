\documentclass[12pt]{article}
\usepackage{amsmath, amssymb, amsfonts}
\usepackage{graphicx}
\usepackage{hyperref}
\usepackage{geometry}
\geometry{margin=1in}

\title{Black Hole Singularities as Zero Entropy States}
\author{Juha Meskanen}
\date{2019 ... 2025}

\begin{document}
\maketitle

\begin{abstract}
We investigate black hole singularities from an information-theoretic perspective,
viewing geometric spacetime and computational simulation as two equivalent interpretations
of an underlying informational structure. We introduce the notion of execution-trace entropy,
the Shannon entropy of a fixed encoding of the simulated spatial configuration, and show that
it decreases monotonically as particles approach the singularity. Our results indicate that singularities
should be understood as well-defined, smooth informational fixed points corresponding to zero-entropy states.
Infinities signal overcounting of informationally indistinguishable configurations; when physical measure is 
defined by information content, singularities and divergences collapse into smooth zero-measure limits.
\end{abstract}

\section{Introduction}

We apply a computational and information-theoretic framework to black hole singularities. 
The guiding hypothesis is that spacetime geometry and the execution of a simulation 
can be viewed as complementary descriptions of the same underlying informational structure.

Our central question is whether approaching singularities as information-theoretic phenomena 
provides new insights into their structure and physical interpretation. To explore this, 
we develop both analytic reasoning and numerical simulations of collapsing dust clouds.

\section{Methods}

We model black hole evolution through computational simulation. 
We compute the \emph{execution-trace entropy} of the discretized particle positions, 
which captures the number of distinguishable geometric configurations encoded in memory.

\subsection{Execution Trace}

Let \(\mathcal{M}\) denote the set of all memory locations in a deterministic computing system. 
A \emph{machine state} \(\mathcal{S} \in \mathcal{M}\) is a complete assignment of values 
to all memory elements. Let \(\mathcal{P} = (I_1, I_2, \dots, I_n)\) be a finite sequence of deterministic instructions, 
with \(I_k : \mathcal{S} \rightarrow \mathcal{S}\). The \emph{execution trace} \(\mathcal{T}\) of program \(\mathcal{P}\) 
is the ordered sequence of states
\[
\mathcal{T} = (s_0, s_1, \dots, s_n), \quad s_{k+1} = I_{k+1}(s_k).
\]

The execution trace encodes all information about the simulated phenomenon. 
In practice, we focus on the \emph{bitstring sub-trace} that encodes spatial geometry.

\subsection{Mapping Geometry to Bitstrings}

Let a geometric state be encoded as a bitstring \(b \in \{0,1\}^L\), representing the discretized positions 
of all particles in the dust cloud. Let
\[
C = \{0,1\}^{3k}
\]
be the space of 3D configurations. Define
\[
f : C \to \mathbb{Z}^3, \quad f(b) = (\phi(b_1), \phi(b_2), \phi(b_3)),
\]
where \(\phi : \{0,1\}^k \to \mathbb{Z}\) decodes fixed-length binary segments. 
Quantization ensures that the encoding reflects true positional differences, 
minimizing spurious entropy from floating-point representations.

\subsection{Execution-Trace Entropy}

To quantify distinguishable configurations, we compute the Shannon entropy of the bitstring representation:

\[
H(b) = -p_0 \log_2 p_0 - p_1 \log_2 p_1,
\]

where \(p_0\) and \(p_1\) are the empirical frequencies of 0- and 1-bits in \(b\). 

\textbf{Important clarification:} This entropy is \emph{not} thermodynamic entropy or ensemble Shannon entropy. 
It measures the number of distinguishable configurations encoded in a deterministic simulation at a given timestep.

Over a sequence of states (the bitstring sub-trace), the entropy trajectory reflects the collapse of distinguishable geometric states as particles fall toward the singularity.

\section{Simulations}

We simulated both Schwarzschild (non-rotating) and Kerr (rotating) black holes. 
All simulations show a monotonic decrease in execution-trace entropy as particles fall inward. 
Extrapolation indicates that entropy approaches zero at the singularity.

\begin{figure}[h!]
\centering
\includegraphics[width=0.7\textwidth]{figures/schwarzschild.png}
\caption{Execution-trace entropy of particles falling into a Schwarzschild black hole.}
\end{figure}

\begin{figure}[h!]
\centering
\includegraphics[width=0.7\textwidth]{figures/kerr.png}
\caption{Execution-trace entropy of particles falling into a Kerr black hole.}
\end{figure}

\subsection{Lemma: Vanishing Execution-Trace Entropy Implies Geometric Singularity}

Let \(\mathcal{B}_t \subset \{0,1\}^L\) denote the set of bitstrings encoding geometry over a finite time window near \(t\). 
If
\[
H(\mathcal{B}_t) \to 0,
\]
then the set of distinguishable geometric configurations collapses to a single equivalence class under \(f\). That is, the geometry degenerates to a unique configuration.

\section{Discussion}

Our simulations confirm the Entropy–Singularity Lemma in a computational context. 
Although global black hole thermodynamic entropy increases, the local execution-trace entropy encoding spatial geometry decreases toward the singularity. 
This reflects the narrowing of accessible informational states and growing determinism in geometry.

\subsection{Singularities as Smooth Informational Fixed Points}

The term ``singularity'' does not denote divergent curvature or ill-defined geometry in our framework. 
Instead, it denotes a collapse of distinguishable geometric degrees of freedom. 
The zero-entropy limit corresponds to a unique, maximally symmetric equivalence class. 
This state is smooth (no relational variation) but singular in that it supports no internal structure, particles, or local observables.

\subsection{Bitstring Encoding Procedure}

To encode geometry:

\begin{enumerate}
    \item Quantize continuous coordinates: \(\mathbb{R} \to \mathbb{Z}\)
    \item Convert integers to fixed-width binary strings
    \item Concatenate binary segments into a full state bitstring
\end{enumerate}

Entropy is then computed as above. The collapse of this entropy directly reflects the loss of distinguishable structure.

\section{Conclusion}

Black hole simulations cannot directly resolve classical singularities due to numerical instability and breakdown of general relativity. 
However, execution-trace entropy analysis reveals that singularities correspond to zero-entropy states: well-defined, smooth informational fixed points with no internal structure. 
This reframes singularities as informational phenomena rather than divergent densities, supporting the broader information-theoretic interpretation of spacetime.

\section*{Simulation Code}

Source code is available for:

\begin{itemize}
    \item \href{simulations/schwarzschild.py}{Schwarzschild black hole simulation}
    \item \href{simulations/kerr.py}{Kerr black hole simulation}
    \item \href{simulations/blackhole.py}{Common base classes}
\end{itemize}
   

\end{document}
