\documentclass[11pt]{article}

% --- Packages ---
\usepackage[utf8]{inputenc}
\usepackage[T1]{fontenc}
\usepackage{amsmath, amssymb, amsthm}
\usepackage{geometry}
\usepackage{graphicx}
\usepackage{hyperref}

% --- Page Setup ---
\geometry{letterpaper, margin=1in}
\setlength{\parindent}{0pt}
\setlength{\parskip}{1em}

% --- Document Metadata ---
\title{\textbf{Spacetime Curvature at the Singularity}}
\author{Juha Meskanen}
\date{December 2025}

\begin{document}

\maketitle

\section*{Abstract}
This paper explores the hypothesis that the interior geometry of a black hole is not a pathological singularity but a smooth well-defined point. We argue that if a zero-entropy state is information-free, it cannot define individual particles. This concept has profound implications for understanding spacetime curvature at the black hole's core.

\section{Introduction}
The standard Schwarzschild solution predicts infinite curvature at the singularity. To reconcile these views, we replace the interior solution with a lognormal-shaped distribution of effective curvature, anchored at zero entropy. 

In the semi-classical gravity framework, the Einstein field equations are
\[
G_{\mu\nu} = 8 \pi G \, \langle \hat{T}_{\mu\nu} \rangle,
\]
where $\langle \hat{T}_{\mu\nu} \rangle$ is the expectation value of the stress-energy tensor in a given quantum state. 

If the singularity corresponds to a zero-entropy state --- i.e., no distinguishable microstates, no matter, and no structures --- then
\[
\langle \hat{T}_{\mu\nu} \rangle = 0,
\]
and the Einstein equations reduce to the vacuum form,
\[
G_{\mu\nu} = 0,
\]
describing a well-defined geometry.

\begin{figure}[h]
    \centering
    \includegraphics[width=0.8\textwidth]{figures/schwarzschild_lognormal.png}
    \caption{Schwarzschild metric as a probabilistic particle distribution: probability density vanishes at the singularity.}
    \label{fig:schwarzschild-lognormal}
\end{figure}

\section{Conclusions}
This construction is admittedly arbitrary in form, but it has the virtue of dissolving the singularity, preserving information, and yielding well-behaved particle trajectories inside the horizon.

The lognormal distribution suggests that the probability of particles existing rises sharply just outside this zero-entropy state. This means there is a massive concentration of information in the region immediately surrounding the singularity. This could be a testable prediction for a future theory of quantum gravity.

Unlike black holes, where the interior is hidden behind an event horizon, the primordial singularity of our universe is observable indirectly. The early universe leaves its imprint on the cosmic microwave background (CMB) and large-scale structure. From an information-theoretic perspective:
\begin{itemize}
    \item The Big Bang corresponds to a zero-entropy state, where no emergent particles or structures exist.
    \item As entropy increases, geometric structure unfolds, producing particles whose abundances follow a lognormal distribution.
    \item The probability distribution of curvature or matter density peaks at intermediate scales, reflecting the mode of the lognormal distribution.
    \item These statistical peaks could, in principle, leave observable signatures in the CMB, encoding information about the early universe’s geometry.
\end{itemize}

Thus, while black hole interiors remain inaccessible, the cosmological singularity provides a natural laboratory for testing predictions of the information-theoretic framework.

\begin{figure}[h]
    \centering
    \includegraphics[width=0.8\textwidth]{figures/cmb.png}
    \caption{Emergence of Structures from Increasing Entropy (Lognormal Distribution).}
    \label{fig:cmb}
\end{figure}



\section{Future Work}
Fully probabilistic information theoretical model of black hole, based on semiclassical-gravity:
\begin{itemize}
    \item Entropy relative to surface area of the event horizon.
    \item \href{einstein_as_probability.pdf}{Einstein Equation as Probability Law}.
\end{itemize}

\section*{Simulation Code}
The implementation is available in \texttt{simulations/schwarzschild\_lognormal.py}.

\end{document}
